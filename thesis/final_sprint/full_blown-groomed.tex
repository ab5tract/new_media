%\enableregime[utf]  % use UTF-8

\setupcolors[state=start]
\setupinteraction[state=start, color=black] % needed for hyperlinks

\usemodule[simplefonts]
\setmainfont[linlibertineo]
\setmonofont[inconsolata][11pt]

\setuppapersize[A4][A4]  % use A4 paper
%\setuplayout[width=middle, backspace=1.5in, cutspace=1.5in,
%             height=middle, header=0.75in, footer=0.75in] % page layout
%\setuppagenumbering[alternative=doublesided, location={header,margin}]  % number pages
\setuppagenumbering[location=]
\setupbodyfont[12pt]  % 12pt font
\setupwhitespace[medium]  % inter-paragraph spacing

\setupinterlinespace[line=18pt] % should be 3/2 spacing

\setupcaptions[headstyle=smallcaps]

\setupindenting[medium]
\indenting[always]

\setuphead[section][style=\sc, alternative=margin]
\setuphead[subsection][style=\sc, alternative=margin]
\setuphead[subsubsection][style=\sc, alternative=margin]

\setupheadertexts
	[{\sc \getmarking[chapter]}]	[{\pagenumber}]
	[{\sc \getmarking[Doctitle]}] [{\pagenumber}]	

% this is for truly empty pagebreaks
\definepagebreak
  [mychapterpagebreak]
  [yes,header,right]
%\setuphead
%  [chapter]
%  [page=mychapterpagebreak]

% setup title page
\unprotect
\definemarking[Author]
\definemarking[Doctitle]

\def\doctitle#1{\gdef\@title{#1}\marking[Doctitle]{#1}}
\def\docsubtitle#1{\gdef\@subtitle{#1}}
\def\author#1{\gdef\@author{#1}\marking[Author]{#1}}
\def\date#1{\gdef\@date{#1}}
\date{\currentdate[day, month, year]}  % Default to today unless specified otherwise.

\def\maketitle{
  \startalignment[center]
    \blank[force,2*big]
      {\scd \@title}
		\blank[big]
			{\scb \@subtitle}
	\stopalignment
	\startalignment[flushleft]
    \blank[25*big]
		\starttabulate[|l|l|]
    			\NC Name: \NC \@author
			\NR \NC Student Number: \NC 6100473
			\NR \NC Email: \NC john.haltiwanger@gmail.com
			\NR \NC Website:	\NC http://drippingdigital.com/
			\NR
		  \NR	\NC Date: \NC \@date
			\NR \NC Supervisor: \NC Richard Rogers
			\NR \NC Second reader: \NC Geert Lovink
			\NR \NC Institution: \NC Universiteit van Amsterdam
			\NR \NC Department: \NC Media and Culture (New Media)
		\stoptabulate
		\blank[3*big]
		\starttabulate[|l|]
		\NC {\sc Keywords} \NR
		\stoptabulate
		medium theory, ontogenesis, transduction, generative design, process oriented perspective, typesetting, ideological computing
\stopalignment}
\protect


% define descr (for definition lists)
\definedescription
  [descr]
    [headstyle=bold,style=normal,location=top, hang=20,
  width=broad,
	command=\hskip-1cm,margin=1cm]

% prevent orphaned list intros
\setupitemize[autointro]

% define defaults for bulleted lists 
\setupitemize[1][symbol=1][indentnext=no]
\setupitemize[2][symbol=2][indentnext=no]
\setupitemize[3][symbol=3][indentnext=no]
\setupitemize[4][symbol=4][indentnext=no]

\setupthinrules[width=15em]  % width of horizontal rules


% define a special head type of bibliography
\definehead		[bibliography] [chapter]
\setuphead		[bibliography] [number=no]
%\definecombinedlist		[content][chapter,section,bibliography]
\setuplist		[bibliography] [headnumber=no]

\definehead		[intro]	[chapter]
\setuphead		[intro]	[number=no]
\definecombinedlist		[content][intro,chapter,section,subsection,subsubsection,bibliography]
\setupcombinedlist		[content][alternative=c,interaction=all]
\setuplist		[intro]	[headnumber=no]

% let's get pretty chapters
\def\MyChapterCommand#1#2% #1 is number, #2 is text
  {\framed[frame=off,bottomframe=on,topframe=on]
     {\vbox{\blank\headtext{chapter} #1\blank#2\blank}}} % \vbox is needed for \blank to work
\def\MyEmptyChapterCommand#1#2% is a comment necessary?---apparently so...
	{\framed[frame=off,bottomframe=on,topframe=on]
			{\vbox{\blank#2\blank}}}

\setuphead[chapter][command=\MyChapterCommand, style={\scc},page=mychapterpagebreak,header=empty]

\setuphead[bibliography][command=\MyEmptyChapterCommand, style={\scc},page=mychapterpagebreak,header=empty]

\setuphead[intro][command=\MyEmptyChapterCommand, style={\scc},page=mychapterpagebreak,header=empty]

\setupheadtext[chapter=Chapter] % used by \headtext


% for block quotations
\unprotect

\startvariables all
blockquote: blockquote
\stopvariables

\definedelimitedtext
[\v!blockquote][\v!quotation]

\setupdelimitedtext
[\v!blockquote]
[\c!left=,
\c!right=,
before={\blank[medium]},
after={\blank[medium]},
]

% for long quotes
\definestartstop
  [longquote]
  [before={\indenting[never]
    \setupnarrower[left=0.5in,right=0.5in]
    \startnarrower[left,right]},
  after={\stopnarrower
    \indenting[yes]}]

% for bibliographic entries

% following hanging indent code (also in workscited) taken from 
%  http://www.ntg.nl/pipermail/ntg-context/2004/005280.html
% [NTG-context] Re: Again: "hanging" for a lot of paragraphs?
%  ~ Patrick Gundlach
\def\hangover{\hangafter=1\hangindent=0.5in}
\definestartstop[workscited][
  before={
    \page[no]
    \indenting[never]
    \startalignment[left]
    \bibliography{Bibliography}
    \stopalignment
    \setupwhitespace[medium]
    \bgroup\appendtoks\hangover\to\everypar
    },
  after={
    \egroup
    \indenting[yes]}]

\protect

\definestartstop
  [abstract]
  [before={\blank[4*big]
					 \midaligned{\sc Abstract}
           \startnarrower[2*middle]},
   after={\stopnarrower
          \blank[big]}]

\setupheader[state=start]


\starttext
	\doctitle{Grammars of the Damned}
	\author{John C. Haltiwanger}

\docsubtitle{Mapping Ontogenesis in Generative Design}


%\setuppagenumbering[location=]
\setupheader[state=stop]
\maketitle
\page

%\setuppagenumbering[alternative=doublisided, location={header,margin}] 


\startabstract
Here goes the abstract!

Will it do multi-line stuff for me?

Eh????
\stopabstract
\page
\setupheader[state=start]

\placecontent
%\startfrontmatter
%\intro{Acknowledgments}

%This is where I acknowledge everyone.

%\intro{Introduction}

%This is a chipper little introduction, don't you think?
% should work in general, i hope!

%\stopfrontmatter
\startfrontmatter
\intro{Acknowledgments}

I'd like to thank Natalia, for agreeing to review this Special
Edition of my ridiculous thesis.

\intro{Introduction}

Today's new media theory increasingly invokes {\em materiality} as
a significant, perhaps even {\em the} significant, mode of
investigating digital objects and the media through which they are
delivered. This thesis questions such a centrality of materiality
through a practice-based, process-oriented approach. {\em Process}
is proposed as the atomic unit of that which new media theory
investigates. This is true on a formal material level: applications
run as either as individual process or as assemblages of process
which are managed by an operating system and through which the
application's code is accomplishes all of its tasks, from memory
and access to algorithmic execution on the central processing unit.
A process-oriented approach will be shown to provide superior
methodologies for engaging with and understanding software than
material analysis alone provides. For instance, certain
problematics within Lev Manovich's concept of
\quote{media hybridity} will be resolved by a re-orientation
towards process (Manovich 2008). Process also allows a fresh
perspective for examining human-digital relations. Human processes
and digital processes are seen as inextricably intertwined, leaving
any discussion of digital process that excludes relevant dimensions
of human process necessarily unfinished.

The method proposed to demonstrate these points is two-fold. The
first is an analytic approach---the modes of operation of designers
themselves are examined. Starting from the proprietary Mac OS X
operating system, described here as a unique and powerful example
of {\em process hybridity}, we progress to a discussion of the
operations of designers as constrained by FLoSS
(Free/Libre/open/Source Software). The second aspect of the method
is a detailed interrogation of actual practice in the form of
{\em digital typesetting}. This topic was chosen for several
reasons. The first is a general lack of focus on the processes
behind typesetting among new media theory---while the surfaces of
text and textual interfaces have been investigated in numerous ways
(Bolter 2001; Fuller 2000), there has been a general lack of
theoretical concern (or capacity) regarding the underlying
processes of text in the metamedium (computers). This is especially
evidenced as regards the {\em command line interface}, a realm
where text becomes kinetic. Yet I found that very little theory has
been written regarding the command-line, despite its place as the
historical interface (once contemporary with batch punch cards) by
which digital processes were initiated. Far from being obsolete,
both Microsoft and Apple ship command line interfaces within their
operating systems. In Microsoft's case, significant money has been
spent developing a new grammar and implementing new functionalities
into their modern command line implementation Powershell (as
opposed to the grammar and functionalities of DOS).

The second reason for choosing typesetting is the supposed lack of
media hybridity of typesetting---according to Manovich's
definitions of the terms, typesetting has failed to move beyond
\quote{multimedia} to a state of \quote{media hybridity} (this is
opposed to typography, which according to Manovich has achieved
media hybridity) (2008: 86). Media hybridity is Manovich´s
formulation of the increasingly common ¨sharing of languages¨
between media. When media share language, they develop new
dimensions (2008: 86). Language, then, demonstrates its capacity
for modulation in a new context. While the proposition that
¨language can add dimensions to things¨ may at first consideration
seem a bit too obvious for stating out loud, the kinetic properties
of language within the context of the metamedium---that the code
enabling the language sharing that enables media hybridity is
{\em itself made of language and made executable by language}---seem
to beg for consideration. Whereas much of the new media discourse
relating to changes in media trends toward contemplating fast-paced
visual cultures such as video games and cinema, this thesis aims to
take the opportunity to contemplate the much slower-moving medium
of text. This contemplation of screenic text leads to questions
about the nature of media within a medium as well as to the
introduction of a conception of processual hybridity that both
underpins and exceeds the dynamics of media hybridity.

The third aspect is the allowance of a truly reflexive
investigation in which multiple processes of digital typesetting
are utilized to generate the thesis itself. This provides a means
to integrate the process-oriented perspective into a software study
of FLoSS typesetting software. Not only this, it provides a means
to attempt what could be considered a {\em refractional}
methodology. Inspired by Gilbert Simondon´s adoption of the
language of chemistry in the formulation of {\em transduction}
within his theory of ontogenesis, this thesis can be viewed as a
distinct crystallization process, the composition of a whole from
the process of that whole´s unfolding. The applicability of
Simondon´s ontogenesis to matters of generative design will be
interrogated in contrast to Jay David Bolter and David Grusin´s
remediation theory (Bolter and Grusin 1996; Bolter 2001).
Ontogenesis, albeit without Simondon, has already proven an
effective angle for approaching Web 2.0 platforms (Langlois,
McKelvey, Elmer, and Werbin 2009). Here the description of this
thesis´own workflow will demonstrate Simondon´s ontogenesis as
making unique contributions to the process-oriented perspective
which this thesis attempts to invoke and instantiate.

The fourth is the simple fact that screenic text has not been
interrogated on a {\em subtextual} level---surface analysis of text
(and hypertext) have driven the discourse of screenic text in new
media.

\section{Screens}

As digital typesetting provides the focus for the application of
the process-oriented perspective, the point of origin is
necessarily that of the screen. Information transmission is
increasingly screen-based, a fact that only intensifies with the
exponentializing ubiquity of mobile devices such as the iPhone. The
long-awaited advent of cheap \quotation{tablet} computers and
e-readers is also now at hand. These devices may all be seen as
mediums for {\em screenic processes} in that their entire
configuration and all of its computation exists to serve as the
basis for screenic interactions with {\em human processes}. These
phrasings introduce the perceptual angle attendent with this
thesis, namely the centering of {\em process} as the atomic unit of
what is discussed in new media theory. The term {\em screenic}
simply means \quote{screen-based,} or (perhaps)
\quote{screen-native.} It is analagous to \quote{printed.}

One way to define screens is in terms of their interactivity. Some
screens, such as television screens, offer very limited
interactivity: the choice of content. This choice itself can be
constrained by varying degrees, such as the number of available
channels and playback formats (VHS, DVD, Xvid, etc.), even to the
point of disappearing (in the case of many televisions that appear
in public spaces).
\footnote{Mobile devices are beginning to ship IR transeivers with full
hardware access through software. That is, the
{\em entire potential} of the IR spectrum is available to them.}
The medium of the remote control should not be underestimated in
its effects on human processes, to say nothing of the screens at
which they are aimed. Indeed, they drive the interactivity of the
video game consoles, an interactivity that clearly represents the
cultural cutting edge of what a television screen can offer.

The computer screen, on the other hand, is defined by its seemingly
limitless degree of interactivity. Remote controls can be run as
screenic processes and can not only change television
channels---processes on remote systems can be controlled with
similar ease. Indeed, the entire screenic composition of one
computer can be controlled over a network by a second computer
using included, or easy to obtain, applications. Furthermore, the
very interfaces to the screen (keyboards and mice) are examples of
remote controls in cases where the screen has not itself become its
own remote control (touch-screen devices). Typically the only
element of a computer screen that the user does not effectively
control are the structure and visual language of an operating
system's graphical user interface (GUI). Even this, however, is
generally accomplishable by a significantly informed user. In the
case of GNU/Linux the task is not only accomplishable: in the case
of a \quotation{from scratch} installation,
\footnote{Such as is demanded by no-frills distributions such as Gentoo and
ArchLinux, where manual installation and configuration of a GUI is
required for use.}
the user is literally forced to make a choice of GUI structure and
visual styling. Microsoft has generally shipped their operating
systems with multiple choices for widget
\footnote{A widget is the technical term for a GUI element. Scrollbars,
titlebars, menus, and close/minimize/maximize buttons are widgets
attached to most of the \quotation{windows} that appears on any
given GUI-driven computer.}
presentation, including re-mediations of widgets from previous
versions of Windows. Users also developed Apple, however, maintains
strict control of widget presentation, especially on their mobile
devices.

\subsection{Screens as material, screens as process}

Screens offer an ideal point of juxtaposition between the material
and processual frames. From a material view, the very formulation
of \quotation{screens} as {\em the} interface between humans and
computers is problematic: what of the interfaces that have been
developed to work around instances of blindness or other
[disabilities] that prohibit visually screenic interaction?

From a processual orientation, the question becomes: how do
interactions between humans and computers resolve themselves? The
answer returns in the form of the {\em available} remote controls
and the {\em available} response interfaces. The next step might be
to investigate the degree of variance between these availabilities,
and whether they problematize any umbrella-classification. While it
would be {\em insensible} to argue that material differences in
inputs and outputs can---or do---not lead to a huge amount of
variation between experiences within humans. Such variation is
likely to occur in differentials. That is to say, the spectrum of
possible feedback occurs at the level of the human
individual---one's experiences are functionally irrepresentable
without translation of some kind. [We can choose to call these
translations mediums, or we can choose to call these processes.]

At this point the question becomes, then, whether it is necessary
to instantiate these inherent divergences in every evocation of a
broad level discussion of input and output mechanisms or whether
the inherent, {\em core} similarity between them all remains that
in all instances they serve as {\em the point of contact} between
human and digital processes. Does it make a processual difference
if the output technology is a braille screen or an LCD screen? Only
inasmuch as to what degree the process being examined is unique to,
or highlights differences between, one or the other. From a
discursive level, {\em controls} and {\em screens} can capture the
essence of these dual \quotation{action spaces} that together form
the single point of contact between human and digital process.

Is it possible to remediate of the term screen into discussions of
previous mediums? For instance could one speak of the
\quotation{screen} of a newspaper or the \quotation{screen} of a
cave wall? What about the \quotation{screen} of a radio? From a
linguistic-conceptual perspective the final example certainly
pushes the limits. From a process perspective, though, the presence
of the radio/what it is playing/what listening choices are
available/how and to what degree does the hardware support
frequency tuning: these questions can all be conceived in terms of
\quote{control} and \quote{screen.} The sounds of a radio do emit,
after all, from the vibrations of a stretched membrane.

This thesis proposes a conceptual-linguistic shift in the
discussions of screens as the {\em site of discourse} through which
digital processes yield the results of their execution. Likewise,
the remote control, or simply {\em control}, is the site of
discourse through which which human processes instigate and extend
into the digital. There is no removing or reducing of this dyadic
assemblage---even when the control and the screen are literally
fused (as in most contemporary smartphones) the distinction between
{\em control} and {\em screen} holds on both a conceptual and
material level. Conceptually, human process still extends through
the control into digital process, which still produces feedback
through the screen. Materially, the screen is a Liquid Crystal
Display driven by a graphics card that interfaces with coded
drivers and display subsystems in the device's operating system.
The control, on the other hand, is the glass suspended over the LCD
which, through one or more of the multitude of available technical
solutions for the process, reads point(s) of contact, pressure, and
vectors (velocity and direction) of movement.

\section{From Screens to Text}

To discuss computer screens one must necessarily engage with the
concept of {\em interface}, a topic that rightfully occupies a
great deal of current new media discourse. Interface, then,
represents one point of departure from our origin. While interfaces
often utilize many visual metaphors (most of them inherited from
the work done developing the first GUI at Xerox's Palo Alto
Advanced Research Lab (PARC) in the 1970s), there are yet few
computer interfaces that do not rely on text as their dominant
mechanism for organizing and presenting a program's internal
capabilities to a user. (Mobile screens, on the other hand,
increasingly display developing trends of icon-only design, though
the web browser remains a popular application). Despite the success
of the GUI over the text-only command-line interface (CLI), text
remains central to contemporary experiences of computer screens.

The command line is seen as a space of contestation for traditional
modes of media analysis. Remediation, for instance, will be
demonstrated as inappropriate for discussing the CLI. As Google has
just recently released a command line interface for interacting
with Google services, I believe a discussion of the command line is
essential for new media (Holt and Miller 2010).

(Unfortunate to note, this historiographic aspect is still
{\bf \quote{to-do}}:

The centrality of text to the experience of computer screens
represents the main avenue by which we proceed from the origin,
constituting a trunk from which many additional concerns fork away
and then face examination. The arguments of the paper are augmented
by the inclusion of a historiography of digital typesetting.
Engaging critically with the history of {\em software itself} is
considered a requisite for responsible software studies: a full
range of influences (economic, cultural, technological) should be
considered in the re-telling of a given processual unfolding. In
this aspect of focus, it extends Lev Manovich's admirable
positioning of history as central to a software study by broadening
the scope of historical considerations.
\footnote{{\bf Note:} This work largely remains unfinished in this draft, as
it became apparent that I needed to work back through more
discussions of basic infrastructural elements such as operating
systems in order to fully describe the assemblage of process upon
which computer-based design is situated.)}
Inspiring this enagement is the work of Robin Kinross, whose
{\em Modern typography: an essay in critical history} is one of but
a few texts covering a history of typography to adequately engage
with the influence of factors outside of that field on the field
itself (Kinross 2004). By integrating a critical history of digital
typesetting with a process perspective, an equilibrium between
human and digital processes will be illustrated.

\subsection{Recognizing the Ontogenesis in Generativity}

In his text {\em The Position of the Problem of Ontogenesis},
Simondon writes,

By transduction we mean an operation---physical, biological,
mental, social---by which an activity propagates itself from one
element to the next, within a given domain, and founds this
propagation on a structuration of the domain that is realized from
place to place: each area of the constituted structure serves as
the principle and the model for the next area, as a primer for its
constitution, to the extent that the modification expands
progresively at the same time as the structuring operation.
(Simondon 2009: 11).

Note the distinct lack of \quote{computational} in Simondon's list
of operations. Written prior to the advent of Manovich's
formulation of the age of cultural computing, this absence might
simply be read as a matter of temporal context. Nevertheless,
Simondon's solution to the ontogenesis problematic provides a
framework for describing digital processes of a generative nature.

This leads to another important element of this thesis, one that
runs throughout the entirety of itself---the underlying processes
of presentation required to \quote{typeset} the text itself.
Through the utilization of FLoSS software, multiple output formats
will be not only be investigated but also materially instantiated
through a designed mechanism of process---a
{\em processual hybridity}. These output formats represent two of
the top formats currently used to manage and display texts
digitally: HTML and PDF.

The process(es) of their generation offers an attempt at mapping
Gilbert Simondon's language of ontogenesis onto file format
translation or, to begin the project immediately,
{\em individuation}. Coupled with Simondon's individuation is this
concept of {\em transduction}. Repurposed from the language of
chemistry, Simondon's metaphorically images transduction with the
example of a substrate---swelling with {\em metapotential}---that
crystallizes. The final formation is the substrate fulfilling this
metapotential, a fulfillment that arises only through an
unpredictable unfolding involving emergent factors. (The language
of chemistry was likewise appropriated for the term
\quote{interface} (Cramer and Fuller 2008: 149)).

Through this mapping I hope to provide a convincing argument for
shared properties between what I am calling process and
individuation, and between transduction and what I am calling
instantiation.

This relates with the increasingly generative nature of
contemporary design. All of which are generated from a plain-text
file whose syntax conforms to a format standard called
\quote{markdown.} The polycephalous nature of {\em the text itself}
thus demands further branching into a discussion of formats. What
are the attributes of the class of process to which formats belong?
Formats are seen as stable, yet they move like glass (or glaciar)
in the nano-magnitudes of the digital. Formats provide another
point of contrast between process and material perceptual
orientations.

The discussion of generativity provides further means to
demonstrate the equilibrium of human and digital processes.
Analyzed materially, these processes are chunks of code
electronically lifted from hard drive platters, loaded into system
memory, and then executed via the assemblage of chips on the
computer's motherboard by way of instructions from the operating
system currently residing as a mass of memory heaps in RAM chips.
Analyzed {\em processually}, however, these digital processes are
properly seen as deriving from interactions with human beings. That
is to say, digital and human processes are intimately intertwined,
from the design of their physical landscape of execution
(microcircuitry) to the instructions derived from the user. From a
process angle the computer becomes something of an external nervous
system, extending and modifying the realm of human potentiality
even as it surpasses the capacity of a single mind to functionally
comprehend the entirety of its workings.
\footnote{The chips produced by Intel, for example, are too complex for any
single person to ever hope to entirely understand.}

\subsection{Print is static, code is process}

The flat/deep distinction proposed by Hayles is, by its
formualation, material. Problematizing this material focus is the
interwoven history of text and code: the lens of typesetting allows
us to focus on a unique intersection of the two. As the
historiographic case will demonstrate, typesetting is a
{\em non-reducible} process (NP-Complete). This non-reducibility of
typesetting reflects the non-reducibility of computational
processing of language, as well as the non-reducibility of
language, as signifier, into that which is actually signified. This
\quotation{turtles all the way down} scenario has intriguing
implications from a process perspective as we investigate the
methods that have been developed in order to work around this
non-reducibility.

When Hayles states that
\quotation{materiality thus cannot be specified in advance; rather it occupies a borderland---or better, performs as connective tissue,}
she is provisionally correct (Hayles 2004: 72). However, this
metaphor-ization of process is exemplary of new media practices:
reference the complex with an abstract metaphor, obscuring complex
and important dynamics with a metaphor. The metaphor works, to be
sure. One could even consider it an ideal formulation. At issue is
the fact that this borderland is not discussed in a technically
correct manner.

\section{Remote Controls}

I think it may be reasonable to take the remote control and use it
to create a metaphor for all human-computer interaction.

Every digital process has, at its origin, a human. The rate of
computation has increased the impact of human-digital processes in
that the results deliver their results faster. The results will
either match the intentions of the originating human process, or
they will not. In the second case we can find the first evidence of
the effects of digital process on human process:
{\em the code behind the digital process will be re-arranged in an attempt to deliver an output that satisfies the intention of the human processes.}
Whether this modulation of the executed code is through
sliders/input boxes/etc within a GUI interface or through direct
reworking of the source code itself, the effect is the same: the
code executed has been re-configured according to the goal of human
process. The result(s) of the digital process, experienced through
a screen, can match, exceed, or fail this goal. In turn, human
process is effected and the next move is made according to new
goals or revised digital processes.

Video games, for example, can easily be represented by this model.
Human process is obviously shaped by digital at the outset: there
are a finite number of actions that a game offers within its
context. In addition, these actions are often presented as pre-set
mappings of action to controller button.

\section{Attributes of Process}

[It needs to be asserted that I am willingly engaging in my own
appropriation of the term \quote{process} outside of any traditions
other than my own. As the process oriented perspective arose under
the looming shadow of the draft deadline, I admit to a lack of
historical perspective on the use of this word in either new media
or other contexts. Withstanding that, however, I sense a real
applicability of this term in the discourse of new media. I'm
looking forward to working on the final draft and using some of
that time to construct historical perspective for this shift to
process. One important angle is Ned Rossiter's work on
\quotation{Processual Media Theory} in his book
{\em Organized Networks} (Rossiter 2007: 166--192), which this
draft does make use of but which I would like to interweave more
deeply. As it stands, this introduction was left relatively alone
for the sake of fleshing out the middle part of this thesis. This
was at the suggestion of the second reader.]

Process is reflective. It's outputs reflect its inputs.
Additionally, process reconfigures the metapotential in any given
system. It's reflectivity, then, has material effect. As it
reflects the inputs into the outputs, the outputs in turn reflect
new (or else simply different) potentials back into the
{\em context} which is the reciprocal contact point in which the
processes began. This language is extrapolative into any set of
intersections. This paper considers just the subset of
human-digital recipricity, and within the relatively static domain
of typesetting.

A new configuration of metapotential in any system results in the
reconfiguration of (all) other systems as well. This fact reflects
the {\em fractal} nature of process---there is a degree of
non-reducibility inherent in any discussion of process, as
ultimately certain factors in its functioning are unknown to us.

\section{Why free software?}

There are multiple points of consideration that lead me to
concentrate on free software. The first is its relative lack of
presence within new media circles. Time and again I arrive at a
conference only to see a room full of computers booted into
proprietary operating systems. While I am not a \quote{zealot} who
disavows any potential use or need for proprietary software, I find
the general population of new media's reliance on proprietary
operating systems---chiefly, by way of personal and anecdotal
evidence, Mac OS X---disturbing. Hans Magnus Enzensberger outlined
in his \quotation{Constituents of a Theory of the Media} the
importance of issues of control with relation to mediums. Let us
move through the juxtaposed elements of repressive versus
emancipatory uses of media which Enzensberger provides and
interrogate them in relation to Mac OS X and GNU/Linux
(Enzensberger 1970: 269):

{\em Repressive versus Emancipatory}

\startdescr{Centrally controlled program vs.~Decentralized program}
This question is answered by asking the question:
\quotation{Where is the source code of the operating system?} In
the case of OS X, the source code resides only within the confines
of Apple's corporate computers. It is likely heavily guarded by
multiple mechanisms. Whereas in the case of GNU/Linux, the
operating system source code is spread across dozens of mirrors on
the Internet as well as the computers of programmers and users
around the world. Each of these copies can be readily modified to
the designs of any given user, demonstrating decentralized (in
fact, distributed) control. Apple maintains sole, central control
of the code and thus fully determines the functional possibilities
of the operating system.
\stopdescr

\startdescr{One transmitter, many receivers vs.~Each receiver a potential transmitter}
This is already demonstrated above: the code for GNU/Linux is
globally distributed across hundreds of thousands of computers.
Each one of these has the ability to modify the software and share
those modifications with anyone who will listen. OS X can be
modified by no one.
\stopdescr

\startdescr{Immobilization of isolated individuals vs.~Mobilization of the masses}
OS X encourages the use of proprietary applications. These
applications have restrictive license that generally allow only one
individual the right to run the application. GNU/Linux, meanwhile,
\stopdescr

\startdescr{Passive consumer behavior vs.~Interaction of those involved, feedback}
A major advantage for both users and developers in a free software
ecosystem is the feedback that occurs between them. Users may
suggest new features at any time. If they have the skill and/or
time, they can add these features themselves. If the addition of
the features is contentious in any way, the contributer can simply
fork the codebase and continue evolving the software in new
directions. In OS X, you run the binaries you are given.
\stopdescr

\startdescr{Depoliticization vs.~A political learning process}
Mac OS X is pro-capitalist and promotes consumer culture. It can
probably be said that it is politically \quotation{neutral} in its
codedness, but this very codedness remains obfuscated and
proprietary. GNU/Linux, in conservative judgment, at least does not
actively promote consumerism. In an idealistic formulation, it
destabilizes the capitalist ecosystem.
\footnote{It is important to note that free software also plays a significant
role in supporting this infrastructure, as the license provides no
recourse on the terms of the softwares use (Pasquinelli 2008).}
It's politics are as multifaceted as its user base. In its
well-deserved reputation as
\quote{taking some work to make it work,} GNU/Linux forces its
users to become active in the system's administration. This induced
learning of an open approach to computer systems could be said to
have political dimension.
\stopdescr

\startdescr{Production by specialists vs.~Collective production}
This seems self-explanatory.
\stopdescr

\startdescr{Control by property owners or bureaucracy vs Social control by self-organization}
Are you getting the picture?
\stopdescr

In a presentation at the Libre Graphics Meeting 2010, Florian
Cramer explains his theoretical positioning of free software as an
entry point into media criticism. Aymeric Mansoux, also of the
Networked Media design faculty at the Piet Zwart Institute and
present with fellow faculty member Michael Murtough, describes the
critical engagement in the error message common to GNU/Linux
distributions, found in the Totem media player program complaining
of a missing codec library that is required to decode common
patent-encumbered media formats such as MPEG-Layer 3 (Cramer,
Mansoux, and Murtaugh 2010). Behind the error message lies an
assemblage of inter-related issues of intellectual property rights,
cultural practices, and media accessibility. This is a clear
instantiation of a \quotation{political learning process.}

On 21 June 2010, Apple changed its privacy settings to allow the
company to
\quotation{collect store and share \quote{precise location data, including real-time the geographic location of your Apple computer or device}}
(Marco 2010).

\subsection{Caveats}

Free software is not, however, a \quotation{magic bullet}---tied to
the open systems theory which is philosophically related to the
underpinnings of the Chicago school of economics, some of the
philosophical foundations of free software,
{\em and especially open source}, need to be interrogated (Cramer,
Mansoux, and Murtaugh 2010; Pasquinelli 2008). Liberation does not
automatically lead to a distribution of tools to all those that
need them. However, even in this instance we see the power of FloSS
in its capacity to inspire critical engagement with media.

\stopfrontmatter \chapter{Substrates of Digital Process}

\section{Alan Kay and a \quote{Metamedium} Vision for Personal Computing}

In his text {\em Software Takes Command}, it is Manovich's
inclination to focus on the work of Alan Kay at Xerox PARC when
discussing the development of {\em cultural software}. He notes
that there are multiple entry potential entrypoints for
consideration: the work of Douglas Englebart and his team, the
development of the LINC computer at the MIT's Lincoln Lab, and
Donald Sutherland's SketchPad. The development of the Xerox Alto,
however, is unique in multiple ways. First and foremost is the
architecture of the software: by developing and employing an
object-oriented approach to software design, users were positioned
as inventors of new media through their ability to design their own
interfaces that both enabled and spurred new modes of creation
native to the screen. These screenic modes of creation represented
a new, vital dimension to computing---the willful,
{\em shaping into existence} through design and implementation of
new digital processes. Kay's team was specifically dedicated to
applying intersections between education and computation. In the
process of teaching the system to children and adults alike, those
they taught often ended up developing their own unique applications
out of the {\em objects} that could be shared between applications
as well as extended through the inheritance model of
object-oriented programming.

Cultural software---and the {\em cultural computing} which it
facilitates---is defined by Manovich as software that is
\quotation{directly used by hundreds of millions of people} and
that
\quotation{carries \quote{atoms} of culture (media and information, as well as human interactions around these media and information)}
(2008: 3). Alan Kay is a pioneer figure in the computing world, an
individual who not only theoritically formulated a vision of the
computer as a \quote{metamedium} but also did a great deal of
practical work in order to achieve this vision. Unfortunately, as
is all too common in the lives of visionaries, key elements of
Kay's ideal never breached into the mainstream even as other
aspects where appropriated and commodified wholesale by Apple and
Microsoft.

The first, and probably most important, element of Kay's original
platform that failed to transfer from his ideal \quotation{Alto}
personal computing platform to the commercial GUI-drive operating
systems that now define the landscape of cultural
software[\letterhat{}foss\letterunderscore{}exception] is the
concept of {\em easy programmability}. To this end Kay founded the
basis of all the Xerox PARC work in personal computing on a
programming language called Smalltalk. In his text
\quotation{The Early History of Smalltalk,} he explains that the
computer science work of the 1960s began to look like
\quotation{almost a new thing} as the funding of the United States'
Advanced Research Projects Agency (ARPA) led to the development of
techniques of \quotation{human-computer symbiosis} that include
screens and pointing devices (Kay 1993). Kay's focus became
investigating what, from this standpoint of
\quotation{almost a new thing,} what the
\quotation{actual \quote{new things} might be.} Any shift from
room-size, mainframe computers to something suitable for personal
use presented new requirements of
\quotation{large scope, reduction in complexity, and end-user literacy}
that
\quotation{would require that data and control structures be done away with in favor of a more biological scheme of protected universal cells interacting only through messages that could mimic any desired behavior}
(Kay 1993)[\high{smalltalk]. [}smalltalk]:
http://gagne.homedns.org/\lettertilde{}tgagne/contrib/EarlyHistoryST.html

To this end, Kay and other members of the PARC team developed this
language Smalltalk to facilitate a
\quotation{qualitative shift in belief structures---a new Kuhnian paradigm in the same spirit as the invention of the printing press}
(Kay 1993). The technical details of Smalltalk are indeed
impressive to those who can read and understand them: Smalltalk is
{\em object-oriented}, {\em dynamically typed}, and
{\em reflective} ({\em Smalltalk} 2010). All three of these
approaches have found increasing adoption in computer programming
languges. For instance, Yukhiro Matzumoto's popular language Ruby
offers all three of these characteristics and the founder readily
admits that his language is greatly influenced by Smalltalk
(!CITE!). For the sake of reference, the idea of dynamic typing
remains controversial among programming language designers from the
implementation of Smalltalk in the 1970s until this very day.

One of dynamic typing's greatest attribute---and perhaps at least a
part of why some choose to object to it---is that it makes the
programmer's job easy. Whereas {\em statically typed} languages
such as Java or C require the declaration of the kind
(\quote{type}) of data the programmer expects to store in a given
variable, a dynamically typed language has no such limiting
expectations. Instead anything can be stored within any variable,
something which allows for a great deal of flexibility in the hands
of a programmer.

Object-oriented programming, though, represents an even larger
shift in the potentials presented to programmers.
Object-orientation metaphorically maps the concept of
\quote{objects} onto the practice of programming, providing a means
to encapsulate sections of code in a way that encourages
reusability and thus increases programmer productivity. It is also
said to be easier for the human mind to learn to think in the
metaphor of objects over, for instance, the functional programming
paradigm that is heavily influenced by the lambda calculus and
which represents relatively steep initial barriers to those for
whom discrete mathematics is not a second language. Since objects
can contain other objects as well as maintain inherited
relationships with yet other objects (using the abstract form
called a \quote{class,} which is a pre-instantiated description of
what an object will contain once it is called into existence), the
potential for constructing programs out of the pieces of other
programs was greatly increased.[\letterhat{}I will return to this
topic when we reach discussions of \quote{process hybridity.}]

The last attribute of Smalltalk on this list, its
{\em reflectivity}, simply means that every object can explain its
own properties, i.e.~which \quotation{methods} (blocks of code
oriented towards performing a specific task) it provides, what
variables it is keeping track of, etc. The result is that
\quotation{objects can easily be inspected, copied, (de)serialized and so on with generic code that applies to any object in the system}
({\em Smalltalk} 2010). This capacity for writing
\quotation{generic} code once again has a positive impact on
programmer productivity, as demonstrated by Kay and Goldberg's
description of a hospital simulation designed to be able to compute
\quotation{any hospital configuration with any number of patients}
in their ground-breaking essay presentation of their work,
\quotation{Personal Dynamic Media} (Kay and Goldberg
1977[\letterhat{}kg]: 400). [\letterhat{}kg]: The page numbers
refer to the article's position within {\em The New Media Reader},
published in 2003. The original date of publication has been
maintained in order to respect the publication's true chronological
position in computer history.

This digression into the specific mechanisms by which Smalltalk
eases the tasks of programming highlights the focus of Kay and his
group (the Learning Research Group) at Xerox PARC. The intentions
of their research---as Kay says in a passage quoted earlier---was a
paradigm shift on the level of the printing press. Given the degree
to which the mechanics of contemporary computing received their
\quotation{start} within the Learning Research Group as well as the
pervasive impact of that contemporary (\quotation{cultural})
computing, it can reasonaly be said that they achieved this goal.
Further testament to the success of their system was the fact that
\quotation{children really can write programs that do serious things}
(Kay and Goldberg 1977: 394). Indeed, a
\quotation{young girl, who had never programmed before, decided that a pointing device {\em ought} to let her draw on the screen. She then built a sketching tool without ever seeing ours}
(399).

There is, however, one major element of their
implementation---central to the capacities for children
programming---that did not find manage to find its way through the
commercialization of their ideas via Apple and Microsoft: that of
unlimited access to the building blocks of the system itself. While
Kay clearly saw not only the potentials but also the real,
invigorating effects of providing
\quotation{exceptional freedom of access so kids can spend a lot of time proving for details, searching for a personal key to understanding the processes they use daily}
(404), unfortunately for the un-folding of cultural computing
neither the Apple Macintosh nor Microsoft Windows shipped with the
capacity for unfettered tinkering with the platform itself. Even
worse, they generally shipped without the programming tools
required to simply make a program on one's own, irregardless of the
degree of availability of \quotation{building blocks.} Considering
the enthusiasm with which children interacted with Kay's system, it
is hard not to question where computer literacy---the ability to
read {\em and} write within the computer---in the age of the GUI
might be should either of those seminally-positioned products have
shipped with the kind of development environment to be found in
Kay's largely-forgotten prototype.

--- kay and the ipad
http://www.tomshardware.com/news/alan-kay-steve-jobs-ipad-iphone,10209.html

\section{Operating Systems as the Substrates of Contemporary Design}

This line of inquiry into lost historical possibilities serves only
to highlight the serious degree to which history affects the
unfolding of the metamedium that we call a computer. Historical
developments have also lead to further opportunities, as well. The
manifestation of the metamedium as a space operating nearly
exclusively on proprietary software in the 1990s drove many
freedom-minded individuals to begin, and contribute to, free and
libre software. What at first approach looked like an impossible
task (Richard Stallman's pledge to implement a completely free
implementation of Unix called Gnu's Not Unix (GNU) OS) has resulted
in a snow-ball effect that provides not only the backdrop for much
of this thesis' discussion but indeed the GNU project provides the
compiler with which Apple builds its invisible, proprietary
codebase into the commodity known as Mac OS X.
\footnote{In his text
\quotation{The Ideology of Free Culture and the Grammar of Sabotage,}
Italian media theorist Matteo Pasquinelli provides a critical
analysis combining Marxist discourse on {\em resource extraction}
with Michael Serres' conception of the {\em parasite} in an
analysis of contradictory intersections between capitalism and
FLoSS (Pasquinelli 2008). Apple´s investment in developing a
competitor to the GNU compiler, LLVM, is the result of Apple´s
decision to avoid the GNU Public License as much as possible (LLVM
is licensed under an easily-coopted open source license).}

\subsection{The evolving nature of operating systems}

The conceptions and roles of operating systems have been evolving
along with the hardware. Originally the operating system of a
computer was intimately tied to its hardware. It was the
development and {\em dissemination} of Unix by software engineering
luminaries Ken Thompson and Dennis Ritchie that sparked the
ontogenesis of operating systems as a ¨hardware independent¨
process, enabling {\em cross-platform code} as a dimension in the
software process. While operating systems had always existed to
abstract the system to some extent, for instance to hold segments
of code for handling routine tasks (incidentally, these segments of
code are often called `code routines´) such that they need not be
input every time that operation is to be performed, the
introduction of Unix accelerated this process of abstraction.

In 1999, more than twenty years after this milestone in
abstraction, speculative fiction writer and trained programmer Neal
Stephenson released a novella-length essay entitled ¨In the
Beginning, Was the Command Line¨ (Stephenson 1999). This essay
delivers an analysis of (consumer) operating systems and a critique
of the reign of the graphical user interface. By means of
distinguishing the cultures and manifestations of the operating
systems---Microsoft´s Windows, Apple´s original Mac OS, Be´s BeOS,
and GNU/Linux---Stephenson offers cars as analogy. Mac OS becomes
an expensive European car, streamlined but expensive to maintain.
Windows becomes a station wagon, the un-sexy, utility-oriented
choice of the masses. The (now-defunct) BeOS is a Batmobile. And
GNU/Linux is a tank without a price tag.

Computer programmer Garrett Birkel, writing in his update/response
to Stephenson---¨The Command Line in 2004¨---notes the extreme pace
of evolution of computer hardware. Despite the expanding set of
elements contained within a personal computer ¨we don't call our
new Dell machine a
\type{computing collective´. We consider it one device, with a singular name. And our concept of an}Operating
System´ has evolved right along, neatly obscuring this complexity¨
(Birkel 2004). This leads Birkel to clarify that what Stephenson´s
text concerns itself is not so much the distinctions and
separations between hardware and software as the mode of
interactions with which we engage information:

*****-T-he crucial separation here is not between the computer
(hardware) and the Operating System (software). Those are so deeply
integrated these days that they have effectively merged into the
same beast. The crucial division is between ourselves, and our
information. And how that division is elucidated by the computer,
with hardware and software working in tandem, really determines how
useful the computer is to us. In other words, it's all about User
Interface, and even though
\quotation{In the Beginning, There Was the Command Line}, it's also
true that In The Beginning, Information Took Fewer Forms. (Birkel
2004)

The operating system is the initial site of formation for the
processes that define this division between ¨ourselves¨ and ¨our
information,¨ but it is far from the end of story, and perhaps far
from the most influential. I will return to Stephenson and Birkel
in the next chapter, which begins with a discussion of the command
line. For now, the point has been made that there is no true means
of separating hardware from software---without an operating system,
there is no means for hardware to access its own functionality. The
materiality, then, of a personal computer is inherently tied to
process---without an OS, the PC is not a metamedium, it is dead
weight.

\subsection{Unix and Code Drift}

The capacity of Unix to migrate across hardware while maintaining a
consistent interface to both users and programmers has been
criticized as well as lauded, for example in the
{\em UNIX-HATER´s Handbook} wherein the authors liken it to ¨a
computer virus with a user interface¨ (Garfinkel, Weiss, and
Strassman 1994: 4). The author´s complain that the portability of
Unix results from an under-designed infrastructure, which they call
{\em incoherent}. The virus metaphor works, for them, because of
this portability (whatever its origins), the ability to ¨commandeer
the resources of hosts¨ (of which there were, even at the time of
their writing, many), and Unix´s capacity for rapid mutation
(Garfinkel, Weiss, and Strassman 1994: 4).

This capacity for rapid mutation is evidence of Unix´s
{\em evolutionary superiority} over other operating systems, which
the UNIX-HATERS note is not concomitant with
{\em technical superiority} and, they assert, in the case of Unix
the balance is far too the side of evolution over technical
soundness (6). They decry the introduction of ¨junk genetics¨ into
the Unix codebase, resulting in either vestigial code for devices
that no one uses anymore or divergent implementations (mutations of
mutations) (7). This idea of ¨code drift¨ has been recently invoked
by Arthur and Marilouise Kroker as the ¨laws of motion¨ of a
digital cosmology (Kroker and Kroker 2010). Unix, then, could be
considered an early object governed by this law of motion, from its
initial existence as a system shared freely in code across research
institutions to its fractured existence as inconsistently
implemented proprietary offerings to its resurgence as free
(GNU/Linux) and open source (Free/Open/Net-BSD) software.

The introduction of an ontological premise for software, Richard
Stallman´s {\em free software}, has multiplied the evolutionary
superiority of operating systems by orders of magnitude. Though his
GNU operating system project is, as a whole, unfinished after more
than two decades of development, the environment surrounding that
effort---the GNU compilers, the libraries of code upon which they
depend, and a vast array of available commands and
applications---provides a ¨genetic library¨ from which one can
extract whatever processes are useful for one´s own ends (provided
one follows the rules of the code´s licensing terms, the GNU Public
License (GPL)). As a result, GNU code appears in not only virtually
every non-proprietary platform, it also ships with, and provides
important underpinnings to, that flagship of the designing class:
Mac OS X.

\subsection{Mac OS X: Process Hybridity in Action}

As the Apple platform has long been the preferred environment for
professional designers---since before OS X's first release in
2001---it necessarily invokes itself as an important object of
study. It allows us to investigate the concept of
{\em process hybridity} that I introduced within the introduction.
Mac OS X is a multi-layered system that contains both open and
closed source elements with distinct historical lineages. As an
assemblage, it can be represented as containing a high degree of
process hybridity as it combines not only various separate projects
into unified commodity, it also bridges ideological boundaries by
combining proprietary, open, and free code into a single commercial
object.

This hybridity reaches into the very core of OS X, into the kernel
itself. The kernel of an operating system handles the specific task
of delivering the instructions deriving from software to the
specific pieces of hardware that pertain to those intructions. A
simplistic analogy can be made to a traffic cop, whose role it is
to direct traffic in a way that the competing interests at an
intersection proceed in an orderly fashion. Kernels come in two
flavors: the {\em monolithic kernel} and the {\em microkernel}. The
distinction between the two resides in the size of the tasks they
are expected to perform, as well as a material distinction between
where operating system code should be executed ({\em kernel} memory
or {\em user} memory) ({\em Kernel (computing)} 2010).

Without entering into a lengthy discussion of the differences
between the two---a discussion that has sparked a great degree of
vitriol between their respective advocates---let us note that a
monolithic kernel contains the code for everything from networking
to video display drivers to encryption mechanisms. A microkernel,
on the other hand, delegates these processes to {\em servers} that
run outside of the kernel's memory space in the user memory (or
{\em userland}, as it is often called). The result is that there is
a great deal less chance of crashing a system that runs on a
microkernel, as a bug in user space cannot affect the operation of
the hardware whereas a bug in kernel space necessarily can.
Microkernel's are also much more modular, as servers can be
replaced with (theoretic) ease due to the fact that it sits outside
the kernel's code. OS X's kernel, however, is a hybridization of
the monolithic and microkernel designs ({\em XNU} 2010). Due to
speed concerns, it was decided that elements of the FreeBSD
project's monolithic kernel would be welded onto the Mach
microkernel. {\externalfigure[images/os_x_architecture.png]}

Moving up from the kernel we encounter the system utilities. These
are directly imported from BSD Unix, specifically FreeBSD. This is
the last layer of what constitutes the \quotation{core OS}, which
is released under an open-source license. It must be noted,
however, that the vast majority of code within this part of the
operating system was licensed under terms that allow Apple to
withhold any and all code whatsoever (i.e. {\em open source}
licenses such as BSD or MIT). The major exception to this is the
GNU C Compiler (GCC) and its attendant libraries, which are
{\em free} (as in freedom) software and is licensed under terms
that force Apple to make available the code of their modifications
whenever they publicly release new binaries of their version of the
GCC. On the other hand, specific improvements to BSD-licensed code
is available only at Apple's whim. It is impossible to monitor
whether the source code they make available represents all the
improvements they have made or whether they have been selective
with their source in order to maintain their proprietary advantage.

Above the core OS level we find further evidence of OS X's storied
history. The roots of Mac OS X reside in the operating system
NeXTSTEP, developed at Steve Jobs' NeXT, Inc. after he left Apple
in 1985. Like all GUI-based operating systems, NeXTSTEP necessarily
incorporates a large number of metaphors developed first at Xerox
PARC. In addition, however, it was built from the ground up to be
easier to develop applications for than other operating systems at
the time. To this end the Objective-C language, object-oriented and
heavily influenced by Smalltalk, was adopted. While certain systems
(such as the microkernel) might be written in C, applications were
developed in Objective-C and all the servers above the kernel spoke
exclusively with applications in this language. The programming
environment was also a leap forward in ease of use: GUI windows
could be designed in a WYSIWYG fashion, where their component
widgets (buttons, menus, etc) could be easily tied to blocks of
code which would inform the computer what was to be done upon a
given widget interaction. NeXTSTEP thus represented somewhat of a
resurgance for Alan Kay's original vision of personal computing.
However, NeXT's first computers were priced at \$9,999
dollars---resulting in slow adoption outside of academic or
institutional contexts. It was within such a context that Tim
Berners-Lee developed the first web browser, WorldWideWeb, on a
NeXT computer.

Three other hybridities occur within this top level of Mac OS X:
Carbon, Classic, and Java (the programming environment of NeXTSTEP
was renamed Cocoa in OS X). Carbon was developed as a new way of
writing Macintosh applications that would allow developers an
easier transition path from Mac OS Classic to OS X. Applications
written for Carbon could run on both the older Mac OS as well as OS
X, allowing companies like Adobe to refashion their code without
changing programming languages. Classic was an emulation layer that
allowed \quotation{non-Carbonized} Mac applications to run on OS X
(this hybridity has been removed since Apple's transition to Intel
microprocessors). Java is a programming language and environment
designed to allow universal execution of programs regardless of
underlying architectures: if Java has been ported to an OS, then
theoretically any Java program can then be run on that OS
\footnote{However, there are many different versions of Java, making this
ideal of \quotation{write once, run everywhere} somewhat
problematic. Microsoft was even successfully sued for trying to
hijack the language with by leveraging an incompatible
implementation through the virtual industry standardization of its
Visual Studio development environment.}.

The hybridities present in Mac OS X have material effects from the
execution of code to the variability of its cultural enablement. It
allows for a vast assortment of applications from various lineages
and paradigms to coexist and interoperate within a single operating
system context. Short of labelling each of these hybridized
elements a \quotation{medium,} there is currently no proposed
language within new media for articulating the hybridity of the
assemblage as a whole. In this fashion, {\em process hybridity}
provides a mechanism with which we can describe the facets of OS X.
Perhaps UNIX is a medium, perhaps not. However, as a standard, it
can be recognized as a formalized resolution of a design process.
In an environment where 3D cinema is not even presented as a new
medium in popular discourse, the term itself appears to me to have
lost its utility. Further in the thesis I will introduced more
localized and dissectable instances of process hybridity. At this
point it should be convincingly demonstrated that Mac OS X
hybridizes not only material processes (code) but also ideological
processes (licensing terms). This allows OS X its status as the
only successful, commercial UNIX for a desktop as well as its
status as an easy-to-use, GUI driven development platform
(something no other UNIX can claim as convincingly), while
maintaining a clean integration with the lineage of previous
versions of Macintosh operating systems.

\startblockquote
\startblockquote
\startblockquote
\startblockquote
This is where that chapter ended mysteriously ;)
\stopblockquote
\stopblockquote
\stopblockquote
\stopblockquote

\chapter{Design Beyond the Proprietary}

Much of the prominent research within the field of software studies
thus far has been dedicated to investigating the operation of
various proprietary applications (Fuller 2000; Manovich 2008).
Culturally speaking this may make sense: study what the users
actually use. However, from a moral or political angle it can
easily be seen as irresponsible or, at the very least, incomplete.
Use of proprietary operating systems and application software still
seems to dominate the field of new media. This is in spite of Hans
Magnus Enzensberger's classic critique of \quote{repressive}
media---if you do not control the mechanisms of your medium, those
mechanisms can be used to control you.

Whether for political or economic reasons, there is a small, yet
increasing, number of designers and artists who have begun to use
FLoSS either exclusively or at some stage in their workflows.
Economic considerations should not be underestimated: Adobe charges
no less than \$599.99 for a new (non-upgraded) version of
Photoshop, while bundles of their software sell for \$1699
(Production Premium CS5) or \$1899 (Design Premium CS5).
\footnote{All prices sourced from the Adobe online store on 18 June 2010
(\useURL[1][http://store1.adobe.com/cfusion/store/index.cfm?store=OLS-US&view=ols_cl&nr=0][][http://store1.adobe.com/cfusion/store/index.cfm?store=OLS-US\&view=ols\letterunderscore{}cl\&nr=0]\from[1])}
That makes Photoshop more expensive than an entry level desktop
system, and the bundles each more expensive than a
higher-performance computer! The long-standing solution to this
situation is simply piracy. The newer solution, really only
available for the last decade or less, is to use FLoSS software to
accomplish these goals instead.

Though at a superficial level one can highlight FLoSS design
programs as mere reproduction of proprietary process---the GNU
Image Manipulation Program (GIMP) reproduces the process of photo
editing that is the domain of Adobe Photoshop---Florian Cramer
declares that free softwares are not simply cheap imitations or
reproductions of proprietary software (Cramer 2010). Instead, he
argues, they represent new and unique avenues for accomplishing a
given task.

The degree to which the shape of our tools shapes that which we
create with those tools has long been established within media
theory, though the focus on the dynamics of given medium
specificities often overshadow the processes through which those
media are filled with content.

Why do relatively so few professional designers use FLoSS? For
designer and free software advocate ginger coons
\footnote{ginger coons does not capitalize her names for moral reasons
related to the judgment implications of capitalizing some nouns
over others.},
the answer lies in a feedback loop between academia, industry, and
the designers themselves:

Graphics professionals learn proprietary software in school because
the industry runs on it. Employers demand knowledge of specific
software because it is the norm. Connected industries, like
printing, run on those de facto standards because of their clients.
This feedback loop cements the place of proprietary software in the
graphics industry. (coons 2010)

Part of the problem lies in the evolutionary nature of the open
source development model. Whereas proprietary programs are expected
to be \quotation{feature complete} with any given release, FLoSS
integrates features as they are added. Any potential proprietary
competitor to Adobe Photoshop, for instance, has to ship its first
version under the expectations of proprietary software: a) it
should offer at least the features that are used every day in
Photoshop; and b) it should be stable. Otherwise there is no reason
for a user to purchase this new proprietary option over the older,
more mature one. Also, if coons' formulation of the feedback loop
supporting proprietary software is correct, a new proprietary
entrant into the field of image manipulation faces vertical
economic barriers to adoption: it isn't an industry standard, so
the industry isn't hiring those who only know that software. Thus,
coons' formulation not only explains why professionals are slow to
adopt FLoSS alternatives, it also serves to demonstrate why there
is no competition in the space of professional proprietary image
manipulation programs---competition, insofar as it exists, is
solely within the consumer space where relatively few features are
needed to meet user requirements.

\section{Traditional Design Interfaces in FLoSS}

In a personal conversation between myself and professional web
designer Andy Fitz, he explained how he came to use FLoSS. In
college, the Windows operating system on his computer broke down.
His roommate restored the computer, but replaced the operating
system with a variant of GNU/Linux. After realizing that this meant
no more working with Photoshop, Andy gave GIMP a try. Frustrated by
the difficulty of transitioning, he complained to his roommate that
GIMP simply was not powerful enough. It didn't have all the
features of Photoshop. Nothing was in the same place.

\quotation{How long did it take you to learn Photoshop?} his
roommate replied.

\quotation{Five years,} replied Andy.

\quotation{Learn GIMP for five years. Then you can complain.}

Five years later and counting, Andy Fitz is no longer complaining.

\section{Software Time}

Five years is both a long and short time for FLoSS. It is long
because a piece of software that hasn't been updated for five years
not only seems horribly outdated, the chances are high that it will
not even compile against the versions of the software it depends on
that exist today. It is also a short time because there are many
projects for which five years is just a small slice of the time it
has been in development. For instance, GNUStep (a free software
implementation of the same NeXTSTEP technologies behind Cocoa in
Mac OS X) has been in development since the mid--1990s, which means
it predates the development of Cocoa by five or more years itself.
The typesetting system {\TEX} began development 32 years ago---while
the main program is no longer extended, new variations that fix
longstanding issues with the original implementation continue to
evolve. Five years can be a long time in terms of new features and
code growth, yet it is not necessarily a long time in the lifespan
of a FLoSS project.

Scott Rosenberg provides an important investigation into software
time in his book {\em Dreaming in Code} (2007). While previous
books have detailed the dynamics of software management in
important ways---Frederick Brooks' {\em The Mythical Man-Month}
being the prototypical example---they are all engaged in
documenting proprietary development processes (Brooks 1975).
{\em Dreaming in Code} instead focuses on the development of
Chandler, a personal-information manager carrying with it ambitions
of silo-less information management (calender events can be
\quotation{stamped} as emails, a process that grafts the grammar of
email (To:, Subject:, etc) onto the event). While it is an
open-source project, it also has the advantage of a

Five years ago the Inkscape project began. Inkscape is a vector
graphics editor similar in goals to that of Adobe Illustrator. Five
years ago it did not exist. Today it is known for enabling
workflows that incorporate both generativity and WYSIWYG design
approaches. After five years of development, it has effectively
implemented the features necessary for graphic design on the web.

\chapter{Screenic Text}

\section{Text as Interface/Text as Process}

The command-line interface (CLI), once a culturally universal site
of intersection between human and digital process, has found itself
virtually superceded by the visually metaphoric instrumentation of
the graphical user interface (GUI). The mechanism of this
transition from CLI to GUI within mainstream computing was the
introduction of Microsoft Windows into the ecosystem of
IBM-compatible PCs. The result was the injection of an additional
semiotic layer, charged with a new modality of visual
signification, between the user and the hardware (Stephenson 1999).
For almost two decades consumer versions of Windows, however, were
\quotation{DOS front-ends} that could not function without real,
historical dependencies fulfilled by the presence of DOS deep
within the guts of the operating system. Windows 1.0, for instance,
used DOS's file operation functions ({\em Windows 1.0} 2010). This
dependency on DOS recedes over time, eventually disappearing
entirely in Windows XP, in which the DOS interface and
functionality still exists but has migrated out of the substrate
and into a virtual machine ({\em Windows XP} 2010).

The roots of the command line lie in a very physical process: the
teletype. A teletype is resembles a typewriter in that it presented
the users with a standard typewriter keyboard as a control.
Pressing a key would result in an inked stamp of that keys
respective character smacking onto the paper and retract, leaving
its mark. Simultaneously the triggering of the key might be punched
into a tape as a binary sequence representing the character. If so,
the control was thus separated intrinsically between human and
digital---it was not, as in todays keyboards, simply electrical
signals converted into numbers transparently beneath our fingertips
but rather also a physical instantiation of the sequence on a paper
strip. The screen of this human-digital intersection was
instantiated on the same paper as the recording of the input, using
the same ink and stamps but now powered by the response of the
machine to its human input.

Stephenson identifies an extremely formal dynamic of interacting
through teletypic screens he encountered when learning to program
in high school:

Anyway, it will have been obvious that my interaction with the
computer was of an extremely formal nature, being sharply divided
up into different phases, viz.: (1) sitting at home with paper and
pencil, miles and miles from any computer, I would think very, very
hard about what I wanted the computer to do, and translate my
intentions into a computer language---a series of alphanumeric
symbols on a page. (2) I would carry this across a sort of
informational cordon sanitaire (three miles of snowdrifts) to
school and type those letters into a machine---not a
computer---which would convert the symbols into binary numbers and
record them visibly on a tape. (3) Then, through the rubber-cup
modem, I would cause those numbers to be sent to the university
mainframe, which would (4) do arithmetic on them and send different
numbers back to the teletype. (5) The teletype would convert these
numbers back into letters and hammer them out on a page and (6) I,
watching, would construe the letters as meaningful symbols.
(Stephenson 1999)

In this text, titled ¨In the beginning was the command line,¨ Neal
Stephenson proceeds to identify the underlying mechanisms of
human-digital processual intersections: ¨computers do arithmetic on
bits of information. Humans construe the bits as meaningful
symbols.¨ He notes, however, that this act of translation is
increasingly obscured by ever-increasing metaphoric abstraction,
starting with the GUI and carrying on over the course of the
evolution of graphical interfaces. Command-line interfaces are
close to the bottom of the ¨stack¨ of the cross-translation between
symbols and bits, whereas ¨[w]hen we use most modern operating
systems, though, our interaction with the machine is heavily
mediated. Everything we do is interpreted and translated time and
again as it works its way down through all of the metaphors and
abstractions¨ (Stephenson 1999).

Text, as the least abstracted of the available sites of symbol
translation between digital binary forms (which can be considered
the text of a different alphabet) and human process, is the formal
level of computing. As such, the general non-consideration of the
command-line interface in new media discourse is a disservice to
the metamedium with which much of that discourse concerns itself.

One of the angles by which the CLI approached---tangentially---are
discussions of code poetry. Florian Cramer provides a step by step
engagement with computation both within digital contexts and
written language that leads to an important insight:

The cultural history of computation shows that it is as rich and
contradictory as that of any other symbolic form. It encompasses
opposites, algorithms as a tool versus algorithms as a material of
aesthetic play and speculation, computation as inner workings of
nature (as in Pythagorean thought) or God (as in Kabbalah and
magic) versus computation as culture and a medium of cultural
reflection (starting with Oulip and hacker cultures in the 1960s),
computation as a means of abolishing semantics (Bense) versus
computation as a means to structure and generate semantics (as in
Lullism and Artificial Intelligence), computation as a means of
generating totality (Quirinus Kuhlmann) versus computation as a
means of taking things apart (Tzara, cut-ups), software as
ontological freedom (GNU) versus software as ontological
enslavement (Netochka Nezvanova), ecstatic computation (Kuhlmann,
Kabbala, Burroughs) versus rationalist computation (from Liebniz to
Turing) versus pataphysical computation as the parody of both
rationalist and irrationalist computation (Oulipo and generative
psychogeography), algorithm as expansion (Lullism, generative art)
versus algorithm as constraint (Oulipo, net.art), code as chaotic
imagination (Jodi, codeworks) versus code as structured description
of chaos (Tzara, John Cage).

This contradictory nature envelops the command line as well. A tool
at once more powerful and more flexible yet equally more opaque and
unyielding. To begin to understand the command-line is to begin
shooting lit arrows in the dark, lighting fires of process that
will burn the results of their functioning onto your harddrive,
your graphics card, your BIOS, or your network as onto your screen.
The Unix command

\type{rm -rf /*}

will erase the entire contents of that Unix´s filesystem from the
hard drive. The code for \type{rm} loaded into memory survives to
delete itself from a core component of its materiality, that is,
the raw 1s and 0s on the magnetic platters that constitute the
persistent body of the command. It will not, however, survive the
reboot inevitably awaiting such a fubar´d system.

\section{Remediation and the Command Line}

While the modern command line may be a remediated teletype machine,
it is crucial to note that the commands available at the prompt
{\em re-mediate nothing}. The processes embodied in file operation
commands, for instance, instantiate into material effects on hard
drives. They are abstractions of processual hybridization that
results in the same command in the same operating system having the
same effect, in this instance, on the file system. The modules
loaded into the assemblage offering this abstraction depend on the
format of the file system (NTFS, HFS+, ext*, etc.), the
motherboard-to-disk controller protocol (IDE, SCSI, SATA, etc.) and
the driver specificities of that disk controller. All of these
elements are unique, digital assemblages. The
{\em embodied processes} that are typed commands cannot be
accurately held to the standard of a theory that is based on a
conception of mediums as containing and extending previous mediums.

These commands, these arrows flaming into the dark, have only the
output of text in order to satisfy the needs of {\em immediacy}.
There are interactive commands, to be sure. Text editors and email
clients are two commonly abstracted interfaces. However, do to the
natural reliance of the keyboard as the control of the command
line, these interfaces are as likely to rely on combinations (or
\quotation{chords}) of key presses for purposes of navigation and
process instantiation. This is a new type of immediacy only
available within the electronic writing space, an immediacy that
comes with a steep learning curve but which---once
mastered---rewards the user with productive potentials beyond what
is accomplishable through the semiotic abstractions of the GUI.
Some hackers joke that their favorite operating system is their
text editor \type{emacs}, which is noted for accomplishing
everything from coding to typesetting (through the powerful AUCTeX
extension) to emails and calenders to web browsing. All interaction
is accomplished through these chords of key presses. The learning
curve requirement means that
\quotation{so much needs to be filled in} by the user: commands are
clearly \quote{cool,} in the language of Marshall McLuhan (1964).

Neatly obscured and packaged as a \quote{command,} processes such
as the aforementioned \type{rm} are {\em nothing} if not process.
Commands are either latent or instantiated process.

\section{Language is Programming}

As N. Katherine Hayles puts it,
\quotation{screenic text and programming are logically, conceptually, and instrumentally entwined}
(Hayles 2004: 80).

In their book {\em The Alphabetization of the Popular Mind}, Ivan
Illich and Barry Sanders write of the dimensionality that phonetic
alphabets introduce to words. For them, \quotation{language} does
not proceed our capacity to store representations of words as
sounds:

The historian misreads prehistory when he assumes that
\quotation{language} can be spoken in that word-less world. In the
oral beyond, there is no \quotation{content} distinct from the
winged word that always rushes by before it has been fully grasped,
no \quotation{subject matter} that can be conceived of, entrusted
to teachers, and acquired by pupils (hence no
\quotation{education,}learning," and \quotation{school}). For it is
the record in phonetic writing that first carries what is heard
across a chasm separating two heterogeneous eras of speech. The
alphabetic scribe carries what is spoken from the ever-passing
moment and sets down what he has heard in th permanent space of
language. Only with this act can knowledge, separate from speech,
be born. (Illich and Sanders 1988: 7--8)

Though their phrasing is inflammatory in its assured rejection of
the possibilities of \quotation{knowledge} and \quotation{learning}
amongst oral societies, they do however illustrate the new
dimensions of potential arising from the intersection of words and
the phonetic alphabet. The encoding of the {\em sounds themselves}
into writing transforms
\quotation{the page into a mirror of speech,} freezing
\quotation{the flow of speech itself onto the page} (11, 13). For
Illich and Sanders, the resulting new dimensions delivered to
speech through this synthesis of speech and alphabet into language
include nothing less than \quotation{knowledge,}
\quotation{education,} and even \quotation{logic} itself. Though
Illich and Sanders do not use the language of Simondon's
ontogenesis, we can map how, as these dimensions begin to fill and
expand over time, the metastability of the system continuously
fluctuates as new potentials are described and, through that
description, affect the landscape upon which further potentials
unfold. For instance, in 1492 language becomes recognized as an
avenue of control as the Spanish royalty begin the project of
implementing a standard language to consolidate their subjects into
a more rule-able assemblage.

Illich and Sanders note the lineage of Orwell's Newspeak, a product
of the utopian writerly tradition of positioning language as the
means of subjugation (in dystopias) (110). Through Newspeak the
power of the State supplants the exercising of power by
elites---the power of the State is exercised {\em on them}, rather
than by them---as
\quotation{the State has turned into a book that is constantly rewritten}
(111). Illich and Sanders, however, use the term Newspeak to refer
to
\quotation{an approach or an attitude that treats language as a system and a code}
(112). Fitting directly into this phrase, then, is cybernetics
theory.

While digital processes necessarily change the understandings and
uses of any word used to describe them, it is important to note
that for the most part the words of computer science are
appropriations from language that existed before computers. For
instance, the word \quote{program} was first applied to computers
in 1945 to refer to the
\quotation{act of expressing an operation in the terms appropriate for the performance of a computer}
(Illich and Sanders: 113). Thus, the word was appropriated from
physical schedules of performances by human beings in an event and
mapped onto the logospace of computational performance.

If the ¨so-called \quote{language} of physics is a code, a system
of signs, a formal theory, an analytic tool that derives part of
its value from its near-independence from ordinary speech" (Illich
and Sanders: 116), then the `language´of the command-line fits this
description as well, with one further caveat:
{\em text on the command-line is kinetic}.

Florian Cramer provides an important perspective when he notes ¨the
technical principle of controlling matter through the manipulation
of symbols, is the technical principle of computer software as
well¨ (Cramer 2005: 16). This manipulation of symbols underpins the
entirety of the digital assemblage, and in that way the digital
reflects humanity´s relationship with language. In an earlier essay
called ¨Digital Code and Literary Text,¨ Cramer identifies a
privileged relationship between language and binary code (Cramer
2001).

The connection to magic is instantiated culturally in the language
of the hacker class. Firstly, in the formulation of the individual
command-line entry as an `invocation.´ This recognition is
important, for text has never been so kinetic as it is on the
command line. From the literally typed input of the teletype
machines to the virtual terminals with transparent backgrounds and
multi-aphabet encoding running in a GUI, the textual input of the
command-line represents a site of language that promotes words from
their status as {\em signifiers of meaning} to become
{\em signifiers of action}. Not merely the description (evocation)
of action, but the literal {\em invocation} of action through
words.

\section{Language adds dimensions}

Language can be conceived as a \quote{program} for decoding the
strings of symbols we call an alphabet into the meaning those
symbols were arranged in order to convey. Language is thus adding
the dimension of meaning to the digital code of the alphabet.
Contrary to the popular misconstruing of \quote{digital} with
\quote{binary}, the term \quote{digital} just means a system with
discrete units capable of formally representing some {\em thing}.
For instance, alphabets are utilized for calculations in some
fields of informatics. Likewise, characters of the alphabet are
used to stand in for entire mathematical algorithms in the case of
physics. Due to the discrete jumps between characters, they are
utilizable for computation as well as representation (one could
even argue that there is a computation occuring in the decoding of
the representation from signifier to signified).

A good example to demonstrate this aspect of the alphabet is what
is known as hexadecimal notation. Rather than our familiar base--10
(decimal) system of representing numbers, hexadecimal is a base--16
system. Counting from 1--9 is the same as in base--10, but starting
at 10 we run into the difference. Being base--16 means that any
given column in a numerical sequence must \quotation{count} 16
times (including 0). So 10--15 are represented in hexadecimal by
the alphabetic characters A-F. Hexadecimal is a common notation in
computation due to its extreme translatability back and forth
between binary while at the same time maintaining a more compact
system of representing numbers than either binary or decimal
notation. Whether one would consider such utilization of the
alphabet for representing numbers constitutes a remediation is a
question that points to the fragility of the remediation concept:
by focusing on \quote{media}, the term becomes useless to discuss
appropriation between anything that is not considered a medium.
That is, it invites increasingly expansive definitions of
\quote{medium} or risks blocking itself from application in a
diverse range of instances. Processual hybridity, however, is a
means for denoting this intersection of digits and alphabet, thus
providing a handle by which to grasp hexadecimal notation in a
critical context without resorting to materialist theories.

In the words of Florian Cramer, the
\quotation{alphabet of both machine and human language is interchangeable, so that \quote{text}---if defined as a countable mass of alphabetical signifiers---remains a valid descriptor for both machine code sequences and human writing}
(2001). The difference lies in syntax and semantics---that is,
\quotation{computer algorithms are, like logical statements, a formal language and thus only a restrained subset of language as a whole}
(Cramer 2001). In recognizing this fact, note that syntax and
semantics also influence the dimensional modulation of language in
its application to systems. Whereas \quotation{language} as a whole
can be conceived as a program for decoding
\quote{alphabetical signifiers} into meaning, the subsets of
language known as programming languages undergo a process of
{\em transcoding}. Formulated as action from its first instance,
words in computer code are {\em signifiers of digital process}
rather than meaning. (Or, in the case of \quotation{codeworks} and
code poetry, the meaning of the words is a meaning constructed of
references to both the signifieds of both human language and
digital process).

The relationship of code to language is that of a subset
constrained by the specificities of syntax (Cramer 2001). The
digital computer is ruled by syntax, which could be considered the
defining means of mediation between digital computers and human
processes. These processes include the actions of the users, the
objects created/stored/distributed/displayed on digital computers,
{\em and the processes by which these digital operations are instantiated}.

¨Computation and its imaginary are rich with contradictions, and
loaded with metaphysical and ontological speculation. Underneath
those contradictions and speculations lies an obsession with code
that executes, the phantasm that words become flesh. It remains a
phantasm because again and again, the execution fails to match the
boundless speculative expectations invested in it.¨ (Cramer 2005:
125).

\section{Top Down, Bottom Up}

\subsection{WYSIWYG}

WYSIWYG, meaning \quotation{What You See Is What You Get,} is a
mode of interface design in which operations are performed in an
extremely top-down manner. In terms of typesetting, the definitive
example of WYSIWYG is Microsoft Word. Word can be deemed a
remediation of the typewriter. However, it may be useful instead to
consider it as an {\em appropriation of the grammar} of the
typewriter. That is to say, rather than speak of it in terms of
\quote{remediation,} which makes less and less sense the more
digitally-unique features are added to the interface of Word. What
does our knowledge that Word remediates the typewriter add to a
discussion of WYSIWYG?

WYSWIWYG is clearly an instance of an attempted
{\em immediacy}---by remediating the then-familiar modality of the
typewriter into the context of the computer screen. But the entire
concept of media seems complicated by this remediation: if the
typewriter is a medium, something which few media theories would
likely argue against, then is Microsoft Word then a medium as well?
This begins a slippery slope: would not all applications become, or
have some claim to be considered as, media?

To avoid this I would argue that what is called remediation is,
rather than media \quotation{consuming and extending} previous
media, a dynamic in which the {\em grammars} of one processual
hybridity (the keys-stamps-ink-paper assemblage of the typewriter)
are appropriated and {\em re-conditioned} by another process (MS
Word).

The WYSIWYG interface is clearly a top down approach, as all
manipulations come from invoking processes onto what has already
been placed into the interface.

\subsection{Processed Text}

Processed text comes in two flavors: {\em semantic} and
{\em formal}. Semantic formats such as HTML and XML are far more
widely used than formal formats such as {\TEX} (and La{\TEX}/Con{\TEX}t).
While both are bottom up in contrast to WYSIWYG, there is a
distinction even here between top down and bottom up. (They are
both bottom up in that the typesetting goals of any block of text
are specified at the beginning of that block, i.e.~\type{<em>} and
`\quote' are both typed before the block of text begins.

HTML is top down, however, because that is its rendering model. By
imbuing blocks of text with semantic qualities, one abstracts away
the process of displaying those semantic blocks. Order is imposed
from above, both through Cascading Style Sheets (CSS) and through
the rendering algorithm of a given browser's implementation.

\chapter{Constraints: Generative Design in FLoSS}

\section{Enter the Conditional}

The design world is increasingly integrating generative approaches
to their workflows. The most explicit manifestation, in writing, of
this impulse is the \quotation{Conditional Design Manifesto}
(Maurer, Paulus, Puckey, Wouters 2008). Written by Amsterdam-based
designers Luna Maurer, Edo Paulus, Jonathan Puckey, and Roel
Wouters, this manifesto outlines the approach that they have named
conditional design and presents its three central elements:
{\em process}, {\em logic}, and {\em input}. The statements of
their manifesto on these topics are reproduced in full below:

\startlongquote
**Process**
- The process is the product.

- The most important aspects of a process are time, relationship and change.

- The process produces formations rather than forms.

- We search for unexpected but correlative, emergent patterns.

- Even though a process has the appearance of objectivity, we realize the fact that it stems from subjective intentions.


**Logic**   

- Logic is our tool.

- Logic is our method for accentuating the ungraspable.

- A clear and logical setting emphasizes that which does not seem to fit within it.

- We use logic to design the conditions through which the process can take place.

- Design conditions using intelligible rules.

- Avoid arbitrary randomness.

- Difference should have a reason.

- Use rules as constraints.

- Constraints sharpen the perspective on the process and stimulate play within the limitations.

**Input**
- The input is our material.

- Input engages logic and activates and influences the process.

- Input should come from our external and complex environment: nature, society and its human interactions.
\stoplongquote

\quotation{The process is the product,} the manifesto declares.
Employing the
\quotation{methods of philosophers, engineers, inventors and mystics,}
the four authors of the manifesto seek to abandon the idea of a
product in favor of
\quotation{things that adapt to their environment, emphasize change and show difference.}

These principles have been applied both within and outside of the
domain of software. For instance there is the application of
cellular automata rules (rules which algorithmically govern the
placement of points on an iterative graph) to seating in an
auditorium. By invoking rules based on gender, the application of
generative design into the physical world of seating arrangements
provides angles of critical engagement with gender politicals and
social organization.

Dutch designer Pietr van Blokland, faced with the task of producing
brochures in 32 different languages as the house designer of
Rabobank, has \quotation{thrown away} Adobe and WYSIWYG design in
favor of a generative workflow. In fact, the firms constituting the
cutting edge of design in the Netherlands work within the context
of generative processes (Cramer, Mansoux, and Murtaugh 2010).
Florian Cramer identifies this transition to the generative as an
opportunity for free software to become assert a dominant position
in the field.

\section{Generative Design is driven by text}

As Lev Manovich notes, the program Processing is gaining an
increasing degree of marketshare among designers---in fact,
Manovich singles out Processing as
\quotation{coming close to [Alan] Kay's vision} of an easy-to-use
programming environment that allows users to
\quotation{develop complex media programs and also to quickly test out ideas}
(Manovich 2008: 79--80).

Processing requires hand-coding of generative algorithms, but it
provides extensive support in both a simplified syntax (in relation
to Java) and a large set of library functions that enable easy
integration of common processes.

\section{Cases}

\subsection{The Piet Zwart Institute}

Working with students from a diverse range of backgrounds, the
Networked Media design program at Piet Zwart is guided by a
\quotation{simple formula}---\quotation{it's not about designing {\em with} media, but it's design {\em of} media}
(Cramer, Mansoux, and Murtaugh 2010). In their joint presentation
of three of the faculty of this program at the Libre Graphics
Meeting 2010 in Brussels, Florian Cramer explains that
\quotation{free software provides the building blocks---the Lego bricks---for this kind of self-created media practice.}
\footnote{All quotes in this section, unless otherwise denoted, are from the
presentation.}

Not all media designed at the Piet Zwart Institute is powered by
software: for example, the Open Streetlamp is nothing more than an
ornamented wooden box .

Cramer mentions the program's stance against the division of
programmer and designer, citing a
\quotation{classical failure of new media projects}---the lack of
shared language between designer and coder leads to lackluster
results.

Proprietary program interfaces are a product of the 1980s. The
transition from \quotation{manipulation to more symbolic thinking}
is a positive aspect. Mansoux notes that graphical interfaces for
design can become an \quotation{annoyance} that
\quotation{locks artists into constrained workflows.}

\quotation{We don't simply want free versions of existing tools\ldots{} What we are more interested in is to see FLoSS as an entry point into a different media practice based on the comprehensive critical rethinking of communication in its relation to technology}

They push their students to use git.

\subsection{Open Source Publishers}

Femke Snelting and Pierre Huyghebaert are two members of Open
Source Publishers, a collection of designers who collaborate on
projects using only free or open source software. In a personal
interview, they shared their reasons for going to open source. It
was an apparent mixture of dissatisfaction with the interfaces of
the proprietary offerings, which they found to be holding them
back, and the ideology of free software itself (Snelting and
Huyghebaert 2010).

In FLoSS, OSP finds \quotation{liberating constraints,} a
sensibility that finds form in a text prepared for
{\em Libre Graphics \#0!}, a magazine released at the Libre
Graphics Meeting 2010:
\quotation{With FLOSS, the resistance of the tool is now for us such a daily meal, that it has become a work field, an investigation space, and a playground}
(OSP 2010).

Whereas the constraints of Adobe products were enough to inspire
these designers to consider dropping them as tools, the constraints
of open source seem to inspire OSP. Adobe products are shaped by
the demands of industry, whereas FLoSS products are shaped by the
demands---and self-initiative---of the community.

Snelting again and again invokes constraints as the most engaging
aspect of working with FLoSS. The tools, often unfinished or
limited in some way, necessarily inform the shape of the output
that she is attempting to produce. This echoes one of the
applications of logic found within the Conditional Design
Manifesto:
\quotation{Constraints sharpen the perspective on the process and stimulate play within the limitations.}

\chapter{Crystalized Process: Text That Typesets Itself}

The time has come to for the self-reflective approach of this
thesis to come into play. For a book called {\em Writing Space},
this work by Jay David Bolter provides scant discussion of actual
writing environments on the computer. Originally written before the
expansion of the World Wide Web into the sphere of popular culture,
{\em Writing Space} is concerns itself with
\quotation{the space of electronic writing.}
\footnote{The second edition of the book, published in 2001, was used for
this thesis. While it is updated to include the Web, its roots in a
significantly older text are worth noting.}
In defining this space as
\quotation{both the computer screen, where text is displayed, and the electronic memory, in which it is stored,}
Bolter belies the relative absence of process in materialist forms
of media analysis (2001: 13).

Bolter's over-simple definition of electronic writing space does
not incorporate the act of writing, only the display of it. This
surface-level analysis fits well the application of his remediation
theory---the surface of a medium (it's \quotation{screen}) is the
host site of remediation. Bolter delves below the surface in his
explanation of a shift to topographical writing. In the electronic
writing space,
\quotation{any relationships that can be defined as the interplay of pointers and elements}
are representable (32). The writing space
\quotation{itself has become a hierarchy of topical elements} (32).
This is the effect of the computer's affinity for
symbol-processing:
\quotation{Any symbol in the space can refer to another symbol using its numerical address}
(30). To highlight the dimensional shift in the writing space,
Bolter describes the operation of outline processors. These
programs abstract a text to the level of sections. These sections
can be moved around and manipulated.

\section{Environment of Operation}

This text is not typed in the manner that you see it. The above
header is instead written like this:

\starttyping
  # Environment of Operation #
\stoptyping

Through the wrapper program \type{pandoc}, this input (written in
Markdown) is converted into HTML and {\CONTEXT} outputs.

\startdescr{HTML}
\letteropenbrace{}
Environment of Operation
\letterclosebrace{}
\stopdescr

\startdescr{{\CONTEXT}}
\letteropenbrace{}
\letterbackslash{}section\letteropenbrace{}Environment of
Operation\letterclosebrace{} \letterclosebrace{}
\stopdescr

The syntax of HTML represents a semantic operation:
\quotation{Dear Mr.~Browser, treat this as a header of level 1.}
The syntax of {\CONTEXT}, however, represents a macro command within a
programming language. What it says is
\quotation{call the sections of code that translate the text within the brackets to the parameters specified for the \type{\section{}} command.}

The literal \quote{writing space} of this thesis is a program
called Textroom. Textroom is a minimalist text editor in which
there are no buttons, taskbars, or other clutter. Only you, your
words, and (optionally) informational text reporting the time, word
count, percentage to accomplishing your writing goal, etc. By
writing in plain-text, I open myself to the opportunities afforded
me by version control systems. Developed to enable collaboration of
programmers on a code base, version control systems can track
changes in text across time (useful for this project) and allow for
massively distributed workflows involved tens of thousands of
individuals (useful for the Linux kernel).

\subsection{Constraints}

\placefigure[here,nonumber]{Textroom and gvim. gvim is displayed on this screen but running on my netbook.}{\externalfigure[images/two_editors.png]}

Above you see a necessary adaptation within my workflow. My netbook
took a fall and lost the ability to use its screen. Because of the
nature of the command line, I was able to log in to computer and
execute commands that allowed me to establish remote access using a
piece of software called \type{ssh}. This remote connection can
also support the transmission of GUI applications using the
client-server model at the heart of the X Windows system (which
drives the GNU/Linux GUI). This was important a) to get files off
the netbook, and b) the version of \type{pandoc} on my desktop had
stopped working after a modular dependency was upgraded and I was
finding it impossible to upgrade \type{pandoc} itself. In the image
above we see the minimal editor Textroom and the editor gvim.
Because of the pandoc incompatibility on the desktop, I was forced
to use the netbook as the site of typesetting. This shaped the
output of the project most likely by time. However, the simple
ability to log in and enable remote access shaped this project
immensely by allowing me to extract important work that would
otherwise have been lost.

An unfortunate constraint is the inability to take advantage of
elements of the {\TEX} landscape that are reknowned for making life
easier. The chief among these is BibTeX, which allows for a
bibliography to be dynamically generated and citations to be
inserted according to a variety of formats (that one can change
with a single line of text, if desired). By abstracting myself from
{\TEX} by using Markdown as the \quotation{pre-format,} I've lost the
opportunity to easily manage bibliographic data and instead must
input it by hand. That said, the MLA format is not currently
available in BibTeX meaning that---even if I could use this
software---the output would be necessarily shaped by the
constraints of the tools.

\startworkscited

Birkel, Garrett. (2004). ¨The Command Line In 2004¨. Web.
\letterless{}\useURL[2][http://garote.bdmonkeys.net/commandline/index.html][][http://garote.bdmonkeys.net/commandline/index.html]\from[2]\lettermore{}
(last accessed 20 June 2010).

Bolter, Jay David. (2001).
{\em Writing Space: Computers, hypertext, and the remediation of print}.
New York: Routledge. Print.

Bolter, Jay David and Richard A. Grusin. (1996).
\quotation{Remediation}. {\em Configurations} 4:3. PDF. Bringhurst,
Robert. (2008).
{\em The Elements of Typographic Style, version 3.2}. Vancouver:
Hartley \& Marks. Print.

Brooks, Frederick. (1975).
{\em The Mythical Man-Month: Essays on software engineering}.
Massachusetts: Addison-Wesley. PDF.

coons, ginger. (2010). \quotation{Why F/LOSS, why not F/LOSS}.
{\em Libre Graphics \#0!}. Belgium: Drukkerij Bulckens nv. Print.

Cramer, Florian. (2001).
\quotation{Digital Code and Literary Text}. {\em netzlituratur}.
Web.
\letterless{}\useURL[3][http://www.netzliteratur.net/cramer/digital_code_and_literary_text.html][][http://www.netzliteratur.net/cramer/digital\letterunderscore{}code\letterunderscore{}and\letterunderscore{}literary\letterunderscore{}text.html]\from[3]\lettermore{}
(last accessed 5 June 2010).

Cramer, Florian. (2005).
{\em Words Made Flesh: Code, Culture, Imagination}. Rotterdam: Piet
Zwart Institute. PDF.
\letterless{}\useURL[4][http://pzwart.wdka.hro.nl/mdr/research/fcramer/wordsmadeflesh/wordsmadefleshpdf][][http://pzwart.wdka.hro.nl/mdr/research/fcramer/wordsmadeflesh/wordsmadefleshpdf]\from[4]\lettermore{}

Cramer, Florian and Matthew Fuller. (2008). \quotation{Interface}.
In {\em Software Studies: a lexicon}, edited by Matthew Fuller. MIT
Press: Cambridge. Print.

Cramer, Florian, Aymeric Mansoux and Michael Murtaugh. (2010).
\quotation{How to Run an Art School on Free and Open Source Software}.
Presentation at the Libre Graphics Meeting 2010, Brussels. Online
video.
\letterless{}\useURL[5][http://river-valley.tv/how-to-run-an-art-school-on-free-and-open-source-software/][][http://river-valley.tv/how-to-run-an-art-school-on-free-and-open-source-software/]\from[5]\lettermore{}
(last accessed 20 June 2010).

Fuller, Matthew.
\quotation{It looks like you're trying to write a letter: Microsoft Word}.
2000. Web.
\letterless{}\useURL[6][http://www.nettime.org/Lists-Archives/nettime-l-0009/msg00040.html][][http://www.nettime.org/Lists-Archives/nettime-l--0009/msg00040.html]\from[6]\lettermore{}
(last accessed 5 June 2010).

Galloway, Alexander. (2010). \quotation{Interface}. Presented at
{\em A wedge between public and private conference} on 22 April
2010, Amsterdam.

Gitelman, Lisa. (2008).
{\em Always Already New: Media, history, and the data of culture}.
Cambridge: MIT Press. Print.

Gillimore, Dan. (2010). ¨This Mac devotee is moving to Linux¨.
{\em Salon.com}. 20 June 2010. Web.
\useURL[7][http://www.salon.com/technology/apple/index.html?story=/tech/dan_gillmor/2010/06/20/from_mac_to_linux][][http://www.salon.com/technology/apple/index.html?story=/tech/dan\letterunderscore{}gillmor/2010/06/20/from\letterunderscore{}mac\letterunderscore{}to\letterunderscore{}linux]\from[7]
(last accessed 21 June 2010).

Hagen, Hans. (2009). {\em The history of luaTeX}. Netherlands:
Pragma ADE. Web.
\letterless{}\useURL[8][http://www.pragma-ade.com/general/manuals/mk.pdf][][http://www.pragma-ade.com/general/manuals/mk.pdf]\from[8]\lettermore{}
(last accessed 5 June 2010).

Hayles, N. Katherine. (2004)
\quotation{Print is Flat, Code is Deep: The Importance of Media Specific Analysis}.
{\em Poetics Today} 25:1. PDF.

Hayles, N. Katherine. (2008).
{\em Electronic Literature: new horizons for the literary}. Notre
Dame: University of Notre Dame. Print.

Holt, Jason and Tom Miller. (2010).
\quotation{Introducing the Google Command Line Tool}.
{\em Open Source at Google} blog. 18 June 2010. Web.
\letterless{}\useURL[9][http://google-opensource.blogspot.com/2010/06/introducing-google-command-line-tool.html][][http://google-opensource.blogspot.com/2010/06/introducing-google-command-line-tool.html]\from[9]\lettermore{}
(last accessed 18 June 2010).

% we should have an open works-cited going
\stopworkscited

\stoptext

