%\enableregime[utf]  % use UTF-8

\setupcolors[state=start]
\setupinteraction[state=start, color=black] % needed for hyperlinks

\usemodule[simplefonts]
\setmainfont[linlibertineo]
\setmonofont[inconsolata][11pt]

\setuppapersize[A4][A4]  % use A4 paper
%\setuplayout[width=middle, backspace=1.5in, cutspace=1.5in,
%             height=middle, header=0.75in, footer=0.75in] % page layout
%\setuppagenumbering[alternative=doublesided, location={header,margin}]  % number pages
\setuppagenumbering[location=]
\setupbodyfont[12pt]  % 12pt font
\setupwhitespace[medium]  % inter-paragraph spacing

\setupinterlinespace[line=18pt] % should be 3/2 spacing

\setupcaptions[headstyle=smallcaps]

\setupindenting[medium]
\indenting[always]

\setuphead[section][style=\sc, alternative=margin]
\setuphead[subsection][style=\sc, alternative=margin]
\setuphead[subsubsection][style=\sc, alternative=margin]

\setupheadertexts
	[{\sc \getmarking[chapter]}]	[{\pagenumber}]
	[{\sc \getmarking[Doctitle]}] [{\pagenumber}]	

% this is for truly empty pagebreaks
\definepagebreak
  [mychapterpagebreak]
  [yes,header,right]
%\setuphead
%  [chapter]
%  [page=mychapterpagebreak]

% setup title page
\unprotect
\definemarking[Author]
\definemarking[Doctitle]

\def\doctitle#1{\gdef\@title{#1}\marking[Doctitle]{#1}}
\def\docsubtitle#1{\gdef\@subtitle{#1}}
\def\author#1{\gdef\@author{#1}\marking[Author]{#1}}
\def\date#1{\gdef\@date{#1}}
\date{\currentdate[day, month, year]}  % Default to today unless specified otherwise.

\def\maketitle{
  \startalignment[center]
    \blank[force,2*big]
      {\scd \@title}
		\blank[big]
			{\scb \@subtitle}
	\stopalignment
	\startalignment[flushleft]
    \blank[25*big]
		\starttabulate[|l|l|]
    			\NC Name: \NC \@author
			\NR \NC Student Number: \NC 6100473
			\NR \NC Email: \NC john.haltiwanger@gmail.com
			\NR \NC Website:	\NC http://drippingdigital.com/
			\NR
		  \NR	\NC Date: \NC \@date
			\NR \NC Supervisor: \NC Richard Rogers
			\NR \NC Second reader: \NC Geert Lovink
			\NR \NC Institution: \NC Universiteit van Amsterdam
			\NR \NC Department: \NC Media and Culture (New Media)
		\stoptabulate
		\blank[3*big]
		\starttabulate[|l|]
		\NC {\sc Keywords} \NR
		\stoptabulate
		medium theory, ontogenesis, transduction, generative design, process oriented perspective, typesetting, ideological computing

%		{\tfa Author: \@author}
%		\blank[medium]
%		{\tfa Student Number: 6100473}
%		\blank[medium]
%		{\tfa Supervisor: Richard Rogers}
%		\blank[medium]
%		{\tfa Program: New Media}
%		\blank[medium]
%		{\tfa Faculty: Media \& Culture}
%		\blank[medium]
%		{\tfa Date: \@date}
\stopalignment}
\protect


% define descr (for definition lists)
\definedescription
  [descr]
    [headstyle=bold,style=normal,align=right,location=hanging,
  width=broad,margin=1cm]

% prevent orphaned list intros
\setupitemize[autointro]

% define defaults for bulleted lists 
\setupitemize[1][symbol=1][indentnext=no]
\setupitemize[2][symbol=2][indentnext=no]
\setupitemize[3][symbol=3][indentnext=no]
\setupitemize[4][symbol=4][indentnext=no]

\setupthinrules[width=15em]  % width of horizontal rules


% define a special head type of bibliography
\definehead		[bibliography] [chapter]
\setuphead		[bibliography] [number=no]
%\definecombinedlist		[content][chapter,section,bibliography]
\setuplist		[bibliography] [headnumber=no]

\definehead		[intro]	[chapter]
\setuphead		[intro]	[number=no]
\definecombinedlist		[content][intro,chapter,section,subsection,subsubsection,bibliography]
\setupcombinedlist		[content][alternative=c,interaction=all]
\setuplist		[intro]	[headnumber=no]

% let's get pretty chapters
\def\MyChapterCommand#1#2% #1 is number, #2 is text
  {\framed[frame=off,bottomframe=on,topframe=on]
     {\vbox{\headtext{chapter} #1\blank#2}}} % \vbox is needed for \blank to work
\def\MyEmptyChapterCommand#1#2% is a comment necessary?---apparently so...
	{\framed[frame=off,bottomframe=on,topframe=on]
			{\vbox{#2}}}

\setuphead[chapter][command=\MyChapterCommand, style={\sc},page=mychapterpagebreak,header=empty]

\setuphead[bibliography][command=\MyEmptyChapterCommand, style={\sc},page=mychapterpagebreak,header=empty]

\setuphead[intro][command=\MyEmptyChapterCommand, style={\sc},page=mychapterpagebreak,header=empty]

\setupheadtext[chapter=Chapter] % used by \headtext


% for block quotations
\unprotect

\startvariables all
blockquote: blockquote
\stopvariables

\definedelimitedtext
[\v!blockquote][\v!quotation]

\setupdelimitedtext
[\v!blockquote]
[\c!left=,
\c!right=,
before={\blank[medium]},
after={\blank[medium]},
]

% for long quotes
\definestartstop
  [longquote]
  [before={\indenting[never]
    \setupnarrower[left=0.5in,right=0.5in]
    \startnarrower[left,right]},
  after={\stopnarrower
    \indenting[yes]}]

% for bibliographic entries

% following hanging indent code (also in workscited) taken from 
%  http://www.ntg.nl/pipermail/ntg-context/2004/005280.html
% [NTG-context] Re: Again: "hanging" for a lot of paragraphs?
%  ~ Patrick Gundlach
\def\hangover{\hangafter=1\hangindent=0.5in}
\definestartstop[workscited][
  before={
    \page[no]
    \indenting[never]
    \startalignment[left]
    \bibliography{Bibliography}
    \stopalignment
    \setupwhitespace[medium]
    \bgroup\appendtoks\hangover\to\everypar
    },
  after={
    \egroup
    \indenting[yes]}]

\protect

\definestartstop
  [abstract]
  [before={\blank[4*big]
					 \midaligned{\sc Abstract}
           \startnarrower[2*middle]},
   after={\stopnarrower
          \blank[big]}]

\setupheader[state=start]


\starttext
	\doctitle{Grammars of Process}
	\author{John Haltiwanger}

\docsubtitle{Mapping Ontogenesis in Generative Design}


%\setuppagenumbering[location=]
\setupheader[state=stop]
\maketitle
\page

%\setuppagenumbering[alternative=doublisided, location={header,margin}] 


\startabstract
Here goes the abstract!

Will it do multi-line stuff for me?

Eh????
\stopabstract
\page
\setupheader[state=start]

\placecontent
%\startfrontmatter
%\intro{Acknowledgments}

%This is where I acknowledge everyone.

%\intro{Introduction}

%This is a chipper little introduction, don't you think?
% should work in general, i hope!

%\stopfrontmatter
\section{Acknowledgments}

\section{Introduction}

Today's new media theory increasingly invokes {\em materiality} as
a significant, perhaps even {\em the} significant, mode of
investigating digital objects and the media through which they are
delivered. This thesis questions such a centrality of materiality
through a practice-based, process-oriented approach. {\em Process}
is proposed as the atomic unit of that which new media theory
investigates. This is true on a formal material level: applications
run as either as individual process or as assemblages of process
which are managed by an operating system and through which the
application's code is accomplishes all of its tasks, from memory
and access to algorithmic execution on the central processing unit.
A process-oriented approach will be shown to provide superior
methodologies for engaging with and understanding software than
material analysis alone provides. For instance, certain
problematics within Lev Manovich's concept of
\quote{media hybridity} will be resolved by a re-orientation
towards process (Manovich 2008). Process also allows a fresh
perspective for examining human-digital relations. Human processes
and digital processes are seen as inextricably intertwined, leaving
any discussion of digital process that excludes relevant dimensions
of human process necessarily unfinished.

=== FRESH ===

In this introduction, I will first briefly re-trace the vectors of
medium theory as they have developed since the introduction of
\quote{new media} as an academic institution. Such re-tracing
necessarily begins with McLuhan, as his theory is intrinsic to one
of the earliest theoretical frameworks of new media, the
{\em remediation} model of Jay David Bolter and Richard Grusin.
This framework was eventually superceded by a discipline-wide turn
towards investigations of materiality and medium specificity,
spearheaded by the work of N. Katherine Hayles. Finally, Lev
Manovich's manuscript {\em Software takes command} provides
concepts of {\em media hybridity} and {\em deep remixability}.
Throughout this swath of theory is woven an intrinsic focus on the
{\em material} modes of media. Media hybridity, for instance,
relies on a medium having a specific dimensionality that is
enlarged or otherwise augmented through hybridization with other
mediums. In a significant example, the dimension of typography
obtains velocity and physicality as it enters the 3D void of the
After Effects window. I pose the following question to this
explanation [WHUT WASS IT AGGIN\lettermore{}\lettermore{}??].
Rather than \quote{dimensionality,} I propose that we conceptualize
these hybridizations as generating new levels of metapotentiality.
This is a transposition of Gilbert Simondon's language of
ontogenesis into new media discourse. Simondon utilizes the ideas
of metastability and metapotential to describe active forms of
those concepts: a metastability is likened to a substrate in which
massive activity occurs. One example is a fluid suspension that
happens to contain ideal conditions for crystalization: the
metastability of the suspension drives its crystalization, and
crystalization is the actualization of the suspension's
metapotential. Yet crystalization is a singular process, a form of
individuation for which identical crystals are said not to exist in
nature. The metapotential of each substrate is fulfilled uniquely.
And in natural or otherwise unbounded systems, crystals are often
on-going processes of {\em individuation}---Simondon's term for the
movement of a metastability through the courses of its own
metapotential. This thesis proposes that grammars are a key factor
of enabling hybridization. Hybridization, in turn, expands and
extends the metapotentials of involved processes. Thus, grammars
are crucial mechanisms through which the metastability expands
along its metapotential.

The method proposed to demonstrate these points is two-fold. The
first is an analytic approach---the modes of operation of designers
themselves are examined. Starting from the proprietary Mac OS X
operating system, described here as a unique and powerful example
of {\em process hybridity}, we progress to a discussion of the
operations of designers as constrained by FLoSS
(Free/Libre/open/Source Software). The second aspect of the method
is a detailed interrogation of actual practice in the form of
{\em digital typesetting}. This topic was chosen for several
reasons. The first is a general lack of focus on the processes
behind typesetting among new media theory---while the surfaces of
text have been investigated in numerous ways (Bolter 2001; Fuller
2000), there has been a general lack of concern (or capacity)
regarding the underlying processes of text in the metamedium
(computers). This is especially evidenced as regards the
{\em command line interface}, a realm where text becomes kinetic.
Yet I found that very little theory has been written regarding the
command-line, despite its place as the historical interface (once
contemporary with batch punch cards) by which digital processes
were initiated. Far from being obsolete, both Microsoft and Apple
ship command line interfaces within their operating systems. In
Microsoft´s case, significant money has been spent developing a new
grammar and implementing new functionalities into their modern
command line implementation Powershell (as opposed to the grammar
and functionalities of DOS).

The second reason for choosing typesetting is the supposed lack of
media hybridity of typesetting---according to Manovich's
definitions of the terms, typesetting has failed to move beyond
\quote{multimedia} to a state of \quote{media hybridity} (this is
opposed to typography, which undoubtedly has) (2008: 86). Media
hybridity is Manovich´s formulation of the increasingly common
¨sharing of languages¨ between media. When media share language,
they develop new dimensions (2008: 86). Language, then,
demonstrates its capacity for modulation in a new context. While
the proposition that ¨language can add dimensions to things¨ may at
first consideration seem a bit too obvious for stating out loud,
the kinetic properties of language within the context of the
metamedium---that the code enabling the language sharing that
enables media hybridity is
{\em itself made of language and made executable by language}---seem
to beg for consideration. Whereas much of the new media discourse
relating to changes in media trends toward contemplating fast-paced
visual cultures such as video games and cinema, this thesis aims to
take the opportunity to contemplate the much slower-moving medium
of text. This contemplation of screenic text leads to questions
about the nature of media within a medium as well as to the
introduction of a conception of processual hybridity that both
underpins and exceeds the dynamics of media hybridity.

The third aspect is the allowance of a truly reflexive
investigation in which multiple processes of digital typesetting
are utilized to generate the thesis itself. This provides a means
to integrate the process-oriented perspective into a software study
of FLoSS typesetting software. Not only this, it provides a means
to attempt what could be considered a {\em refractional}
methodology. Inspired by Gilbert Simondon´s adoption of the
language of chemistry in the formulation of {\em transduction}
within his theory of ontogenesis, this thesis can be viewed as a
distinct crystallization process, the composition of a whole from
the process of that whole´s unfolding. The applicability of
Simondon´s ontogenesis to matters of generative design will be
interrogated in contrast to Jay David Bolter and David Grusin´s
remediation theory (Bolter and Grusin 1996; Bolter 2001).
Ontogenesis, albeit without Simondon, has already proven an
effective angle for approaching Web 2.0 platforms (Langlois,
McKelvey, Elmer, and Werbin 2009). Here the description of this
thesis´own workflow will demonstrate Simondon´s ontogenesis as
making unique contributions to the process-oriented perspective
which this thesis attempts to invoke and instantiate.

The fourth is the simple fact that screenic text has not been
interrogated on a {\em subtextual} level---surface analysis of text
(and hypertext) have driven the discourse of screenic text in new
media.

\section{Screens}

As digital typesetting provides the focus for the application of
the process-oriented perspective, the point of origin is
necessarily that of the screen. Information transmission is
increasingly screen-based, a fact that only intensifies with the
exponentializing ubiquity of mobile devices such as the iPhone. The
long-awaited advent of cheap \quotation{tablet} computers and
e-readers is also now at hand. These devices may all be seen as
mediums for {\em screenic processes} in that their entire
configuration and all of its computation exists to serve as the
basis for screenic interactions with {\em human processes}. These
phrasings introduce the perceptual angle attendent with this
thesis, namely the centering of {\em process} as the atomic unit of
what is discussed in new media theory. The term {\em screenic}
simply means \quote{screen-based,} or (perhaps)
\quote{screen-native.} It is analagous to \quote{printed.}

One way to define screens is in terms of their interactivity. Some
screens, such as television screens, offer very limited
interactivity: the choice of content. This choice itself can be
constrained by varying degrees, such as the number of available
channels and playback formats (VHS, DVD, Xvid, etc.), even to the
point of disappearing (in the case of many televisions that appear
in public spaces).
\footnote{Mobile devices are beginning to ship IR transeivers with full
hardware access through software. That is, the
{\em entire potential} of the IR spectrum is available to them.}
The medium of the remote control should not be underestimated in
its effects on human processes, to say nothing of the screens at
which they are aimed. Indeed, they drive the interactivity of the
video game consoles, an interactivity that clearly represents the
cultural cutting edge of what a television screen can offer.

The computer screen, on the other hand, is defined by its seemingly
limitless degree of interactivity. Remote controls can be run as
screenic processes and can not only change television
channels---processes on remote systems can be controlled with
similar ease. Indeed, the entire screenic composition of one
computer can be controlled over a network by a second computer
using included, or easy to obtain, applications. Furthermore, the
very interfaces to the screen (keyboards and mice) are examples of
remote controls in cases where the screen has not itself become its
own remote control (touch-screen devices). Typically the only
element of a computer screen that the user does not effectively
control are the structure and visual language of an operating
system's graphical user interface (GUI). Even this, however, is
generally accomplishable by a significantly informed user. In the
case of GNU/Linux the task is not only accomplishable: in the case
of a \quotation{from scratch} installation,
\footnote{Such as is demanded by no-frills distributions such as Gentoo and
ArchLinux, where manual installation and configuration of a GUI is
required for use.}
the user is literally forced to make a choice of GUI structure and
visual styling. Microsoft has generally shipped their operating
systems with multiple choices for widget
\footnote{A widget is the technical term for a GUI element. Scrollbars,
titlebars, menus, and close/minimize/maximize buttons are widgets
attached to most of the \quotation{windows} that appears on any
given GUI-driven computer.}
presentation, including re-mediations of widgets from previous
versions of Windows. Users also developed Apple, however, maintains
strict control of widget presentation, especially on their mobile
devices.

\subsection{Screens as material, screens as process}

Screens offer an ideal point of juxtaposition between the material
and processual frames. From a material view, the very formulation
of \quotation{screens} as {\em the} interface between humans and
computers is problematic: what of the interfaces that have been
developed to work around instances of blindness or other
[disabilities] that prohibit visually screenic interaction?

From a processual orientation, the question becomes: how do
interactions between humans and computers resolve themselves? The
answer returns in the form of the {\em available} remote controls
and the {\em available} response interfaces. The next step might be
to investigate the degree of variance between these availabilities,
and whether they problematize any umbrella-classification. While it
would be {\em insensible} to argue that material differences in
inputs and outputs can---or do---not lead to a huge amount of
variation between experiences within humans. Such variation is
likely to occur in differentials. That is to say, the spectrum of
possible feedback occurs at the level of the human
individual---one's experiences are functionally irrepresentable
without translation of some kind. [We can choose to call these
translations mediums, or we can choose to call these processes.]

At this point the question becomes, then, whether it is necessary
to instantiate these inherent divergences in every evocation of a
broad level discussion of input and output mechanisms or whether
the inherent, {\em core} similarity between them all remains that
in all instances they serve as {\em the point of contact} between
human and digital processes. Does it make a processual difference
if the output technology is a braille screen or an LCD screen? Only
inasmuch as to what degree the process being examined is unique to,
or highlights differences between, one or the other. From a
discursive level, {\em controls} and {\em screens} can capture the
essence of these dual \quotation{action spaces} that together form
the single point of contact between human and digital process.

Is it possible to remediate of the term screen into discussions of
previous mediums? For instance could one speak of the
\quotation{screen} of a newspaper or the \quotation{screen} of a
cave wall? What about the \quotation{screen} of a radio? From a
linguistic-conceptual perspective the final example certainly
pushes the limits. From a process perspective, though, the presence
of the radio/what it is playing/what listening choices are
available/how and to what degree does the hardware support
frequency tuning: these questions can all be conceived in terms of
\quote{control} and \quote{screen.} The sounds of a radio do emit,
after all, from the vibrations of a stretched membrane.

This thesis proposes a conceptual-linguistic shift in the
discussions of screens as the {\em site of discourse} through which
digital processes yield the results of their execution. Likewise,
the remote control, or simply {\em control}, is the site of
discourse through which which human processes instigate and extend
into the digital. There is no removing or reducing of this dyadic
assemblage---even when the control and the screen are literally
fused (as in most contemporary smartphones) the distinction between
{\em control} and {\em screen} holds on both a conceptual and
material level. Conceptually, human process still extends through
the control into digital process, which still produces feedback
through the screen. Materially, the screen is a Liquid Crystal
Display driven by a graphics card that interfaces with coded
drivers and display subsystems in the device's operating system.
The control, on the other hand, is the glass suspended over the LCD
which, through one or more of the multitude of available technical
solutions for the process, reads point(s) of contact, pressure, and
vectors (velocity and direction) of movement.

\section{From Screens to Text}

To discuss computer screens one must necessarily engage with the
concept of {\em interface}, a topic that rightfully occupies a
great deal of current new media discourse. Interface, then,
represents one point of departure from our origin. While interfaces
often utilize many visual metaphors (most of them inherited from
the work done developing the first GUI at Xerox's Palo Alto
Advanced Research Lab (PARC) in the 1970s), there are yet few
computer interfaces that do not rely on text as their dominant
mechanism for organizing and presenting a program's internal
capabilities to a user. (Mobile screens, on the other hand,
increasingly display developing trends of icon-only design, though
the web browser remains a popular application). Despite the success
of the GUI over the text-only command-line interface (CLI), text
remains central to contemporary experiences of computer screens.

The command line is seen as a space of contestation for traditional
modes of media analysis. Remediation, for instance, will be
demonstrated as inappropriate for discussing the CLI. As Google has
just recently released a command line interface for interacting
with Google services, I believe a discussion of the command line is
essential for new media (Holt and Miller 2010).

(Unfortunate to note, this historiographic aspect is still
{\bf \quote{to-do}}:

The centrality of text to the experience of computer screens
represents the main avenue by which we proceed from the origin,
constituting a trunk from which many additional concerns fork away
and then face examination. The arguments of the paper are augmented
by the inclusion of a historiography of digital typesetting.
Engaging critically with the history of {\em software itself} is
considered a requisite for responsible software studies: a full
range of influences (economic, cultural, technological) should be
considered in the re-telling of a given processual unfolding. In
this aspect of focus, it extends Lev Manovich's admirable
positioning of history as central to a software study by broadening
the scope of historical considerations.
\footnote{{\bf Note:} This work largely remains unfinished in this draft, as
it became apparent that I needed to work back through more
discussions of basic infrastructural elements such as operating
systems in order to fully describe the assemblage of process upon
which computer-based design is situated.)}
Inspiring this enagement is the work of Robin Kinross, whose
{\em Modern typography: an essay in critical history} is one of but
a few texts covering a history of typography to adequately engage
with the influence of factors outside of that field on the field
itself (Kinross 2004). By integrating a critical history of digital
typesetting with a process perspective, an equilibrium between
human and digital processes will be illustrated.

\subsection{Recognizing the Ontogenesis in Generativity}

In his text {\em The Position of the Problem of Ontogenesis},
Simondon writes,

\startlongquote
By transduction we mean an operation--physical, biological, mental, social--by which an activity propagates itself from one element to the next, within a given domain, and founds this propagation on a structuration of the domain that is realized from place to place: each area of the constituted structure serves as the principle and the model for the next area, as a primer for its constitution, to the extent that the modification expands progresively at the same time as the structuring operation. (Simondon 2009: 11).
\stoplongquote

Note the distinct lack of \quote{computational} in Simondon's list
of operations. Written prior to the advent of Manovich's
formulation of the age of cultural computing, this absence might
simply be read as a matter of temporal context. Nevertheless,
Simondon's solution to the ontogenesis problematic provides a
framework for describing digital processes of a generative nature.

This leads to another important element of this thesis, one that
runs throughout the entirety of itself---the underlying processes
of presentation required to \quote{typeset} the text itself.
Through the utilization of FLoSS software, multiple output formats
will be not only be investigated but also materially instantiated
through a designed mechanism of process---a
{\em processual hybridity}. These output formats represent two of
the top formats currently used to manage and display texts
digitally: HTML and PDF.

The process(es) of their generation offers an attempt at mapping
Gilbert Simondon's language of ontogenesis onto file format
translation or, to begin the project immediately,
{\em individuation}. Coupled with Simondon's individuation is this
concept of {\em transduction}. Repurposed from the language of
chemistry, Simondon's metaphorically images transduction with the
example of a substrate---swelling with {\em metapotential}---that
crystallizes. The final formation is the substrate fulfilling this
metapotential, a fulfillment that arises only through an
unpredictable unfolding involving emergent factors. (The language
of chemistry was likewise appropriated for the term
\quote{interface} (Cramer and Fuller 2008: 149)).

Through this mapping I hope to provide a convincing argument for
shared properties between what I am calling process and
individuation, and between transduction and what I am calling
instantiation.

This relates with the increasingly generative nature of
contemporary design. All of which are generated from a plain-text
file whose syntax conforms to a format standard called
\quote{markdown.} The polycephalous nature of {\em the text itself}
thus demands further branching into a discussion of formats. What
are the attributes of the class of process to which formats belong?
Formats are seen as stable, yet they move like glass (or glaciar)
in the nano-magnitudes of the digital. Formats provide another
point of contrast between process and material perceptual
orientations.

The discussion of generativity provides further means to
demonstrate the equilibrium of human and digital processes.
Analyzed materially, these processes are chunks of code
electronically lifted from hard drive platters, loaded into system
memory, and then executed via the assemblage of chips on the
computer's motherboard by way of instructions from the operating
system currently residing as a mass of memory heaps in RAM chips.
Analyzed {\em processually}, however, these digital processes are
properly seen as deriving from interactions with human beings. That
is to say, digital and human processes are intimately intertwined,
from the design of their physical landscape of execution
(microcircuitry) to the instructions derived from the user. From a
process angle the computer becomes something of an external nervous
system, extending and modifying the realm of human potentiality
even as it surpasses the capacity of a single mind to functionally
comprehend the entirety of its workings.
\footnote{The chips produced by Intel, for example, are too complex for any
single person to ever hope to entirely understand.}

\subsubsection{Print is static, code is process}

The flat/deep distinction proposed by Hayles is, by its
formualation, material. Problematizing this material focus is the
interwoven history of text and code: the lens of typesetting allows
us to focus on a unique intersection of the two. As the
historiographic case will demonstrate, typesetting is a
{\em non-reducible} process (NP-Complete). This non-reducibility of
typesetting reflects the non-reducibility of computational
processing of language, as well as the non-reducibility of
language, as signifier, into that which is actually signified. This
\quotation{turtles all the way down} scenario has intriguing
implications from a process perspective as we investigate the
methods that have been developed in order to work around this
non-reducibility.

When Hayles states that
\quotation{materiality thus cannot be specified in advance; rather it occupies a borderland---or better, performs as connective tissue,}
she is provisionally correct (Hayles 2004: 72). However, this
metaphor-ization of process is exemplary of new media practices:
reference the complex with an abstract metaphor, obscuring complex
and important dynamics with a metaphor. The metaphor works, to be
sure. One could even consider it an ideal formulation. At issue is
the fact that this borderland is not discussed in a technically
correct manner.

\section{Remote Controls}

I think it may be reasonable to take the remote control and use it
to create a metaphor for all human-computer interaction.

Every digital process has, at its origin, a human. The rate of
computation has increased the impact of human-digital processes in
that the results deliver their results faster. The results will
either match the intentions of the originating human process, or
they will not. In the second case we can find the first evidence of
the effects of digital process on human process:
{\em the code behind the digital process will be re-arranged in an attempt to deliver an output that satisfies the intention of the human processes.}
Whether this modulation of the executed code is through
sliders/input boxes/etc within a GUI interface or through direct
reworking of the source code itself, the effect is the same: the
code executed has been re-configured according to the goal of human
process. The result(s) of the digital process, experienced through
a screen, can match, exceed, or fail this goal. In turn, human
process is effected and the next move is made according to new
goals or revised digital processes.

Video games, for example, can easily be represented by this model.
Human process is obviously shaped by digital at the outset: there
are a finite number of actions that a game offers within its
context. In addition, these actions are often presented as pre-set
mappings of action to controller button.

\section{Attributes of Process}

[It needs to be asserted that I am willingly engaging in my own
appropriation of the term \quote{process} outside of any traditions
other than my own. As the process oriented perspective arose under
the looming shadow of the draft deadline, I admit to a lack of
historical perspective on the use of this word in either new media
or other contexts. Withstanding that, however, I sense a real
applicability of this term in the discourse of new media. I'm
looking forward to working on the final draft and using some of
that time to construct historical perspective for this shift to
process. One important angle is Ned Rossiter's work on
\quotation{Processual Media Theory} in his book
{\em Organized Networks} (Rossiter 2007: 166--192), which this
draft does make use of but which I would like to interweave more
deeply. As it stands, this introduction was left relatively alone
for the sake of fleshing out the middle part of this thesis. This
was at the suggestion of the second reader.]

Process is reflective. It's outputs reflect its inputs.
Additionally, process reconfigures the metapotential in any given
system. It's reflectivity, then, has material effect. As it
reflects the inputs into the outputs, the outputs in turn reflect
new (or else simply different) potentials back into the
{\em context} which is the reciprocal contact point in which the
processes began. This language is extrapolative into any set of
intersections. This paper considers just the subset of
human-digital recipricity, and within the relatively static domain
of typesetting.

A new configuration of metapotential in any system results in the
reconfiguration of (all) other systems as well. This fact reflects
the {\em fractal} nature of process---there is a degree of
non-reducibility inherent in any discussion of process, as
ultimately certain factors in its functioning are unknown to us.

\section{Why free software?}

There are multiple points of consideration that lead me to
concentrate on free software. The first is its relative lack of
presence within new media circles. Time and again I arrive at a
conference only to see a room full of computers booted into
proprietary operating systems. While I am not a \quote{zealot} who
disavows any potential use or need for proprietary software, I find
the general population of new media's reliance on proprietary
operating systems---chiefly, by way of personal and anecdotal
evidence, Mac OS X---disturbing. Hans Magnus Enzensberger outlined
in his \quotation{Constituents of a Theory of the Media} the
importance of issues of control with relation to mediums. Let us
move through the juxtaposed elements of repressive versus
emancipatory uses of media which Enzensberger provides and
interrogate them in relation to Mac OS X and GNU/Linux
(Enzensberger 1970: 269):

{\em Repressive versus Emancipatory}

\startdescr{Centrally controlled program vs.~Decentralized program}
This question is answered by asking the question:
\quotation{Where is the source code of the operating system?} In
the case of OS X, the source code resides only within the confines
of Apple's corporate computers. It is likely heavily guarded by
multiple mechanisms. Whereas in the case of GNU/Linux, the
operating system source code is spread across dozens of mirrors on
the Internet as well as the computers of programmers and users
around the world. Each of these copies can be readily modified to
the designs of any given user, demonstrating decentralized (in
fact, distributed) control. Apple maintains sole, central control
of the code and thus fully determines the functional possibilities
of the operating system.
\stopdescr

\startdescr{One transmitter, many receivers vs.~Each receiver a potential transmitter}
This is already demonstrated above: the code for GNU/Linux is
globally distributed across hundreds of thousands of computers.
Each one of these has the ability to modify the software and share
those modifications with anyone who will listen. OS X can be
modified by no one.
\stopdescr

\startdescr{Immobilization of isolated individuals vs.~Mobilization of the masses}
OS X encourages the use of proprietary applications. These
applications have restrictive license that generally allow only one
individual the right to run the application. GNU/Linux, meanwhile,
\stopdescr

\startdescr{Passive consumer behavior vs.~Interaction of those involved, feedback}
A major advantage for both users and developers in a free software
ecosystem is the feedback that occurs between them. Users may
suggest new features at any time. If they have the skill and/or
time, they can add these features themselves. If the addition of
the features is contentious in any way, the contributer can simply
fork the codebase and continue evolving the software in new
directions. In OS X, you run the binaries you are given.
\stopdescr

\startdescr{Depoliticization vs.~A political learning process}
Mac OS X is pro-capitalist and promotes consumer culture. It can
probably be said that it is politically \quotation{neutral} in its
codedness, but this very codedness remains obfuscated and
proprietary. GNU/Linux, in conservative judgment, at least does not
actively promote consumerism. In an idealistic formulation, it
destabilizes the capitalist ecosystem.
\footnote{It is important to note that free software also plays a significant
role in supporting this infrastructure, as the license provides no
recourse on the terms of the softwares use (Pasquinelli 2008).}
It's politics are as multifaceted as its user base. In its
well-deserved reputation as
\quote{taking some work to make it work,} GNU/Linux forces its
users to become active in the system's administration. This induced
learning of an open approach to computer systems could be said to
have political dimension.
\stopdescr

\startdescr{Production by specialists vs.~Collective production}
This seems self-explanatory.
\stopdescr

\startdescr{Control by property owners or bureaucracy vs Social control by self-organization}
Are you getting the picture?
\stopdescr

In a presentation at the Libre Graphics Meeting 2010, Florian
Cramer explains his theoretical positioning of free software as an
entry point into media criticism. Aymeric Mansoux, also of the
Networked Media design faculty at the Piet Zwart Institute and
present with fellow faculty member Michael Murtough, describes the
critical engagement in the error message common to GNU/Linux
distributions, found in the Totem media player program complaining
of a missing codec library that is required to decode common
patent-encumbered media formats such as MPEG-Layer 3 (Cramer,
Mansoux, and Murtaugh 2010). Behind the error message lies an
assemblage of inter-related issues of intellectual property rights,
cultural practices, and media accessibility. This is a clear
instantiation of a \quotation{political learning process.}

On 21 June 2010, Apple changed its privacy settings to allow the
company to
\quotation{collect store and share \quote{precise location data, including real-time the geographic location of your Apple computer or device}}
(Marco 2010).

\subsection{Caveats}

Free software is not, however, a \quotation{magic bullet}---tied to
the open systems theory which is philosophically related to the
underpinnings of the Chicago school of economics, some of the
philosophical foundations of free software,
{\em and especially open source}, need to be interrogated (Cramer,
Mansoux, and Murtaugh 2010; Pasquinelli 2008). Liberation does not
automatically lead to a distribution of tools to all those that
need them. However, even in this instance we see the power of FloSS
in its capacity to inspire critical engagement with media.

\stopfrontmatter
\chapter{Crystalized Process: Text That Typesets Itself}

The time has come to for the self-reflective approach of this
thesis to come into play. For a book called {\em Writing Space},
this work by Jay David Bolter provides scant discussion of actual
writing environments on the computer. Originally written before the
expansion of the World Wide Web into the sphere of popular culture,
{\em Writing Space} is concerns itself with
\quotation{the space of electronic writing}.
\footnote{The second edition of the book, published in 2001, was used for
this thesis. While it is updated to include the Web, its roots in a
significantly older text are worth noting.}
In defining this space as
\quotation{both the computer screen, where text is displayed, and the electronic memory, in which it is stored,}
Bolter belies the relative absence of process in materialist forms
of media analysis (2001: 13).

Bolter's over-simple definition of electronic writing space does
not incorporate the act of writing, only the display of it. This
surface-level analysis fits well the application of his remediation
theory---the surface of a medium (it's \quotation{screen}) is the
host site of remediation. Bolter delves below the surface in his
explanation of a shift to topographical writing. In the electronic
writing space,
\quotation{any relationships that can be defined as the interplay of pointers and elements}
are representable (32). The writing space
\quotation{itself has become a hierarchy of topical elements} (32).
This is the effect of the computer's affinity for
symbol-processing:
\quotation{Any symbol in the space can refer to another symbol using its numerical address}
(30). To highlight the dimensional shift in the writing space,
Bolter describes the operation of outline processors. These
programs abstract a text to the level of sections. These sections
can be moved around and manipulated. Writing becomes
{\em topological} in the sense that they now have a spatial aspect.
Sections of text have obtained a modularity and independence that
allows visual arrangement.
\footnote{Almost twenty years later, layout processors have been for the most
part subsumed by word processors. The layout shifting process has
been hybridized into the increasingly feature-ful assemblages of
word processors where it has not morphed into a niche proprietary
product.}
What he does not talk about is the formats generated by the
Macintosh program that he uses in his figures. He mentions passing
layouts around on floppies between friends, but that assumes a
parity of operating systems and proprietary software (at least in
his example) (!CITE!).

(!!! WHERE WAS I GOING?? !!!)

\section{Environment of Operation}

This text is not typed in the manner that you see it. The above
header is instead written like this:

\starttyping
  # Environment of Operation #
\stoptyping

Through the wrapper program \type{pandoc}, this input (written in
Markdown) is converted into HTML and ConTeXt outputs.

\startdescr{HTML}
\type{<h1>Environment of Operation</h1>}
\stopdescr

\startdescr{ConTeXt}
\type{\section{Environment of Operation}}
\stopdescr

The syntax of HTML represents a semantic operation:
\quotation{Dear Mr.~Browser, treat this as a header of level 1.}
The syntax of ConTeXt, however, represents a macro command within a
programming language. What it says is
\quotation{call the sections of code that translate the text within the brackets to the parameters specified for the \type{\section{}} command.}

The literal \quote{writing space} of this thesis is a program
called Textroom. Textroom is a minimalist text editor in which
there are no buttons, taskbars, or other clutter. Only you, your
words, and (optionally) informational text reporting the time, word
count, percentage to accomplishing your writing goal, etc. By
writing in plain-text, I open myself to the opportunities afforded
me by version control systems. Developed to enable collaboration of
programmers on a code base, version control systems can track
changes in text across time (useful for this project) and allow for
massively distributed workflows involved tens of thousands of
individuals (useful for the Linux kernel).

\subsection{Constraints}

\placefigure[here,nonumber]{Textroom and gvim. gvim is displayed on this screen but running on my netbook.}{\externalfigure[images/two_editors.png]}

Above you see a necessary adaptation within my workflow. My netbook
took a fall and lost the ability to use its screen. Because of the
nature of the command line, I was able to log in to computer and
execute commands that allowed me to establish remote access using a
piece of software called \type{ssh}. This remote connection can
also support the transmission of GUI applications using the
client-server model at the heart of the X Windows system (which
drives the GNU/Linux GUI). This was important a) to get files off
the netbook, and b) the version of \type{pandoc} on my desktop had
stopped working after a modular dependency was upgraded and I was
finding it impossible to upgrade \type{pandoc} itself. In the image
above we see the minimal editor Textroom and the editor gvim.
Because of the pandoc incompatibility on the desktop, I was forced
to use the netbook as the site of typesetting. This shaped the
output of the project most likely by time. However, the simple
ability to log in and enable remote access shaped this project
immensely by allowing me to extract important work that would
otherwise have been lost.

An unfortunate constraint is the inability to take advantage of
elements of the TeX landscape that are reknowned for making life
easier. The chief among these is BibTeX, which allows for a
bibliography to be dynamically generated and citations to be
inserted according to a variety of formats (that one can change
with a single line of text, if desired). By abstracting myself from
TeX by using Markdown as the \quotation{pre-format,} I've lost the
opportunity to easily manage bibliographic data and instead must
input it by hand. That said, the MLA format is not currently
available in BibTeX meaning that---even if I could use this
software---the output would be necessarily shaped by the
constraints of the tools.

\subsection{A pre-format necessarily complicates while it simplifies}

Using the \quote{markdown} pre-format complicates several issues
with typesetting documents in both HTML and ConTeXt. For instance,
ConTeXt includes a \type{\chapter{}} macro, which influences the
numbering of sections so that sections are relative to the chapter
number, rather than to the entire text. Thus the first section of
chapter two will be rendered into text as \quote{2.1}. The issue
arises because HTML has no similar distinction: sections are
related to the depth of the header, such that H1 is the highest
level section (the equivalent of ConTeXt's \type{\chapter{}}). In
markdown, the top-level section appears as in the example above,
that is using a single \quote{\type{#}}. My initial solution was to
include two sections with only one \quote{\type{#}} per chapter,
the first of which I would manually change from \type{\section{}}
to \type{\chapter{}}, with the result that the sectioning of the
chapter fits what is expected---chapter 2, section 1 is numbered as
\quote{2.1}.

This does not satisfy HTML, however, as there is nothing to change
the first single \quote{\#} into: it is already inserting the
highest level section, \quote{H1}. Thus, where ConTeXt begins
numbering the sections within the chapter (2.1, 2.2, etc.), HTML
increments the top-level section number, so that the second single
\quote{\#} increments to \quote{3}, instead of to \quote{2.1}. If a
double \quote{\#\#} is used, the sectioning will appear as we
desire in the HTML, but will appear as \quote{2.0.1} in ConTeXt.
This incompatibility requires either modifying the Pandoc source
code directly or else the creation of a specific \quote{helper}
script to create a sectioning parity between the two output
formats. As I do not know Haskell, I've opted for the second
solution.

Since markdown allows passing TeX commands, I solved the problem by
using \type{\chapter{}} to designate the chapter title. Then, using
a command-line script written in the Ruby programming language, a
copy of the markdown file is created in which \type{\chapter{}} is
replaced by a single \quote{\#} and all subsequent
\quote{\#'s are increased by one}\#' until the next
\type{\chapter{}} is reached. From this copy we generate the HTML
version, while the original can be processed into TeX. Perhaps this
technical description appears to be more of a computer science
discussion than it is a media theory one. However it highlights the
ways in which processes hybridize: the conflicting grammars of
ConTeXt and HTML create a complication which the wrapper program
Pandoc either does not or can not address. To work around this
issue, a separate program (the Ruby interpeter) is used to
integrate a script file which deals with the problem. This is a
common feature of a command-line based workflow: \quotation{glue}
scripts are written in order to fuse processes together. This glue
can result from .

\subsubsection{Regular Expressions and Process Hybridity}

Regular expressions represent another avenue for demonstrating
process hybridity. Since their introduction into the early text
editors \type{QED} and \type{ed} by Ken Thompson, regular
expressions have since been incorporated into many Unix commands
such as \type{grep} and \type{awk}, newer editors such as \type{vi}
and \type{emacs}, and programming languages such as Perl, PHP,
Python, Ruby, and many more ({\em Regular expression} 2010).
Regular expressions are a means for describing parameters of text
searches, whereby arranging esoteric control characters in and
around the text one is looking to find allows for finely tuned
pattern matching.

The hybridity of regular expressions lies in it's adoption by
nearly every major programming language: from Wikipedia, the list
includes
\quotation{Java, JavaScript, PCRE, Python, Ruby, Microsoft's .Net Framework, and the W3C's XML Schema}
({\em Regular expression} 2010). This list of languages refers to
those who have hybridized some form or derivative of Perl's
implementation of regular expressions, which is considered more
robust than Ken Thompson's original version.

Is it possible to say that these programming languages are
\quote{remediating} the regular expressions from Perl? It is not
beyond reason to assert that programming languages are
\quote{mediums}---Ken Thompson has referred to Smalltalk as a new
medium, for instance (!CITE!). However, mapping the term medium
onto a programming language falls into the same trap of stretching
the term medium until it becomes incomprehensible. Do different
versions of the same language, representing different capabilities
and even incompatible syntax changes, constitute separate mediums?
What is useful about applying the term medium here, other than it
enables us to discuss the prolific implementation of regular
expressions as an example of \quote{remediation}?

Programming languages often borrow concepts from each other, as
this example of regular expressions clearly demonstrates. Saying
that Perl remediates C syntax because it uses curly braces and
semi-colons under-emphasizes Perl's own syntax. Rather it seems
more evocative to describe ways in which Perl hybridizes elements
of C's grammar while augmenting them with grammar of its own.

In other words, to say that

\starttyping
my $variable = "value";    # defining a variable in Perl
\stoptyping

is a remediation of

\starttyping
char[5] variable = "value"; /* defining a variable in C */
\stoptyping

is an over-simplification. It obfuscates significant algorithmic
differences in the two approaches by focusing on the surface level
syntax (which is relatively similar) over the significant internal
differences in the way the two languages deal with variables (such
as static versus dynamic typing). The grammar of C is hybridized by
Perl---re-implemented rather than remediated, related yet
irreconcilable. Implementation differences have huge implications
on the utility and functionality of the languages, a theoretical
framework that focuses on surface-level similarities is incapable
of expressing the variation that occurs beneath those similarities.

Rather than a remediation of regular expressions, then, we see a
hybridization of specific grammars of regular expressions, with the
most popularly hybridized grammar deriving from the version found
in Perl. However, many of the languages that hybridize the Perl
version of regular expressions only implement a particular subset
of that version. Additionally, extensions may be added that are not
included in Perl. The result is a proliferation of regex---in the
programmer shorthand for \quote{regular expression}---grammars as
they are integrated into various process hybridities such as
programming languages, command line utilities, and text editors.

The website {\em Rubular} stands as an example of how far-reaching
the hybridization of regular expressions has come in terms of
process assemblage complexity (Lovitt 2010). The website utilizes
not only the HTTP protocol that drives the World Wide Web, it uses
AJAX in order to provide real-time representations of pattern
matching within a Ruby interpreter (of which there are many). The
GUI browser is involved in this assemblage by design
\footnote{A zen (?koan?) for the 21\letterhat{}st century: Does a website
truly exist if there is no browser to render it?}.
So is a web framework of some kind, from the looks of it the
increasingly ubiquitous Ruby on Rails There could be an argument
made against such far-reaching hybridity: Ruby can be programmed
interactively, line by line, in it's interpreter. The layers of
code wrapped around the processing of Ruby regexes could be seen as
superfluous---in fact, this is a common attitude of certain hacker
types who look with disdain upon any non-essential fuctionality.
Questions of {\em essentiality} in software remain an
under-discussed topic in new media studies, despite the
ever-present debates among developers on the issue.

\startlongquote
\input tufte
\stoplongquote

\section{Bibliography}

Birkel, Garrett. (2004). ¨The Command Line In 2004¨. Web.
\letterless{}\useURL[1][http://garote.bdmonkeys.net/commandline/index.html][][http://garote.bdmonkeys.net/commandline/index.html]\from[1]\lettermore{}
(last accessed 20 June 2010).

Bolter, Jay David. (2001).
{\em Writing Space: Computers, hypertext, and the remediation of print}.
New York: Routledge. Print.

Bolter, Jay David and Richard A. Grusin. (1996).
\quotation{Remediation}. {\em Configurations} 4:3. PDF. Bringhurst,
Robert. (2008).
{\em The Elements of Typographic Style, version 3.2}. Vancouver:
Hartley \& Marks. Print.

Brooks, Frederick. (1975).
{\em The Mythical Man-Month: Essays on software engineering}.
Massachusetts: Addison-Wesley. PDF. coons, ginger. (2010).
\quotation{Why F/LOSS, why not F/LOSS}. {\em Libre Graphics \#0!}.
Belgium: Drukkerij Bulckens nv. Print.

Cramer, Florian. (2001).
\quotation{Digital Code and Literary Text}. {\em netzlituratur}.
Web.
\letterless{}\useURL[2][http://www.netzliteratur.net/cramer/digital_code_and_literary_text.html][][http://www.netzliteratur.net/cramer/digital\letterunderscore{}code\letterunderscore{}and\letterunderscore{}literary\letterunderscore{}text.html]\from[2]\lettermore{}
(last accessed 5 June 2010).

Cramer, Florian. (2005).
{\em Words Made Flesh: Code, Culture, Imagination}. Rotterdam: Piet
Zwart Institute. PDF.
\letterless{}\useURL[3][http://pzwart.wdka.hro.nl/mdr/research/fcramer/wordsmadeflesh/wordsmadefleshpdf][][http://pzwart.wdka.hro.nl/mdr/research/fcramer/wordsmadeflesh/wordsmadefleshpdf]\from[3]\lettermore{}

Cramer, Florian and Matthew Fuller. (2008). \quotation{Interface}.
In {\em Software Studies: a lexicon}, edited by Matthew Fuller. MIT
Press: Cambridge. Print.

Cramer, Florian, Aymeric Mansoux and Michael Murtaugh. (2010).
\quotation{How to Run an Art School on Free and Open Source Software}.
Presentation at the Libre Graphics Meeting 2010, Brussels. Online
video.
\letterless{}\useURL[4][http://river-valley.tv/how-to-run-an-art-school-on-free-and-open-source-software/][][http://river-valley.tv/how-to-run-an-art-school-on-free-and-open-source-software/]\from[4]\lettermore{}
(last accessed 20 June 2010).

Fuller, Matthew.
\quotation{It looks like you're trying to write a letter: Microsoft Word}.
2000. Web.
\letterless{}\useURL[5][http://www.nettime.org/Lists-Archives/nettime-l-0009/msg00040.html][][http://www.nettime.org/Lists-Archives/nettime-l--0009/msg00040.html]\from[5]\lettermore{}
(last accessed 5 June 2010).

Galloway, Alexander. (2010). \quotation{Interface}. Presented at
{\em A wedge between public and private conference} on 22 April
2010, Amsterdam.

Garfinkel, Simson, Daniel Weise, and Steven Strassman (editors).
(1994). {\em The UNIX-HATER´s Handbook}. San Mateo: IDG Worldwide.
PDF.
\letterless{}\useURL[6][http://web.mit.edu/~simsong/www/ugh.pdf][][http://web.mit.edu/\lettertilde{}simsong/www/ugh.pdf]\from[6]\lettermore{}.

Gitelman, Lisa. (2008).
{\em Always Already New: Media, history, and the data of culture}.
Cambridge: MIT Press. Print.

Gillimore, Dan. (2010). ¨This Mac devotee is moving to Linux¨.
{\em Salon.com}. 20 June 2010. Web.
\useURL[7][http://www.salon.com/technology/apple/index.html?story=/tech/dan_gillmor/2010/06/20/from_mac_to_linux][][http://www.salon.com/technology/apple/index.html?story=/tech/dan\letterunderscore{}gillmor/2010/06/20/from\letterunderscore{}mac\letterunderscore{}to\letterunderscore{}linux]\from[7]
(last accessed 21 June 2010).

Hagen, Hans. (2009). {\em The history of luaTeX}. Netherlands:
Pragma ADE. Web.
\letterless{}\useURL[8][http://www.pragma-ade.com/general/manuals/mk.pdf][][http://www.pragma-ade.com/general/manuals/mk.pdf]\from[8]\lettermore{}
(last accessed 5 June 2010).

Hayles, N. Katherine. (2004)
\quotation{Print is Flat, Code is Deep: The Importance of Media Specific Analysis}.
{\em Poetics Today} 25:1. PDF.

Hayles, N. Katherine. (2008).
{\em Electronic Literature: new horizons for the literary}. Notre
Dame: University of Notre Dame. Print.

Holt, Jason and Tom Miller. (2010).
\quotation{Introducing the Google Command Line Tool}.
{\em Open Source at Google} blog. 18 June 2010. Web.
\letterless{}\useURL[9][http://google-opensource.blogspot.com/2010/06/introducing-google-command-line-tool.html][][http://google-opensource.blogspot.com/2010/06/introducing-google-command-line-tool.html]\from[9]\lettermore{}
(last accessed 18 June 2010).

Illich, Ivan and Barry Sanders. (1988).
{\em ABC: The Alphabetization of the Popular Mind}. San Fransisco:
North Point Press. Print.

Kay, Alan and Adele Goldberg. (1977).
\quotation{Personal Dynamic Media}. From the collection
{\em The New Media Reader}, edited by Noah Wardrip-Fruin and Nick
Montfort, 2003. Cambridge: MIT Press. PDF.
\letterless{}\useURL[10][http://www.newmediareader.com/book_samples/nmr-26-kay.pdf][][http://www.newmediareader.com/book\letterunderscore{}samples/nmr--26-kay.pdf]\from[10]\lettermore{}
(last accessed 15 June 2010).

{\em Kernel (computing)}. (2010). \quotation{Kernel (computing)}.
{\em Wikipedia}. Web.
\letterless{}\useURL[11][http://en.wikipedia.org/wiki/Kernel][][http://en.wikipedia.org/wiki/Kernel]\from[11]\lettermore{}
(last accessed 18 June 2010).

Kinross, Robin. (2004).
{\em Modern typography: an essay in critical history}. London:
Hyphen Press. Print.

Knuth, Donald. (1984). {\em The TeXBook}. Massachusetts:
Addison-Wesley. Print.

Knuth, Donald. (1999). {\em Digital Typography}. Stanford: CSLI.
Print.

Kroker, Arthur and Marilouise Kroker. (2010). ¨Code Drift¨.
{\em CTheory}. 14 April 2010. Web.
\letterless{}\useURL[12][http://ctheory.net/articles.aspx?id=633][][http://ctheory.net/articles.aspx?id=633]\from[12]\lettermore{}
(last accessed 19 June 2010).

Lovitt, Michael. (2010).
\quotation{Rubular: a Ruby regular expression editor}.
{\em rubular.com}. Web.
\letterless{}\useURL[13][http://rubular.com][][http://rubular.com]\from[13]\lettermore{}
(last accessed 28 July 2010).

**Langlois, Ganaele, Fenwick McKelvey, Greg Elmer, and Kenneth
Werbin. (2009). ¨Mapping Commercial Web 2.0 Spaces: Towards a New
Critical Ontogenesis¨. {\em Fibreculture} 14. Web.
\letterless{}\useURL[14][http://journal.fibreculture.org/issue14/issue14_langlois_et_al.html][][http://journal.fibreculture.org/issue14/issue14\letterunderscore{}langlois\letterunderscore{}et\letterunderscore{}al.html]\from[14]\lettermore{}

Manovich, Lev. (2001). {\em The Language of New Media}. MIT Press:
Cambridge. Print.

Manovich, Lev. (2008). {\em Software Takes Comman}. MS Doc
manuscript, published online 20 November 2008.
\letterless{}\useURL[15][http://softwarestudies.com/softbook/manovich_softbook_11_20_2008.doc][][http://softwarestudies.com/softbook/manovich\letterunderscore{}softbook\letterunderscore{}11\letterunderscore{}20\letterunderscore{}2008.doc]\from[15]\lettermore{}.
Web.

Marco, Meg. (2010).
\quotation{Privacy Change: Apple Knows Where Your Phone Is and Is Telling People}.
{\em The Consumerist}. 21 June 2010. Wev.
\letterless{}\useURL[16][http://consumerist.com/2010/06/privacy-change-apple-knows-your-phone-is-and-is-telling-people.html][][http://consumerist.com/2010/06/privacy-change-apple-knows-your-phone-is-and-is-telling-people.html]\from[16]\lettermore{}
(last accessed 22 June 2010).

Maurer, Luna and Edo Paulus, Jonathan Puckey, Roel Wouters. (2008).
\quotation{Conditional Design Manifesto}.
{\em conditionaldesign.org}. Web.
\letterless{}\useURL[17][http://www.conditionaldesign.org/manifesto/][][http://www.conditionaldesign.org/manifesto/]\from[17]\lettermore{}
(last accessed 17 June 2010).

McLuhan, Marshall. (1964). \quotation{Media Hot and Cold}.
{\em Understanding Media: The Extensions of Man}. Cambridge: MIT
Press, 1994. PDF.

OSP. (2010). \quotation{OSP DIN}. {\em Libre Graphics \#0!}.
Belgium: Drukkerij Bulckens nv. Print.

Pasquinelli, Matteo. (2008).
\quotation{The Ideology of Free Culture and the Grammar of Sabotage}.
{\em generation online}. PDF.
\letterless{}\useURL[18][http://www.generation-online.org/c/fc_rent4.pdf][][http://www.generation-online.org/c/fc\letterunderscore{}rent4.pdf]\from[18]\lettermore{}

{\em Regular expression}. (2010). \quotation{Regular expression}.
{\em Wikipedia}. Web.
\letterless{}\useURL[19][http://en.wikipedia.org/wiki/Regular_expression][][http://en.wikipedia.org/wiki/Regular\letterunderscore{}expression]\from[19]\lettermore{}
(last accessed 28 July 2010).

Rosenberg, Scott. (2008). {\em Dreaming in Code}. New York: Three
Rivers Press. Print.

Simondon, Gilbert. (2009).
\quotation{The Position of the Problem of Ontogenesis}. Gregor
Flanders, trans. {\em Parrhesia} 7. PDF.
\letterless{}\useURL[20][http://www.parrhesiajournal.org/parrhesia07/parrhesia07_simondon1.pdf][][http://www.parrhesiajournal.org/parrhesia07/parrhesia07\letterunderscore{}simondon1.pdf]\from[20]\lettermore{}
(last accessed 5 June 2010).

Snelting, Femke. (2009). \quotation{The Making of}.
{\em Tracks in electr(on)ic fields}. Brussels: Constant. Web.
\useURL[21][http://ospublish.constantvzw.org/wp-content/uploads/makingof.pdf][][http://ospublish.constantvzw.org/wp-content/uploads/makingof.pdf]\from[21]
(last accessed 5 June 2010).

Snelting, Femke and Pierre Huyghebaert. (2010). Personal interview,
with John Haltiwanger. Brussels, Belgium.

Stephenson, Neal. (1999). ¨In the Beginning was the Command Line¨.
Essay. Web.
\letterless{}\useURL[22][http://artlung.com/smorgasborg/C_R_Y_P_T_O_N_O_M_I_C_O_N.shtml][][http://artlung.com/smorgasborg/C\letterunderscore{}R\letterunderscore{}Y\letterunderscore{}P\letterunderscore{}T\letterunderscore{}O\letterunderscore{}N\letterunderscore{}O\letterunderscore{}M\letterunderscore{}I\letterunderscore{}C\letterunderscore{}O\letterunderscore{}N.shtml]\from[22]\lettermore{}
(last accessed 16 June 2010).

{\em Windows 1.0}. (2010). \quotation{Windows 1.0}.
{\em Wikipedia}. Web.
\letterless{}\useURL[23][http://en.wikipedia.org/wiki/Windows_1.0][][http://en.wikipedia.org/wiki/Windows\letterunderscore{}1.0]\from[23]\lettermore{}
(last accessed 19 June 2010).

{\em Windows XP}. (2010). \quotation{Windows XP}. {\em Wikipedia}.
Web.
\letterless{}\useURL[24][http://en.wikipedia.org/wiki/Windows_xp][][http://en.wikipedia.org/wiki/Windows\letterunderscore{}xp]\from[24]\lettermore{}
(las

% we should have an open works-cited going
\stopworkscited

\stoptext

