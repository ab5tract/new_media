%\enableregime[utf]  % use UTF-8

\setupcolors[state=start]
\setupinteraction[state=start, color=blue] % needed for hyperlinks

\usemodule[simplefonts]
\setmainfont[linlibertineo]
\setmonofont[inconsolata][11pt]

\setuppapersize[A4][A4]  % use letter paper
%\setuplayout[width=middle, backspace=1.5in, cutspace=1.5in,
%             height=middle, header=0.75in, footer=0.75in] % page layout
\setuppagenumbering[location={footer,right}]  % number pages
\setupbodyfont[12pt]  % 11pt font
\setupwhitespace[medium]  % inter-paragraph spacing

\setupindenting[medium]
\indenting[always]

\setuphead[section][style=\tfc]
\setuphead[subsection][style=\tfb]
\setuphead[subsubsection][style=\bf]

% define descr (for definition lists)
\definedescription[descr][
  headstyle=bold,style=normal,align=left,location=hanging,
  width=broad,margin=1cm]

% prevent orphaned list intros
\setupitemize[autointro]

% define defaults for bulleted lists 
\setupitemize[1][symbol=1][indentnext=no]
\setupitemize[2][symbol=2][indentnext=no]
\setupitemize[3][symbol=3][indentnext=no]
\setupitemize[4][symbol=4][indentnext=no]

\setupthinrules[width=15em]  % width of horizontal rules

% for block quotations
\unprotect

\startvariables all
blockquote: blockquote
\stopvariables

\definedelimitedtext
[\v!blockquote][\v!quotation]

\setupdelimitedtext
[\v!blockquote]
[\c!left=,
\c!right=,
before={\blank[medium]},
after={\blank[medium]},
]

% for bibliographic entries

% following hanging indent code (also in workscited) taken from 
%  http://www.ntg.nl/pipermail/ntg-context/2004/005280.html
% [NTG-context] Re: Again: "hanging" for a lot of paragraphs?
%  ~ Patrick Gundlach
\def\hangover{\hangafter=1\hangindent=0.5in}
\definestartstop[workscited][
  before={
    \page[no]
    \indenting[never]
    \startalignment[left]
    \subject{Bibliography}
    \stopalignment
    \setupwhitespace[medium]
    \bgroup\appendtoks\hangover\to\everypar
    },
  after={
    \egroup
    \indenting[yes]}]

\protect

\starttext

\section{Introduction}

Today's new media theory increasingly invokes {\em materiality} as
a significant, perhaps even {\em the} significant, mode of
investigating digital objects and the media through which they are
delivered. This thesis questions such a centrality of materiality
through a practice-based, process-oriented approach. {\em Process}
is proposed as the atomic unit of that which new media theory
investigates. This is true on a formal material level: applications
run as either as individual process or as assemblages of process
which are managed by an operating system and through which the
application's code is accomplishes all of its tasks, from memory
and access to algorithmic execution on the central processing unit.
A process-oriented approach will be shown to provide superior
methodologies for engaging with and understanding software than
material analysis alone provides. For instance, certain
problematics within Lev Manovich's concept of
\quote{media hybridity} will be resolved by a re-orientation
towards process (Manovich 2008). Process also allows a fresh
perspective for examining human-digital relations. Human processes
and digital processes are seen as inextricably intertwined, leaving
any discussion of digital process that excludes relevant dimensions
of human process necessarily unfinished.

=== FRESH ===

In this introduction, I will first briefly re-trace the vectors of
medium theory as they have developed since the introduction of
\quote{new media} as an academic institution. Such re-tracing
necessarily begins with McLuhan, as his theory is intrinsic to one
of the earliest theoretical frameworks of new media, the
{\em remediation} model of Jay David Bolter and Richard Grusin.
This framework was eventually superceded by a discipline-wide turn
towards investigations of materiality and medium specificity,
spearheaded by the work of N. Katherine Hayles. Finally, Lev
Manovich's manuscript {\em Software takes command} provides
concepts of {\em media hybridity} and {\em deep remixability}.
Throughout this swath of theory is woven an intrinsic focus on the
{\em material} modes of media. Media hybridity, for instance,
relies on a medium having a specific dimensionality that is
enlarged or otherwise augmented through hybridization with other
mediums. In a significant example, the dimension of typography
obtains velocity and physicality as it enters the 3D void of the
After Effects window. I pose the following question to this
explanation [WHUT WASS IT AGGIN\lettermore{}\lettermore{}??].
Rather than \quote{dimensionality,} I propose that we conceptualize
these hybridizations as generating new levels of metapotentiality.
This is a transposition of Gilbert Simondon's language of
ontogenesis into new media discourse. Simondon utilizes the ideas
of metastability and metapotential to describe active forms of
those concepts: a metastability is likened to a substrate in which
massive activity occurs. One example is a fluid suspension that
happens to contain ideal conditions for crystalization: the
metastability of the suspension drives its crystalization, and
crystalization is the actualization of the suspension's
metapotential. Yet crystalization is a singular process, a form of
individuation for which identical crystals are said not to exist in
nature. The metapotential of each substrate is fulfilled uniquely.
And in natural or otherwise unbounded systems, crystals are often
on-going processes of {\em individuation}---Simondon's term for the
movement of a metastability through the courses of its own
metapotential. This thesis proposes that grammars are a key factor
of enabling hybridization. Hybridization, in turn, expands and
extends the metapotentials of involved processes. Thus, grammars
are crucial mechanisms through which the metastability expands
along its metapotential.

The method proposed to demonstrate these points is two-fold. The
first is an analytic approach---the modes of operation of designers
themselves are examined. Starting from the proprietary Mac OS X
operating system, described here as a unique and powerful example
of {\em process hybridity}, we progress to a discussion of the
operations of designers as constrained by FLoSS
(Free/Libre/open/Source Software). The second aspect of the method
is a detailed interrogation of actual practice in the form of
{\em digital typesetting}. This topic was chosen for several
reasons. The first is a general lack of focus on the processes
behind typesetting among new media theory---while the surfaces of
text have been investigated in numerous ways (Bolter 2001; Fuller
2000), there has been a general lack of concern (or capacity)
regarding the underlying processes of text in the metamedium
(computers). This is especially evidenced as regards the
{\em command line interface}, a realm where text becomes kinetic.
Yet I found that very little theory has been written regarding the
command-line, despite its place as the historical interface (once
contemporary with batch punch cards) by which digital processes
were initiated. Far from being obsolete, both Microsoft and Apple
ship command line interfaces within their operating systems. In
Microsoft´s case, significant money has been spent developing a new
grammar and implementing new functionalities into their modern
command line implementation Powershell (as opposed to the grammar
and functionalities of DOS).

The second reason for choosing typesetting is the supposed lack of
media hybridity of typesetting---according to Manovich's
definitions of the terms, typesetting has failed to move beyond
\quote{multimedia} to a state of \quote{media hybridity} (this is
opposed to typography, which undoubtedly has) (2008: 86). Media
hybridity is Manovich´s formulation of the increasingly common
¨sharing of languages¨ between media. When media share language,
they develop new dimensions (2008: 86). Language, then,
demonstrates its capacity for modulation in a new context. While
the proposition that ¨language can add dimensions to things¨ may at
first consideration seem a bit too obvious for stating out loud,
the kinetic properties of language within the context of the
metamedium---that the code enabling the language sharing that
enables media hybridity is
{\em itself made of language and made executable by language}---seem
to beg for consideration. Whereas much of the new media discourse
relating to changes in media trends toward contemplating fast-paced
visual cultures such as video games and cinema, this thesis aims to
take the opportunity to contemplate the much slower-moving medium
of text. This contemplation of screenic text leads to questions
about the nature of media within a medium as well as to the
introduction of a conception of processual hybridity that both
underpins and exceeds the dynamics of media hybridity.

The third aspect is the allowance of a truly reflexive
investigation in which multiple processes of digital typesetting
are utilized to generate the thesis itself. This provides a means
to integrate the process-oriented perspective into a software study
of FLoSS typesetting software. Not only this, it provides a means
to attempt what could be considered a {\em refractional}
methodology. Inspired by Gilbert Simondon´s adoption of the
language of chemistry in the formulation of {\em transduction}
within his theory of ontogenesis, this thesis can be viewed as a
distinct crystallization process, the composition of a whole from
the process of that whole´s unfolding. The applicability of
Simondon´s ontogenesis to matters of generative design will be
interrogated in contrast to Jay David Bolter and David Grusin´s
remediation theory (Bolter and Grusin 1996; Bolter 2001).
Ontogenesis, albeit without Simondon, has already proven an
effective angle for approaching Web 2.0 platforms (Langlois,
McKelvey, Elmer, and Werbin 2009). Here the description of this
thesis´own workflow will demonstrate Simondon´s ontogenesis as
making unique contributions to the process-oriented perspective
which this thesis attempts to invoke and instantiate.

The fourth is the simple fact that screenic text has not been
interrogated on a {\em subtextual} level---surface analysis of text
(and hypertext) have driven the discourse of screenic text in new
media.

\section{Screens}

As digital typesetting provides the focus for the application of
the process-oriented perspective, the point of origin is
necessarily that of the screen. Information transmission is
increasingly screen-based, a fact that only intensifies with the
exponentializing ubiquity of mobile devices such as the iPhone. The
long-awaited advent of cheap \quotation{tablet} computers and
e-readers is also now at hand. These devices may all be seen as
mediums for {\em screenic processes} in that their entire
configuration and all of its computation exists to serve as the
basis for screenic interactions with {\em human processes}. These
phrasings introduce the perceptual angle attendent with this
thesis, namely the centering of {\em process} as the atomic unit of
what is discussed in new media theory. The term {\em screenic}
simply means \quote{screen-based,} or (perhaps)
\quote{screen-native.} It is analagous to \quote{printed.}

One way to define screens is in terms of their interactivity. Some
screens, such as television screens, offer very limited
interactivity: the choice of content. This choice itself can be
constrained by varying degrees, such as the number of available
channels and playback formats (VHS, DVD, Xvid, etc.), even to the
point of disappearing (in the case of many televisions that appear
in public spaces).
\footnote{Mobile devices are beginning to ship IR transeivers with full
hardware access through software. That is, the
{\em entire potential} of the IR spectrum is available to them.}
The medium of the remote control should not be underestimated in
its effects on human processes, to say nothing of the screens at
which they are aimed. Indeed, they drive the interactivity of the
video game consoles, an interactivity that clearly represents the
cultural cutting edge of what a television screen can offer.

The computer screen, on the other hand, is defined by its seemingly
limitless degree of interactivity. Remote controls can be run as
screenic processes and can not only change television
channels---processes on remote systems can be controlled with
similar ease. Indeed, the entire screenic composition of one
computer can be controlled over a network by a second computer
using included, or easy to obtain, applications. Furthermore, the
very interfaces to the screen (keyboards and mice) are examples of
remote controls in cases where the screen has not itself become its
own remote control (touch-screen devices). Typically the only
element of a computer screen that the user does not effectively
control are the structure and visual language of an operating
system's graphical user interface (GUI). Even this, however, is
generally accomplishable by a significantly informed user. In the
case of GNU/Linux the task is not only accomplishable: in the case
of a \quotation{from scratch} installation,
\footnote{Such as is demanded by no-frills distributions such as Gentoo and
ArchLinux, where manual installation and configuration of a GUI is
required for use.}
the user is literally forced to make a choice of GUI structure and
visual styling. Microsoft has generally shipped their operating
systems with multiple choices for widget
\footnote{A widget is the technical term for a GUI element. Scrollbars,
titlebars, menus, and close/minimize/maximize buttons are widgets
attached to most of the \quotation{windows} that appears on any
given GUI-driven computer.}
presentation, including re-mediations of widgets from previous
versions of Windows. Users also developed Apple, however, maintains
strict control of widget presentation, especially on their mobile
devices.

\subsection{Screens as material, screens as process}

Screens offer an ideal point of juxtaposition between the material
and processual frames. From a material view, the very formulation
of \quotation{screens} as {\em the} interface between humans and
computers is problematic: what of the interfaces that have been
developed to work around instances of blindness or other
[disabilities] that prohibit visually screenic interaction?

From a processual orientation, the question becomes: how do
interactions between humans and computers resolve themselves? The
answer returns in the form of the {\em available} remote controls
and the {\em available} response interfaces. The next step might be
to investigate the degree of variance between these availabilities,
and whether they problematize any umbrella-classification. While it
would be {\em insensible} to argue that material differences in
inputs and outputs can---or do---not lead to a huge amount of
variation between experiences within humans. Such variation is
likely to occur in differentials. That is to say, the spectrum of
possible feedback occurs at the level of the human
individual---one's experiences are functionally irrepresentable
without translation of some kind. [We can choose to call these
translations mediums, or we can choose to call these processes.]

At this point the question becomes, then, whether it is necessary
to instantiate these inherent divergences in every evocation of a
broad level discussion of input and output mechanisms or whether
the inherent, {\em core} similarity between them all remains that
in all instances they serve as {\em the point of contact} between
human and digital processes. Does it make a processual difference
if the output technology is a braille screen or an LCD screen? Only
inasmuch as to what degree the process being examined is unique to,
or highlights differences between, one or the other. From a
discursive level, {\em controls} and {\em screens} can capture the
essence of these dual \quotation{action spaces} that together form
the single point of contact between human and digital process.

Is it possible to remediate of the term screen into discussions of
previous mediums? For instance could one speak of the
\quotation{screen} of a newspaper or the \quotation{screen} of a
cave wall? What about the \quotation{screen} of a radio? From a
linguistic-conceptual perspective the final example certainly
pushes the limits. From a process perspective, though, the presence
of the radio/what it is playing/what listening choices are
available/how and to what degree does the hardware support
frequency tuning: these questions can all be conceived in terms of
\quote{control} and \quote{screen.} The sounds of a radio do emit,
after all, from the vibrations of a stretched membrane.

This thesis proposes a conceptual-linguistic shift in the
discussions of screens as the {\em site of discourse} through which
digital processes yield the results of their execution. Likewise,
the remote control, or simply {\em control}, is the site of
discourse through which which human processes instigate and extend
into the digital. There is no removing or reducing of this dyadic
assemblage---even when the control and the screen are literally
fused (as in most contemporary smartphones) the distinction between
{\em control} and {\em screen} holds on both a conceptual and
material level. Conceptually, human process still extends through
the control into digital process, which still produces feedback
through the screen. Materially, the screen is a Liquid Crystal
Display driven by a graphics card that interfaces with coded
drivers and display subsystems in the device's operating system.
The control, on the other hand, is the glass suspended over the LCD
which, through one or more of the multitude of available technical
solutions for the process, reads point(s) of contact, pressure, and
vectors (velocity and direction) of movement.

\section{From Screens to Text}

To discuss computer screens one must necessarily engage with the
concept of {\em interface}, a topic that rightfully occupies a
great deal of current new media discourse. Interface, then,
represents one point of departure from our origin. While interfaces
often utilize many visual metaphors (most of them inherited from
the work done developing the first GUI at Xerox's Palo Alto
Advanced Research Lab (PARC) in the 1970s), there are yet few
computer interfaces that do not rely on text as their dominant
mechanism for organizing and presenting a program's internal
capabilities to a user. (Mobile screens, on the other hand,
increasingly display developing trends of icon-only design, though
the web browser remains a popular application). Despite the success
of the GUI over the text-only command-line interface (CLI), text
remains central to contemporary experiences of computer screens.

The command line is seen as a space of contestation for traditional
modes of media analysis. Remediation, for instance, will be
demonstrated as inappropriate for discussing the CLI. As Google has
just recently released a command line interface for interacting
with Google services, I believe a discussion of the command line is
essential for new media (Holt and Miller 2010).

(Unfortunate to note, this historiographic aspect is still
{\bf \quote{to-do}}:

The centrality of text to the experience of computer screens
represents the main avenue by which we proceed from the origin,
constituting a trunk from which many additional concerns fork away
and then face examination. The arguments of the paper are augmented
by the inclusion of a historiography of digital typesetting.
Engaging critically with the history of {\em software itself} is
considered a requisite for responsible software studies: a full
range of influences (economic, cultural, technological) should be
considered in the re-telling of a given processual unfolding. In
this aspect of focus, it extends Lev Manovich's admirable
positioning of history as central to a software study by broadening
the scope of historical considerations.
\footnote{{\bf Note:} This work largely remains unfinished in this draft, as
it became apparent that I needed to work back through more
discussions of basic infrastructural elements such as operating
systems in order to fully describe the assemblage of process upon
which computer-based design is situated.)}
Inspiring this enagement is the work of Robin Kinross, whose
{\em Modern typography: an essay in critical history} is one of but
a few texts covering a history of typography to adequately engage
with the influence of factors outside of that field on the field
itself (Kinross 2004). By integrating a critical history of digital
typesetting with a process perspective, an equilibrium between
human and digital processes will be illustrated.

\subsection{Recognizing the Ontogenesis in Generativity}

In his text {\em The Position of the Problem of Ontogenesis},
Simondon writes,

By transduction we mean an operation---physical, biological,
mental, social---by which an activity propagates itself from one
element to the next, within a given domain, and founds this
propagation on a structuration of the domain that is realized from
place to place: each area of the constituted structure serves as
the principle and the model for the next area, as a primer for its
constitution, to the extent that the modification expands
progresively at the same time as the structuring operation.
(Simondon 2009: 11).

Note the distinct lack of \quote{computational} in Simondon's list
of operations. Written prior to the advent of Manovich's
formulation of the age of cultural computing, this absence might
simply be read as a matter of temporal context. Nevertheless,
Simondon's solution to the ontogenesis problematic provides a
framework for describing digital processes of a generative nature.

This leads to another important element of this thesis, one that
runs throughout the entirety of itself---the underlying processes
of presentation required to \quote{typeset} the text itself.
Through the utilization of FLoSS software, multiple output formats
will be not only be investigated but also materially instantiated
through a designed mechanism of process---a
{\em processual hybridity}. These output formats represent two of
the top formats currently used to manage and display texts
digitally: HTML and PDF.

The process(es) of their generation offers an attempt at mapping
Gilbert Simondon's language of ontogenesis onto file format
translation or, to begin the project immediately,
{\em individuation}. Coupled with Simondon's individuation is this
concept of {\em transduction}. Repurposed from the language of
chemistry, Simondon's metaphorically images transduction with the
example of a substrate---swelling with {\em metapotential}---that
crystallizes. The final formation is the substrate fulfilling this
metapotential, a fulfillment that arises only through an
unpredictable unfolding involving emergent factors. (The language
of chemistry was likewise appropriated for the term
\quote{interface} (Cramer and Fuller 2008: 149)).

Through this mapping I hope to provide a convincing argument for
shared properties between what I am calling process and
individuation, and between transduction and what I am calling
instantiation.

This relates with the increasingly generative nature of
contemporary design. All of which are generated from a plain-text
file whose syntax conforms to a format standard called
\quote{markdown.} The polycephalous nature of {\em the text itself}
thus demands further branching into a discussion of formats. What
are the attributes of the class of process to which formats belong?
Formats are seen as stable, yet they move like glass (or glaciar)
in the nano-magnitudes of the digital. Formats provide another
point of contrast between process and material perceptual
orientations.

The discussion of generativity provides further means to
demonstrate the equilibrium of human and digital processes.
Analyzed materially, these processes are chunks of code
electronically lifted from hard drive platters, loaded into system
memory, and then executed via the assemblage of chips on the
computer's motherboard by way of instructions from the operating
system currently residing as a mass of memory heaps in RAM chips.
Analyzed {\em processually}, however, these digital processes are
properly seen as deriving from interactions with human beings. That
is to say, digital and human processes are intimately intertwined,
from the design of their physical landscape of execution
(microcircuitry) to the instructions derived from the user. From a
process angle the computer becomes something of an external nervous
system, extending and modifying the realm of human potentiality
even as it surpasses the capacity of a single mind to functionally
comprehend the entirety of its workings.
\footnote{The chips produced by Intel, for example, are too complex for any
single person to ever hope to entirely understand.}

\subsubsection{Print is static, code is process}

The flat/deep distinction proposed by Hayles is, by its
formualation, material. Problematizing this material focus is the
interwoven history of text and code: the lens of typesetting allows
us to focus on a unique intersection of the two. As the
historiographic case will demonstrate, typesetting is a
{\em non-reducible} process (NP-Complete). This non-reducibility of
typesetting reflects the non-reducibility of computational
processing of language, as well as the non-reducibility of
language, as signifier, into that which is actually signified. This
\quotation{turtles all the way down} scenario has intriguing
implications from a process perspective as we investigate the
methods that have been developed in order to work around this
non-reducibility.

When Hayles states that
\quotation{materiality thus cannot be specified in advance; rather it occupies a borderland---or better, performs as connective tissue,}
she is provisionally correct (Hayles 2004: 72). However, this
metaphor-ization of process is exemplary of new media practices:
reference the complex with an abstract metaphor, obscuring complex
and important dynamics with a metaphor. The metaphor works, to be
sure. One could even consider it an ideal formulation. At issue is
the fact that this borderland is not discussed in a technically
correct manner.

\section{Remote Controls}

I think it may be reasonable to take the remote control and use it
to create a metaphor for all human-computer interaction.

Every digital process has, at its origin, a human. The rate of
computation has increased the impact of human-digital processes in
that the results deliver their results faster. The results will
either match the intentions of the originating human process, or
they will not. In the second case we can find the first evidence of
the effects of digital process on human process:
{\em the code behind the digital process will be re-arranged in an attempt to deliver an output that satisfies the intention of the human processes.}
Whether this modulation of the executed code is through
sliders/input boxes/etc within a GUI interface or through direct
reworking of the source code itself, the effect is the same: the
code executed has been re-configured according to the goal of human
process. The result(s) of the digital process, experienced through
a screen, can match, exceed, or fail this goal. In turn, human
process is effected and the next move is made according to new
goals or revised digital processes.

\stoptext

