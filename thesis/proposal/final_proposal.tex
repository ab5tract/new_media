% Links
\setupcolors[state=start]
\setupinteraction[state=start,color=darkblue]

% ..
\setuppagenumbering[location={footer,left,margin}]  

% paper, layout, etc.
\setuppapersize[A4]
%\setuplayout[width=6.5in,height=10.5in,topspace=0.5in,backspace=1in,
%  header=0.5in,footer=0.5in]
% double space
%\setupinterlinespace[line=5.6ex]
% 1/2 inch indents
\setupindenting[medium]
\indenting[always]
% CSL is a new form of handling bibliographic data and citations. It also apparently   http://xbiblio.sourceforge.net/csl/
% jagged-right (aligned to the left)
\setupalign[right]
% font
%\setupbodyfont[rm,12pt]
\usemodule[simplefonts]
\setmainfont[Liberation-Serif]
\setmonofont[inconsolata,11pt]

% for long quotes
\definestartstop[longquote][
  before={\indenting[never]
    \setupnarrower[left=0.5in,right=0.5in]
    \startnarrower[2*left,right]},
  after={\stopnarrower
    \indenting[yes]}]

% for heading and header
\def\MLA#1[#2][#3][#4][#5][#6][#7]%
	{\setuppagenumbering[left=#3 ,location={header,right}]
	\indenting[never]
	#2 #3\par#4\par#5\par#6\par\startalignment[middle]#7\stopalignment
	\indenting[yes]}
	
% following hanging indent code (also in workscited) taken from 
%  http://www.ntg.nl/pipermail/ntg-context/2004/005280.html
% [NTG-context] Re: Again: "hanging" for a lot of paragraphs?
%  ~ Patrick Gundlach
\def\hangover{\hangafter=1\hangindent=0.5in}
\definestartstop[workscited][
  before={
    \page[no]
    \indenting[never]
    \startalignment[left]
    \subject{Bibliography}
    \stopalignment
    \bgroup\appendtoks\hangover\to\everypar
    },
  after={
    \egroup
    \indenting[yes]}]


\starttext      

{\tfd Generative Re-mediation}

{\tfb An investigation of cross-media publishing in free and open source software}

\starttabulate[|l|l|]
\NC Student:
\NC John Haltiwanger
\NR
\NC Paper:
\NC Final Thesis Proposal
\NR
\NC Supervisor:
\NC Richard Rogers
\NR
\NC Date:
\NC 6 March 2010
\NR 
\stoptabulate  




\subject{Introduction}

The very term \quote{media} comes from the discipline of
advertising, which in the 1940s began to develop a vocabulary to
discuss the emerging issue of delivering messages across various
distribution channels. While this initial impulse did not
necessarily involve an algorithmic capacity to target multiple
mediums using a single input, the advent of networked computers has
seen just such a desire arise. \quote{Traditional} design tools
such as the Adobe Creative Suite are being replaced in top design
firms with workflows that dynamically generate documents. Rather
than constructing everything by hand in what the What You See Is
What You Get (WYSIWYG) interfaces of those programs, designers are
now using software {\em wrappers} and scripts that facilitate
translations of one format into another. Tied together in a
deliberate workflow, the tools receive relatively simple source
documents as input and produce multiple outputs given the
parameters the designer has implemented into their workflow. This
process is called {\em cross-media publishing}.

Free, Libre, and Open Source Software (FLOSS) often plays an
important role in such workflows. FLOSS programming languages,
wrappers, frameworks, toolkits, and applications are increasingly
incorporated into a design workflow. The success of the generative
art toolkit Processing is exemplary in this regard
\footnote{Processing.
\useURL[2][http://processing.org/][][http://processing.org/]\from[2].}.
The WYSIWYG application Scribus, which aims to provide an interface
with capabilities analagous to Adobe InDesign, provides
Python-based scripting functionality to allow a hybridized
approach. Indeed, Adobe has developed an XML-based input format for
InDesign that allows generative production of InDesign documents.
Cross-media publishing and generative techniques are not only
increasingly successful in the design world, they are also closely
related. This thesis aims to investigate this relationship through
a specific mode of cross-media publishing. Using a specific wrapper
called Pandoc
\footnote{Pandoc.
\useURL[1][http://johnmacfarlane.net/pandoc/][][http://johnmacfarlane.net/pandoc/]\from[1]},
the thesis will be simultaneously typeset in three output formats:
HTML, OpenDocument (ODT), and ConTeXt. Each of these formats
represent a distinct approach to the issue of typesetting---these
approaches are worthwhile to investigate, as they represent the
three approaches to typesetting currently found on the computer:
semantic markup, formal markup, and WYSIWYG. The Holy Grail of
cross-media publishing, according to Florian Cramer, is
\quotation{one system that serves as the universal document source.}
\footnote{From a personal interview.}

\subject{Remediation}

The theoretical framework to be deployed in this investigation of
generative, cross-media typesetting is Jay David Bolter and Richard
Grusin's concept of remediation (Bolter \& Grusin 1999). As defined
by the authors, remediation is
\quotation{the representation of one medium an another} (45).
Bolter and Grusin identify two distinct approaches for messages in
a given medium. In the first, immediacy, the message is tailored in
such a way as to, ideally, make the audience forget the presence of
the medium. In the second, hypermediacy, the opposite is sought:
unique aspects of the medium are highlighted by the message.
Remediation, then, involves an interplay of these two approaches.
That the WYSIWYG style developed for Word (and wholeheartedly
adopted by OpenOffice.org) takes quite seriously the goal of
mimicking a typewriter is then an example of immediacy. Even the
approach of representing a page of paper on a screen is an example
of remediation. Since HTML and ConTeXt can both be used to create
not only remediated paper documents but also \quote{screen-native}
pages as well, the theoretical framework of remediation, immediacy,
and hypermediacy provides a compelling means for developing this
thesis.

\subject{Actor-Network Theory}

At the risk of mis-applying Bruno Latour's Actor-Network Theory
(ANT), this thesis will incorporate the language and conceptual
tools presented in Latour's {\em Reassembling the Social} (Latour
2005) in order to properly position the various applications,
frameworks, and output formats that will collectively generate and
encompass the final output of the thesis. Not only is \type{pandoc}
a mediator in this thesis, the output formats themselves are as
well, as their functionalities, peculiarities, and limitations will
have an effect on the content of the thesis as well as its final
presentation. The programs and scripts that come into play can also
be seen as having mediating roles. I am also an obvious actor in
this network, but in more ways than simply writing the thesis---for
example, questions I ask to the ConTeXt mailing list are
translations
\footnote{In Latour's words, though I may adopt the less confusing
\quote{modulation} as well as \quote{assemblage} over
\quote{network.}}
between actors. The simple production of this thesis, bridging as
it does both free software and generative design, could be thought
of as a point of contention through which once unconnected actors
are brought into contact with one another. Acknowledging this fact
up front and directly through an ANT framing of the project can
only strengthen its potential to one day become a mediating text
itself. Certainly it seems appropriate to begin to situate
softwares as important actors in their own right, especially in
this context where the very materiality of the thesis (let alone
its content) is dependent on the translations produced by these
actors.

\subject{Questions}

\startitemize
\item
  What is the current state of generative design? What drives its
  adoption by designers? What role does FLOSS play?
\item
  What are the individial merits of each of the three open source
  output formats? How do they contrast in relation with each other?
  What are the output formats unique capacities? The figurations of
  their capacities emerging from this document are to be discovered
  in the process of attempting to perform relative tasks between
  them.
\item
  What capacity do the output formats have for addressing issues of
  remediation? That is, does a given output format allow for
  targetting either immediacy, hypermediacy, or a hybridization? Or
  is only one or the other possible in that format?
\item
  What limitations are present that prevent accomplishing certain
  typesetting goals in a given format? What limitations in the
  software tools used will affect and limit the capabilities of this
  thesis?
\item
  Is it possible, through this approach, to generate aesthetically
  pleasing typesetting in any of the three formats? What advantages
  does a \quote{single source input format} have?
\stopitemize

\subject{Approach}

{\em What follows is an example of how the investigation the inclusion of hypertext in a ConTeXt document can provide a springboard for investigating material differences in the typesetting options.}

For instance \useURL[3][http://wikileaks.org/][][this url]\from[3]
was input according to the syntax
\useURL[4][http://daringfireball.net/projects/markdown/syntax][][found here]\from[4].
In the Markdown format this code appears as

\starttyping
For instance [this url](http://wikileaks.org/) was input according to the syntax 
[found here](http://daringfireball.net/projects/markdown/syntax).
\stoptyping

When the text
\type{pandoc -t context -o thesis.tex thesis.markdown} is input
into a shell prompt, the resulting output appears as follows

\starttyping
For instance \useURL[1][http://wikileaks.org/][][this url]\from[1] was input according to the syntax 
\useURL[2][http://daringfireball.net/projects/markdown/syntax][][found here]\from[2].
\stoptyping

However, that in itself is not enough for \type{context thesis} to
generate a valid PDF with hyperlinks. For three additional lines
are required.

\starttyping
\setupinteraction[state=start,color=darkblue] 
\starttext 
For instance \useURL[1][http://wikileaks.org/][][this url]\from[1] 
was input according to the syntax 
\useURL[2][http://daringfireball.net/projects/markdown/syntax][][found here]\from[2].

\stoptext
\stoptyping

The differences in the syntax are clearly explained by their status
as different markup languages. The structural differences, however,
such as the indication of a \quote{start} and a \quote{stop} to
\quote{text} as well as a slightly cryptic \quote{setup} for
\quote{interaction} reflect that TeX is a
{\em batch text processor} and the way that ConTeXt structures its
TeX macros, respectively. In other words, that these additional
commands are required reflects architectural differences that in
turn reflect the goals around which the text processing software
was designed. This is obvious when one considers that the output of
\type{pandoc -t html -o thesis.html thesis.markdown} not only shows
HTML's own structural choices (including it's much less awkward
hyperlink syntax), it will likely also render in a browser without
the addition of HTML's version of \quote{start} and \quote{stop}
(\letterless{}HTML\lettermore{} and
\letterless{}/HTML\lettermore{}) due to architectural decisions
relating to backwards compatability and leniancy belonging to the
browser's design.

Additional work would be required to achieve a decent looking
typesetting of the source code blocks above if the Pandoc utility
did not already do some primitive linebreaking of its own. ConTeXt
has trouble with handling long unbroken strings such as URLs and
even text commands, at least in the relatively primitive version of
a source code block that is being used here (\type{\starttyping}
and \type{\stoptyping}) as seen in the second code block. Meanwhile
the same code blocks look acceptable in ODT and HTML by default,
while the ConTeXt output fails as it to leaves off portions of
text. What procedures must be followed to achieve an acceptable
output in ConTeXt? Are they worth the effort? While the ODT output
is passable, to what extent can it be improved? HTML is only some
lines of CSS away from outputing the complete text of the code
blocks with the nicety of an boxed outline to offset it from the
general flow.

\subject{Expected Findings}

\startitemize[n][stopper=.]
\item
  There will be significant limitations in each format. ODT will be
  frustratingly manual in its requirements for effecting the desired
  typesetting. HTML will likely fall short in line-breaking and print
  quality. ConTeXt will likely prove highly temperamental and
  difficult to utilize to its full extent.
\item
  ConTeXt will be the most suitable for straddling \quote{immediacy}
  and \quote{hypermediacy,} as it allows for targetting both the
  printed page and the computer screen. It's superior typesetting
  algorithms will give it distinct aesthetic advantages, but in the
  end these will be weighed against the effort expended to achieve
  the desired result.
\item
  Generative design workflows will generally involve a majority of
  FLOSS software.
\item
  Typographic programming is not a well-documented field. A
  theoretical-technical of three FLOSS formats in typographic terms
  will provide a distinct contribution to both traditions.
\item
  A \quote{single source input format} that is a) plaintext and b)
  simple and readable is an integral requirement for the
  collaborative writing of texts.
\stopitemize

\startworkscited

Bolter, Jay David and Richard Grusin. {\em Remediation}. Cambridge:
MIT Press. 1999. Print.

Latour, Bruno. {\em Reassembling the social}. Oxford: Oxford
University Press, 2005. Print.

\stopworkscited

\stoptext