\usemodule[simplefonts]
\setmainfont[Liberation-Serif]

\setuppapersize[A4]

\setupwhitespace[medium]

\setuppagenumbering[location={footer,left,margin}]

\setupcolors[state=start]
\setupinteraction[state=start,color=darkblue]

{\tfd An Introductory Note on the Path to a Final Proposal}

The reason what I propose hasn't already happened is because the source has only
just now evolved to the point that can carry it.

\subject{The Holy Grail}

\startitemize
\item
  What is the Holy Grail?
\stopitemize

\subject{Self Injection}

This writer prefers a flexibility in narrative styles. That is, I
would reserve the right to shift perspectives in order to further
the cause of weaving a readable whole, if that right were granted
me. I've found that a text lacking its author is impossible for me
to write well.

But first, about where I come from:

I am here in Amsterdam as a result of a single blog post, found
through the serendipitous contermporality ({\em synchronicity})
afforded by logging into my del.icio.us account to find an
algorithmically suggested page titled
\quote{Review: Software Studies a Lexicon Edited by Matthew Fuller.}
Little did I know I was diving down the rabbit hole. Hosted on a
Netherlands domain, Anne Helmond's review inspired me to inquire
into programs offering opportunities in this new field in that
country. A theoretical approach to understanding and describing
software is a dream field for me---I've always lurked in the
background as a user, and devotee, of FLOSS. This was a clear way
to contribute.
\footnote{And, perhaps, an avenue to articulate the need, as I see it, for a
critical, non-programming point of participation in FLOSS projects.
While there are clear examples of proactive design in open source,
the fact that a vision generally needs to be implemented
{\em a priori} in order for it to exist at all seems like a failing
to me.}

I also expected that it would provide me the language, and
attendant concepts embedded therein, to articulate an on-going
collaborative project that could easily have been described as
\quote{New Media Art,} if we had known to classify it that way. Not
that we could ever fully describe it any, including this,
point---its core focus was always to interrogate the negative space
between a word and the concept it describes. Its plan is to One of
its core adaptations is a total absence of name attribution in the
outward facing interface. How do we refer to concepts when we
cannot simply append an '-ian' to a name and discuss it in that
fashion?
\footnote{A question that bears considerably on the readability of new media,
or so it seems to me.}

This, though, is not the project I'm proposing. The work I intend
to do is based, at a foundational level, with an erstwhile dormant
allergy to ill-formatted text recently activated by investigations
into digital typesetting with TeX. This became coupled to the my
original impulses at developing collaborative frameworks when Geert
Lovink opened the idea of a new \quote{post-journal} unaffiliated
with any academic institution and seeking a new means of vetting
beyond traditional peer-review processes. The platform to deliver
this, then, to me, is the Holy Grail. By being open-source, it is
infinitely embeddable once it is properly accomplished. The
platform that can power such a journal can power a collaborative
classroom, an open source laboratory, even the project I briefly
mentioned before.

Yet that is also not what I seek to accomplish here, especially
software wise. I cannot design a working journal while carrying
through a legitimate theoretical investigation/justification for
the effort. Instead I will be working on a prototype, itself
embeddable, that can provide real-time WYSIWYM editing of a
document that will be exportable into HTML, OpenDocument Format,
and PDF via ConTeXt. The edits to the document will be tracked via
the git version control system. But it will very much resemble a
stand-alone app, with the difference that you run it locally as a
web server and access it via your browser.
\footnote{Once embedded in a larger context, the system could easily
accomodate the kind of platform Anne Helmond suggests as a solution
to the mutability that the {\em Software studies} lexicon implies.

\subject{Whither Theory?}

Where there is a Holy Grail, there are paths littered with
artifacts left by those who seek it. Donald Knuth, originator of
TeX, famously thought that that project would take a summer of
thesis work by a masters student---32 years later, TeX is still in
development as issues of right-to-left formatting and support for
contemporary font features remain just two outstanding issues.

To that end, the theoretical portion of this project can be
described in three threads:

\startitemize[n][stopper=.]
\item
  Establish the various options for typesetting text with FLOSS
  {\em based on a plaintext workflow} (for this project: HTML, ODT,
  and ConTeXt). Simultaneously typeset the thesis in all three output
  formats, exploring the relative strengths and weaknesses of each.
  Using standards established over the history of typography, judge
  them on their materiality as typeset documents. Code cannot be
  separated from politics, and the importance of FLOSS to
  realizing/perpetuating Enzensberger's \quote{emancipatory media}
  must also be discussed.
\item
  Establish the importance of plaintext for collaboration by
  explaining binary file formats, markup syntax, version control, and
  translation utilties. All of these aspects would seem to have
  theoretically heavy implications, not least of which is the
  materiality of documents with multiple authors
  \footnote{It is to be acknowledged that \quote{author} is a problematic word
embodying a concept of creative development that was pushed heavily
by publishing industries in order establish literary works as the
product of a single mind.}
  and multiple output formats. As far as I know, Hayles has never
  dived into a technical discussion of typesetting,
  i.e.~line-breaking algorithms, markup standards, etc. To some
  extent the mystical aspects/history of text and codes must also be
  incorporated here, as well as the vision of the collaboritive
  platform to which the vector of this project eventually aims.
\item
  Establish typesetting for the screen as a significant issue in
  typsetting that is not addressed by ODT and not adequate (in a
  typographic sense) in the case of HTML. Only ConTeXt can take one
  input and, utilizing a project workflow feature called
  \quote{environments,} produced a document meant for the page and a
  document meant for the screen. The distinctions between the two
  will be explored as the thesis targets both mediums.
\stopitemize

\stoptext
