%\enableregime[utf]  % use UTF-8

\setupcolors[state=start]
\setupinteraction[state=start, color=blue] % needed for hyperlinks

\usemodule[simplefonts]
\setmainfont[Liberation-Serif]
\setmonofont[inconsolata]

\setuppapersize[A4][A4]  % use letter paper
%\setuplayout[width=middle, backspace=1.5in, cutspace=1.5in,
%             height=middle, header=0.75in, footer=0.75in] % page layout
\setuppagenumbering[location={footer,right}]  % number pages
\setupbodyfont[12pt]  % 11pt font
\setupwhitespace[small]  % inter-paragraph spacing

\setupindenting[medium]
\indenting[always]

\setuphead[section][style=\tfc]
\setuphead[subsection][style=\tfb]
\setuphead[subsubsection][style=\bf]

% define descr (for definition lists)
\definedescription[descr][
  headstyle=bold,style=normal,align=left,location=hanging,
  width=broad,margin=1cm]

% prevent orphaned list intros
\setupitemize[autointro]

% define defaults for bulleted lists 
\setupitemize[1][symbol=1][indentnext=no]
\setupitemize[2][symbol=2][indentnext=no]
\setupitemize[3][symbol=3][indentnext=no]
\setupitemize[4][symbol=4][indentnext=no]

\setupthinrules[width=15em]  % width of horizontal rules

% for block quotations
\unprotect

\startvariables all
blockquote: blockquote
\stopvariables

\definedelimitedtext
[\v!blockquote][\v!quotation]

\setupdelimitedtext
[\v!blockquote]
[\c!left=,
\c!right=,
before={\blank[medium]},
after={\blank[medium]},
]

\protect

\starttext

\startitemize[n][stopper=.,width=2.0em]
\item
  Robin Kinross posits that "if modernity was imlicit in printing, it
  was not fully or immediately realized by
  Gutenberg\quote{sinvention." The period in which printing begins to document its own process is the point that Kinross identifies as when typography becomes modern. If one were to venture that computer typography contains implicit elements of a 'hyper-modernity}
  (distributed, networked, screenic, fluid, tranlatable), what might
  be the development at which that hyper-modernity is achieved?
  
  \startitemize
  \item
    The first step of the stage was the separation of the
    \quote{printer} from the \quote{typographer.} Is there a chance
    that computer typography will introduce a third category, the
    \quote{coder}? Is it possible to separate the design of a TeX
    document from the coding and imlementation?
  \stopitemize
\item
  To what extent does OSP utilize a \quote{generative} workflow?
  Which utilities/programs do you use? Do you use an input format and
  a \quote{wrapper}, or does the writing process intermingle with the
  typesetting process (occurring in the same file)?
\item
  Florian Cramer identifiesa \quotation{feature} of computer
  typography: that of the \quote{showstopper.} What showstoppers have
  you encountered in regards to ConTeXt? How do they differ from
  showstoppers in, say, Scribus?
\item
  Cramer compares TeX to a player piano, whereas WYSWYG is akin to a
  piano under the fingertips of a pianist. Can you compare and
  contrast the different approaches and qualities that come with
  working in the two styles? Do certain projects fit certain
  workflows better?
\item
  Cramer also identifies the \quotation{Holy Grail} of cross-media
  publishing as a single system that serves as the universal document
  source. How important is cross-media publishing to OSP's goals? Do
  you think it is possible to reach the Holy Grail without complexity
  making it unworkable?
\item
  How does OSP view differences, if any, in typesetting for print
  versus typesetting for screen? Have you produced any documents
  epecially designed for screenic consumption (a la the hyperlinked
  ConTeXt manuals)? Specific design considerations? Font choices
  (sans serif on screen, serif on print)?
\item
  What opinions do you have on open source fonts? Any choice fonts to
  recommend?
\item
  What backgrounds do OSP contributers come from? Generally from
  design? or more from computers?
\item
  How did you and Pierre come to use FLOSS for publishing? Was it an
  economic choice, a freedom choice, or something else altogether?
\item
  How influential is the {\em Conditional Design Manifesto} on you as
  designers? Does it make sense to conceive of typographic outputs as
  processes rather than products? Certainly the very nature of open
  source is process rather than product.
\item
  Do you make available the source to your publications? Why or why
  not?
\item
  Do you see FLOSS expanding within the publishing world? What
  barriers are presently obstructing its adoption?
\item
  What is your dream addition to the FLOSS publishing ecosystem
  (could be whole new software or simply new features)?
\stopitemize

\stoptext

