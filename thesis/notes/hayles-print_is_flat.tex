%\enableregime[utf]  % use UTF-8

\setupcolors[state=start]
\setupinteraction[state=start, color=blue] % needed for hyperlinks

\usemodule[simplefonts]
\setmainfont[Liberation-Serif]
\setmonofont[inconsolata]

\setuppapersize[A4][A4]  % use letter paper
%\setuplayout[width=middle, backspace=1.5in, cutspace=1.5in,
%             height=middle, header=0.75in, footer=0.75in] % page layout
\setuppagenumbering[location={footer,right}]  % number pages
\setupbodyfont[12pt]  % 11pt font
\setupwhitespace[small]  % inter-paragraph spacing

\setupindenting[medium]
\indenting[always]

\setuphead[section][style=\tfc]
\setuphead[subsection][style=\tfb]
\setuphead[subsubsection][style=\bf]

% define descr (for definition lists)
\definedescription[descr][
  headstyle=bold,style=normal,align=left,location=hanging,
  width=broad,margin=1cm]

% prevent orphaned list intros
\setupitemize[autointro]

% define defaults for bulleted lists 
\setupitemize[1][symbol=1][indentnext=no]
\setupitemize[2][symbol=2][indentnext=no]
\setupitemize[3][symbol=3][indentnext=no]
\setupitemize[4][symbol=4][indentnext=no]

\setupthinrules[width=15em]  % width of horizontal rules

% for block quotations
\unprotect

\startvariables all
blockquote: blockquote
\stopvariables

\definedelimitedtext
[\v!blockquote][\v!quotation]

\setupdelimitedtext
[\v!blockquote]
[\c!left=,
\c!right=,
before={\blank[medium]},
after={\blank[medium]},
]

\protect

\starttext

\subject{Hayles, N. Katherine}

\subsubject{\quotation{Print is Flat, Code is Deep}}

\startitemize
\item
  \quotation{Materiality is reconceptualized as te interplay between a text's physical characteristics and its signifying strategies, a move that entwines instantiation and signification at the outset.}
  (67)
\item
  texts as {\bf embodied entities}
\item
  \quotation{It [the reconceptualization] makes materiality an emergent property, so that it cannot be specified in advance, as if it were a pre-given entity.}
  (67)
\item
  \quotation{..MSA insist that \quote{texts} must always be embodied to exist in the world. The materiality of those embodiments interacts dynamically with linguistic, rhetorical, and literary practices to create the effects we call literature.}
  (70)
\item
  \quotation{With significant exceptions, print literature was widely regarded as not having a body, only a speaking mind.}
  (70)
\item
  \quotation{It is crucially important, however, to recognize that the computer can simulate so successfully only because it differs profoundly from print in its physical properties and dynamic processes.}
  (71)
\item
  \quotation{In emphasizing materiality, I do not mean to imply that all aspects of a medium's apparatus will be equally important.}
  (71)
\item
  \quotation{Materiality always matters in some sense, but it matters most to humanists and artists when considered in relation to the prctices it embodies and enacts.}
  (72)
\item
  \quotation{Materiality thus cannot be specified in advance; rather, it occupies a borderland---or better, performs as connective tissu---joining the physical and mental, the artifact and the user.}
  (72)
\item
  \quotation{..the rich connecions between its physical properties and the processes that constitute it as something to be read make up together that elusive object we call a \quote{text}}
  (72)
\item
  \quotation{.. hypertext has at a minimum the following characteristics: multiple reading paths; some kind of linking mechanism; and chucnked text (that is, text that can be treated as discrete units and linked to one another in various arrangements).}
  (72)
\stopitemize

\subsubsubject{Point One: Electronic Hypertexts are Dynamic Images}

\startitemize
\item
  \quotation{In the computer, the signifier exists not as a durably inscribed flat mark but as a screenic image produced by layers of code precisely correlated through correspondence rules, from the electronic polarities that correlate with the bit stream to the bits that correlate with binary numbers to the numbers that correlate with higher-level statements, such as commands, and s on.}
  (74)
\item
  \quotation{This aspect [the constant refreshing of the screen] of electronic hypertext can be mobilized through such innovations as dynamic typography, where words function as both verbal signifiers and visual images whose kinetic qualities also convey meaning.}
  (74)
\item
  \quotation{Whereas all the words and images in the print text are immediately accessible to view, the linked words in Knobel's poem become visible to the user only when they appear through the cursor's action.}
  (75)
\stopitemize

\subsubsubject{Point Two: Electronic Hypertexts Include Both Analogue Resemblance and Digital Coding}

\startitemize
\item
  \quotation{Thus digital computers have an Oreo cookie-like structure with an analogue bottom, a frothy digital middle, and an analogue top.}
  (75)
\item
  \quotation{Although we are accustomed to thinking of digital in binary digits, digital has a more general meaning of discrete versus continuous flow of information.}
  (75)
\item
  \quotation{..the difference between print and electronic hypertext consists not in the presence or absence of digital and analogue modalities, but rather in the ways these modes are moblized as resources.}
  (76)
\item
  iconographic wriing versus alphabetic writing --- CONNECT TO
  MCLUHAN
\item
  \quotation{Typically the computer employs a digital mode at deeper coding levels, whereas in print, analogue continuity and digital coding both operate on the flat surface of the page.}
  (76)
\stopitemize

\subsubsubject{Point Three: Elecronic Hypertexts Are Generated through Fragmentation and Recombination}

\startitemize
\item
  \quotation{As a result of the frothy digital middle of the computer's structure, fragmentation and recombination are intrinsic to the medium.}
  (76)
\item
  Scrabble-based writing hghlights the digital nature of alphabetic
  writing
\item
  \quotation{With digital texts, the fragmentation is deeper, more pervasive, and more extreme than with the alphanumeric characters of print. Moreover much of the fragmentation takes place on levels inacessible to most users. This aspect of digital storage and retrieval can be mobilized as an artistic resource reappearing at the level of the user interface.}
  (77)
\stopitemize

\subsubsubject{Point Four; Electronic Hypertexts Have Depth and Operate in Three Dimensions}

\startitemize
\item
  analogue resemblance example: soundwave translated into diaphragm
  of microphone
\item
  \quotation{Whenever information flows between two differently embodied entities---for example, sound wave and microphone or microphone and recording device---analogue resemblance is likely to come into play because it allows one form of continuously varying information to be translated into a similarly shaped informational pattern in another medium.}
  (77)
\item
  \quotation{In contrast to the continuity of analogue pattern, the discreteness of code enables informatino to be rapidly manipulated and transmitted.}
  (78)
\item
  \quotation{Text on screen is produced through complex internal processes that make every word also a dynamic image, every dscrete letter a continuous process.}
  (78)
\item
  {\em scripton} vs {\em texton} : surface image vs underlying code
  (Aarseth)
  \startitemize
  \item
    stipple engraving as a print example
  \item
    texton can
    \quotation{refer to volatages, strings of binary code, or programming code, depending on whoe the \quote{reader} is taken to be. Scriptons always include the screen image but can also include any code visible to a user who is able to access different layers of program.}
    (78)
  \stopitemize
\item
  \quotation{With eletronic texts these is a clear distinction between scriptons that appear on screen and the textons of underlying code, which normaly remains invisible to the casual user.}
  (79)
\stopitemize

\subsubsubject{Point Five: Electronic hypertexts are bilingual, written in code as well as language}

\startitemize
\item
  \quotation{An electronic text does not exist if it is not generated by the appropriate hardware running the appropriate software.}
  (79)
\item
  \quotation{Rigorously speaking, an electronic text is a {\em process} rather than an object, although objects (like hardware and software) are required to produce it.}
  (79)
\item
  \quotation{The fact that creators o electronic texts always write code as well as natural language has resulted in a significant shift in how writing is understood.}
  (80)
\item
  \quotation{They [two e-text authors] refuse the distinction between writing that appears on the screen as the \quote{real} cretice effort, because they deeply understand, through their own creative practices, that screenic text and programming are logically, conceptually, and instrumentally entwined.}
  (80)
\item
  \quotation{While pushing toward envisioning print texts in electronic terms, however, he [McGann] also deeply understands that simulating print texts in electronic environments involves radically different materialities than the print texts in themselves.}
  (80)
\item
  \quotation{In all these activities, the hardware and software are active partners, facilitating and resisting, enabling and limiting, enacting and subverting. The labor needed to program these effects must be seen as intrinsic to the work of creation.}
  (81)
\item
  \quotation{\ldots{}the writer of an electronic text is intensely aware of the entwining o intellectual, physical, and technological labor that creates the text as a material object.}
  (81)
\stopitemize

\subsubsubject{Point Six: Electronic Hypertexts are mutable and transformable}

\startitemize
\item
  \quotation{The multiple coding levels of electronic textons allow small changs at one level of code to be quickly magnified into large changes at another level. The layered coding levels thus act like linguisti levers, giving a single keystroke the power to change the appearance of a textual image.}
  (81)
\item
  mutability and transformability are enabled by the
  \quotation{very rapid fragmentation and recombination of binary code}
  (81)
\item
  \quotation{In these cases [illusion of 3d in Word, Myst], both scriptons and textons are perceived as havng depth, with textons operating digitally through coding levels and scriptons operating analogically through screenic representation of three-dimensional spaces.}
  (81)
\stopitemize

\subsubsubject{Point Seven: Electronic hypertexts are spaces to navigate}

\startitemize
\item
  \quotation{Electronic hypertexts are navigable in at least two sense. They preent to the user a visual interface thaat must be navigated through choices the user makes to progress through the hypertext; and they are encoded on multiple levels that the user can access using the appropriate software, for example, by viewing the source code of a network browser as well as the surface text.}
  (83)
\item
  \quotation{As a result of its construction as a navigable space, electronic hypertext is intrinsivally more involved with issues of mapping and navigation than are most print texts.}
  (83)
\stopitemize

\subsubsubject{Point Eight: electronic hypertexts are written and read in distributed cognitive environments}

\startitemize
\item
  \quotation{These [cognitively sophistaced acts] frequently include acts of interpretation, as when the computer decides how to display text in a browser independent of choices the user makes.}
  (84)
\item
  \quotation{Books also create rich cognitive environments, but they passively embody the cognitions of writer, reader, and book designer rather than actively participate in cognition themselves.}
  (84)
\item
  \quotation{Whereas computers struggle to remain viable for a decade, books maintain backward compatibility for hundreds of years.}
  (84)
  \startitemize
  \item
    that is a significant goal of TeX: long-term stability at a level
    of 100\% backward compatibility
  \stopitemize
\item
  \quotation{The issue is not the technological superiority of either medium but rather the specific conditions a medium instantiates and enacts.}
  (84)
\item
  \quotation{Thus cognition is distributed not only between writer, reader, and designer (who may or may not be separate peopl) but also between humans and machines (which may or may not be regarded as separate entities).}
  (84)
\item
  comprehension depends
  \quotation{on all the parts working together corretly in this distributed cognitive system}
  (85)
\stopitemize

\subsubsubject{Point Nine: Electronic hypertexts initiate and demand cyborg reading practices}

\startitemize
\item
  \quotation{To be positioned as a cyborg is inevitably in some sense to become a cyborg, so electronic hypertexts, regardless of their content, tend toward cyborg subjectivity.}
  (85)
\item
  \ldots{}\quotation{electronic hypertexts necessarily enact it [cyborg subjectivity] through the specificity of the medium.}
  (85)
\item
  hard to simulate in book technology because book technology
  \quotation{remains remarkably simple to use.} (85)
\item
  \quotation{Hybrid forms, like the electronic book, show reverse remediation in action: as booksbecome more like computers, computers become more like books.}
  (86)
\item
  \quotation{In the rich medial ecology of contemporary literature, media differentiate as well as converge. Attention to material properties enhances our understanding of how some digital works are evolving along trajectories that increasingly diverge rom books as they experiment with the new possibilities opened up by electronic environments.}
  (86)
\item
  \quotation{In retrospect, we can see the view that the text is an immaterial verbal constuction as an ideology that inflicts the Cartesian split between mind and body upon the textual corpus, separating into two fictinoal entities what is in actuality a dynamically interacting whole.}
  (86)
\item
  \quotation{Rooted in the Cartesian tradition, this ideology also betrays a class and economic division between the work of creation---the privileged activity of the author as inspired genius---and the work of producing the book as a physical artifact, an activity relegated to publishers and booksellers.}
  (86)
\item
  \quotation{the traditional split between the work of creation and the work of production no longer obtains}
  (87)
\stopitemize

\stoptext

