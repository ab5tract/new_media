%\enableregime[utf]  % use UTF-8

\setupcolors[state=start]
\setupinteraction[state=start, color=blue] % needed for hyperlinks

\usemodule[simplefonts]
\setmainfont[Liberation-Serif]
\setmonofont[inconsolata]

\setuppapersize[A4][A4]  % use letter paper
%\setuplayout[width=middle, backspace=1.5in, cutspace=1.5in,
%             height=middle, header=0.75in, footer=0.75in] % page layout
\setuppagenumbering[location={footer,right}]  % number pages
\setupbodyfont[12pt]  % 11pt font
\setupwhitespace[small]  % inter-paragraph spacing

\setupindenting[medium]
\indenting[always]

\setuphead[section][style=\tfc]
\setuphead[subsection][style=\tfb]
\setuphead[subsubsection][style=\bf]

% define descr (for definition lists)
\definedescription[descr][
  headstyle=bold,style=normal,align=left,location=hanging,
  width=broad,margin=1cm]

% prevent orphaned list intros
\setupitemize[autointro]

% define defaults for bulleted lists 
\setupitemize[1][symbol=1][indentnext=no]
\setupitemize[2][symbol=2][indentnext=no]
\setupitemize[3][symbol=3][indentnext=no]
\setupitemize[4][symbol=4][indentnext=no]

\setupthinrules[width=15em]  % width of horizontal rules

% for block quotations
\unprotect

\startvariables all
blockquote: blockquote
\stopvariables

\definedelimitedtext
[\v!blockquote][\v!quotation]

\setupdelimitedtext
[\v!blockquote]
[\c!left=,
\c!right=,
before={\blank[medium]},
after={\blank[medium]},
]

\protect

\starttext

\subject{Notes with Florian}

\startitemize[n][stopper=.,width=2.0em]
\item
  Fundamental engineering difference between ConTeXt
  \startitemize
  \item
    no real formal document declaration
  \item
    not a semantic markup language, based on macros
  \item
    TeX is bottom-up, versus semantic (top down)
  \item
    Jade transformer takes XML and SGML into a plain TeX document
  \stopitemize
\item
  pandoc is a wrapper
  \startitemize
  \item
    look into alternatives
  \item
    least common denomitor
  \item
    resembles natural languages in that translations are \quote{lossy}
  \item
    post-design processes
  \item
    proprietary options?
    \startitemize
    \item
      Woodwing: developing similar systems, SmartBooks
    \stopitemize
  \stopitemize
\item
  cross-media publishing
  \startitemize
  \item
    nearly an artificial intelligence problematic
  \item
    point of departure: why do have to start from scratch for all these
    different output formats?
  \item
    unpredictability
  \stopitemize
\item
  generative typographic design
  \startitemize
  \item
    Conditional Design manifesto
  \item
    Luna Maurer, Roel Wouters, Jonathan Puckey (Amsterdam)
  \item
    LUST
  \item
    catalogtree (Rotterdam and Arnhem)
  \item
    OS Publishing (Open Source Publishing, Brussels)
    \startitemize
    \item
      Femke Snelting (uses ConTeXt)
    \stopitemize
  \item
    Scriptographer
  \item
    exhibition in Breda: Info Deco Data
  \item
    Petr van Blokland
    \startitemize
    \item
      \quotation{In our office we no longer use Adobe products, we program everything.}
    \item
      house designer of Rabobank
    \stopitemize
  \item
    Open
  \stopitemize
\item
  wrappers on top of wrappers
  \startitemize
  \item
    error handling
  \stopitemize
\item
  ODT does not live up to its promise
  \startitemize
  \item
    DocBook: most robust XML format
  \item
    TEI (Text Encoding Initiative): for electronic philology
  \item
    Word is a good point for re-mediation, as it was meant to emulate
    the typewriter.
    \startitemize
    \item
      typewriters are not document creation (typesetting) systems
    \item
      literal WYSIWYG
    \item
      allows visual poetry that is not possible elsewhere
    \item
      \quotation{free form typography,} creating a text visually
    \item
      per-line breaking
    \stopitemize
  \stopitemize
\item
  constant back and forth between ease of use vs robustness
  \subsubject{- same problem as exists for semantic web}
\item
  We have hit the limits of WYSIWYG, and designers are realizing
  this.
  \startitemize
  \item
    TexMacs: WYSIWYG
  \stopitemize
\item
  J David Bolter: re-mediation. hypertext/hypermedia. Grousin.
  {\em Re-mediation}.
  \startitemize
  \item
    the new medium holds the old medium, the old medium changes to fit
    the new
  \item
    Michael Heim. {\em Electric Language}.
    \startitemize
    \item
      how word processing changes writing
    \stopitemize
  \stopitemize
\item
  pandoc is not a heterarchy, but a hierarchy, with pandoc on top
  \startitemize
  \item
    \quote{media} was born in advertising in the 1940s
    \startitemize
    \item
      which medium is appropriate for the message,how do you mediate that
      message in transferring it to medium
    \stopitemize
  \item
    pandoc works in this classical media way, addressing concerns of
    mediating the message into various mediums
    \startitemize
    \item
      text-to-speech, morse code, etc.
    \stopitemize
  \item
    issues of scale: programming can be an inefficient solution
  \stopitemize
\item
  TeX limitations
  \startitemize
  \item
    no color separation
  \item
    \quotation{showstoppers}
  \stopitemize
\item
  Adobe has developed an XML format for InDesign; can generate
  programmitically
  \startitemize
  \item
    in FLOSS, generate SVG.
  \stopitemize
\item
  Distributed authorship, open source, to design and art.
  \startitemize
  \item
    the system most used at this point is Drupal
  \item
    FLOSS Manuals uses Drupal and htmldoc for printable documents.
    \startitemize
    \item
      pragmatic
    \stopitemize
  \stopitemize
\item
  relating to theory: intermediation/re-mediation
  \startitemize
  \item
    look into case examples of unfulfilled promises of computer
    technology
    \startitemize
    \item
      Xanadu
    \stopitemize
  \item
    in the 70s it seemed like document transformation was basic; then
    you look at the intricacies. even simple user scenarios tap into
    the problem of what a computer can and can't do; let alone what a
    user can and can't do.
  \stopitemize
\item
  music vs texts
  \startitemize
  \item
    music is not semantic
  \item
    no one expects a computer to produce music with a perfect voice;
    the expectation is something of its own
  \item
    yet with text, we have the expectation of reading text typeset in
    the same way as it is by hand
  \item
    perhaps an aesthetic difference: we don't expect electronic music
    to sound like analog;
  \stopitemize
\item
  Readability
  \startitemize
  \item
    TeX comes close, but
  \item
    player piano vs a real pianist == TeX vs a WYSIWYG enabled
    typographer
  \item
    there is no good kerning algorithm
    \startitemize
    \item
      why is text so difficult to create aesthetically?
    \stopitemize
  \stopitemize
\item
  Broaden perspective and see what else is going on in cross-media
  publishing
  \startitemize
  \item
    media theory
  \item
    perhaps ANT works but it doesn't come out clearly in the proposal
  \item
    focus on one theory that gets to where we want to go
  \item
    core analysis of the discussion, stick to one or two theories
  \stopitemize
\item
  Holy Grail: one system that serves as the universal document
  source.
  \startitemize
  \item
    in the end it might be many holy grails, one for each type of
    documents
  \item
    can you reach the Holy Grail without complexity making it
    unworkable?
  \stopitemize
\stopitemize

\stoptext

