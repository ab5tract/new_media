%\enableregime[utf]  % use UTF-8

\setupcolors[state=start]
\setupinteraction[state=start, color=blue] % needed for hyperlinks

\usemodule[simplefonts]
\setmainfont[Liberation-Serif]
\setmonofont[inconsolata]

\setuppapersize[A4][A4]  % use letter paper
%\setuplayout[width=middle, backspace=1.5in, cutspace=1.5in,
%             height=middle, header=0.75in, footer=0.75in] % page layout
\setuppagenumbering[location={footer,right}]  % number pages
\setupbodyfont[12pt]  % 11pt font
\setupwhitespace[small]  % inter-paragraph spacing

\setupindenting[medium]
\indenting[always]

\setuphead[section][style=\tfc]
\setuphead[subsection][style=\tfb]
\setuphead[subsubsection][style=\bf]

% define descr (for definition lists)
\definedescription[descr][
  headstyle=bold,style=normal,align=left,location=hanging,
  width=broad,margin=1cm]

% prevent orphaned list intros
\setupitemize[autointro]

% define defaults for bulleted lists 
\setupitemize[1][symbol=1][indentnext=no]
\setupitemize[2][symbol=2][indentnext=no]
\setupitemize[3][symbol=3][indentnext=no]
\setupitemize[4][symbol=4][indentnext=no]

\setupthinrules[width=15em]  % width of horizontal rules

% for block quotations
\unprotect

\startvariables all
blockquote: blockquote
\stopvariables

\definedelimitedtext
[\v!blockquote][\v!quotation]

\setupdelimitedtext
[\v!blockquote]
[\c!left=,
\c!right=,
before={\blank[medium]},
after={\blank[medium]},
]

\protect

\starttext
	{\tfd Formatting Processes: A reflexive investigation of generative typesetting using FLOSS}
	\starttabulate[|l|] 
	\NC {\tfa John Haltiwanger} 
	\NR
  \NC 15 April 2010
	\stoptabulate

\subject{Core concerns}

This paper claims that the type of material analysis in which
Hayles engages is an inadequate formulation of a valid
concern.{\em Scriptons} and {\em textons} are two terms proposed by
Esper Aarseth and utilized by Hayles to refer to the surface of
text and the underlying codes that produces that text,
respectively. Confusingly, if some degree of code is available for
investigation, that code is considered part of the {\em scripton}
(Hayles 2004: 78).
\footnote{This isn't to mention the confusion, from a programming
perspective, of using the root words \quote{script} to refer to
surface level and \quote{text} to refer to programmatic layers
beneath that surface. This is counterintuitive to a tradition in
which \quote{scripts} are written to deal with \quote{text}
algorithmically.}
This complicates the capacity for discussion as the code that is
most relevant for the presentation of text on a screen in any given
{\em specific} instance is necessarily the source code. If the
source code is considered part of the \quote{surface text}, there
is little left to truly investigate on a per-document basis. As
important an observation it is that all electronic text begins as
voltage signals, it is not much more informative than stating that
all printed text begins as ink.

One goal of this paper is to push the boundaries of material
analysis to include a detailed examination of surface text and the
code used to generate it. In order to do so this paper will focus
on the relationships of inputs and outputs in a generative design
workflow, specifically the typesetting of the thesis itself. The
variance of the chosen output formats in both typographic quality
as well as compositional elements allows the interrogation of
electronic text on a deeper level. This is not meant to be a simple
features analysis. Rather, the goal is to develop a theoretical
framework for discussing the separate layers of {\em text},
{\em code}, and {\em type} (as in typography). Without considering
issues of presentation (type) or composition (code), a material
analysis remains incomplete. In other words, there is a lot of room
in Hayles' \quotation{frothy digital middle} for a deeper level of
discussion (Hayles 2004: 75). This froth will be examined in light
of the remediation theory provided by Jay David Bolter and David
Grusin.

\subject{Research Questions}

\startitemize
\item
  What claims are made about the qualities of a \quote{format} in
  relation to a \quote{medium}? Within computers, to what extent can
  a specific format yield the same effects/components of a medium?
\item
  What does a material analysis of a {\em generative process} look
  like? If a text is merely an input in a process that yields
  specifically different outputs, what kind of materiality does it
  have?
\item
  If typography has a material effect, on which layer does it
  operate? Should typography be incorporated into the material
  analyses of documents in the tradition of Hayles?
\item
  What claims are made about computer typography? and what claims, if
  any, are specific open source software?
\stopitemize

\subject{Generative design}

Generative practices increasingly take a larger prominence in
design workflows. \quotation{The process is the product,} declares
the {\em Conditional Design Manifesto}. Employing the
\quotation{methods of philosophers, engineers, inventors and mystics,}
the four authors of the manifesto seek to abandon the idea of a
product in favor of
\quotation{things that adapt to their environment, emphasize change and show difference}
(Maurer, et al.). The text is split into a prose introduction and
three sections of manifesto-style declarative sentences:
{\em process}, {\em logic}, and {\em input}. These are the
governing principles of their proposed angle towards design.

By focusing on process, the authors of the manifesto employ a
conscious decision away from productized design. Process produces
\quotation{formations rather than forms} and facilitates a search
for \quotation{unexpected but correlative, emergent patterns}
(Maurer, et al.). Logic is invoked as their method for
\quotation{accentuating the ungraspable} and provides the means for
designing the
\quotation{condtions through which the process can take place}.
Input is their \quotation{material,} and should come from the
\quotation{external and complex environment}. It
\quotation{engages logic and influences the process}

The powerful Java-platform tool named Processing has become wildly
popular amongst visualization designers. The demo scene has long
employed generative techniques to instantiate complexity on-screen
using as few as 4 kilobytes of compiled binary code. Video game
design guru Will Wright's 2008 game {\em Spore} relies heavily on
procedural generation. Many open-source (and, presumably,
otherwise) projects generate their documentation using a
combination of algorithm and in-code markup.
\footnote{Indeed one of the available input formats (still) in consideration
for this thesis is reStructuredText, the markup syntax which lies
at the core of the Python programming language's online
documentation.}
Beyond this, Dutch designer Petr van Blokland has ceased using
Adobe products in the design pipeline at his firm. By
{\em programming} instead of {\em hand-crafting} Blokland is able
to achieve a balance of control and flexibility that can support
the kind of enormous workflow demanded by clients such as Rabobank.

The case study at the core of this paper will explore these
concerns as the materiality of the thesis itself remains ethereal.
It is more than its input, it is more than its output: it is
literally and figuratively very much a {\em process}.

The substrate of the text you are reading right now is a simple
formal markup language called Markdown. Originally written in Perl,
it has spread to any language that ends up touching the web. This
is due to its popularity as an approachable \quote{pre-formatting}
markup language for operations, such as blog posts and comments,
where HTML is the desired output but a simpler
{\em and more readable} input is often preferrable. Using a
\quote{wrapper} tool called {\bf Pandoc}, the Markdown input is
converted to HTML, ConTeXt, and OpenOffice.org Writer (ODT)
formats. Through customizable templates, {\bf Pandoc} provides a
great deal of flexibility to the end-user.

\subsubject{Caveats}

Florian Cramer offers a techno-musical analogy for generative
design in the context of typesetting: TeX is to WYSIWYG typesetting
as a player piano is to a piano under the fingertips of a real
pianist (Cramer 2010). Cramer also Femke Snelting, of Open Source
Publishing in Brussels, has mentioned \quotation{showstopper}
issues facing the enactment of specific design goals in OSP's first
project using ConTeXt (Snelting 2010).

\subject{Theoretical Perspectives}

Without having a great deal of literature on \quotation{formats} to
work with, I would like to assert that formats are sub-species of
mediums. The sitcom is a television format, the action blockbuster
a cinematic format, and the concrete poem a writing format. It is
perhaps within the medium of the microcomputer, however, that the
largest number of formats can be found. As a fully programmable
medium, the microcomputer facilitates the development of an
enormous number of formats. These formats can carry political and
social significance, as in the case of HTML 5 and the debate over
the use of the patent-encumbered h.264 codec versus the
open-source, royalty free Ogg Theora format. In the case of
computers, then, it is perhaps possible to assert that formats have
a more distinct, medium-like dimension than formats in other media
have.

In his piece \quotation{Media Cold and Hot}, Marshall McCluhan
argues that a medium's capacity to capture a sense defines, in some
ways, its social effect. The state of the society to begin with is
critical and will determine the nature of a new medium's impact.
First it is important to establish what his scale is based on and
how he establishes it.
\quotation{A hot medium is one that extends one single sense in \quote{high definion}. High definition is the state of being well filled with data}
(McLuhan 1964: 24). Hot media require little participation because
they are so engrossing---their \quotation{high resolution} (to
switch his phrasing a bit) doesn't provide or require extensive
participation. Cool media, on the other hand, require a great deal
of audience participation.

In McLuhan's view, nationalism and religious wars in sixteenth
century Europe were the result of a \quotation{hotting-up} of the
medium of writing as it transitioned to print (25). Furthermore,
\quotation{a tribal and feudal hierarchy of traditional kind collapses quickly when it meets any hot medium of the mechanical, uniform, and repetitive kind}
(26). In this way media can be seen as highly influential, having
the power to completely alter societies, as the introduction of the
letter press and the subsequent collapse of Catholic hegemony in
various parts of Europe.

\subject{Materialist Format Analysis}

One of the primary objectives of this paper is to force integration
of typographical design principles into a material analysis. Robin
Kinross' argues that the first typographically modern typesetting
began much earlier than what is now conisdered to be the modern era
(Kinross).

\stoptext

