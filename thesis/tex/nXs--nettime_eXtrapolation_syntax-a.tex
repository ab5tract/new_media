\usemodule[simplefonts]
\setmainfont[Liberation-Serif]

\setuppapersize[A4]

\setupwhitespace[medium]

\setuppagenumbering[location={footer,left,margin}]


\starttext

\subject{nettime eXtrapolation Syntax}

\subsubject{A proposal to up the game.}

Turn emails into beautiful presentation of concepts in text.

There can be one input that generates many outputs. Evaluate
quality based on ease of accessibility. Eyes matter, but the way a
text is perceived matters more.

\startitemize
\item
  here is an itemize
\item
  here is another
  \startitemize
  \item
    sub bullet
  \stopitemize
\stopitemize

Eat plain-text with crowdsourcing and turn all archived classics
into new Materials.

{\em If I sound ambitious, its because the software is all there to turn this into reality. More important than available toolsets, then, is}
Motivation.

Halfway to four 'M's, then, I'll re-iterate the importance of More.
All of life is the product of over abundance, so it seems stupid to
deny one the right to be many, in whatever pieces seem necessary.

Because the only thing missing is a proper bibliographic standard,
defer to the least surprising available standard (CSL?) but aim for
the driving of the best (RDFa).

Where a resource differs, a resource forks. Allow a standard
syntactical interface to provide access to things such as
\quote{opposite}, \quote{refutation}, and \quote{analogous}.
\footnote{See Daniel Yoder's elaborations on RESTfulness, a hot buzz among
the web application developers in the Ruby/Rails space.}
Anyone can contribute connections.

Mobility, is the rounding out word, heretofore delivered before
More, but comprising the hidden capacity of Four. Let's
numerologize on our way to somewhere different. It's going to take
a standard way of talking with one another. Don't let the old
legends of a scournful diety deter you---this quest is not only
righteous but ancient, and necessary at the level of the
contemporary. We can speak the same language again.

Texts should always be attached to links not necessarily of the
originators' choosing. This is called Vocal Democracy. An Equal
Opportunity for Voices, embedded because it is already present in
human sociality. Let the assemblage reflect itself.
\footnote{This project is, in part, based on the idea that ANT theory is best
applied in real-time. For an easy example, see
\useURL[1][http://ubervu.com/][][ubervu.com]\from[1].}

But most importantly, integrate cleanly with plaintext. It's more
than a motto, it's a question of Mobility. Never get caught pants
down with a binary format in a land without translators.

In other words, thank god for open source. Thank God for the urge
for Freedom. That, ultimately, is the difference---for one morality
comes with a lower-case \quote{m}.

Because if there is nothing besides the consciousness inside you,
your definitions of what is necessary are necessarily skewed. So,
I'd expect, is the lot of all of us.

This cyberspace was always a land of words. It's time we typeset it
properly. Let's let our words speak as loudly as the medium will
let us. Never in some back alley mail archive typewriter font,
unless that's the way they request it. And in that case, for the
audience, it's the best possible way to look good.

Which brings us back to the realm of the Material. Manifested
through a web application that translates the document into
{\em whatever format you have requested}, this document is
electronically prepared for your mind.

Never forget that a screen and a sheet are different. The best
texts account for both. That's why we provide an easy option to
print something else than plaintext. Not just something else, but
something suitable for the saturation of tree carcass in ink
symbols.
\footnote{Or, in a more sustainable world, hemp cadavers.}

\stoptext