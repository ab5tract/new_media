
\starttext

\startlines

The Electronic literature Organization has defined electronic literature \quotation{to include both work performed in digital media and work created on a computer but published as print.} (3)

electronic literature even extends in to VR; Robert Coover's {\it Screen} provides an experience of text crumbling from the walls of the CAVE, forcing the reader to interact with text in a new mode; held back by the costs of CAVE, \quotation{thus sacrificing the portability, low cost, robust durability, and mass distribution that made print literature a transformative social and cultural force.} (14)

(Quoting Janet Murray) \quotation{Giving the audience access to the raw materials of creation runs the risk of undermining the narrative experience,} she writes, while still acknowledging that \quotation{calling attention to the process of creation can also enhane the narrative involvement by inviting readers/viewers to imagine themselves in the place of the creator.} (16)

(Regarding \quote{code work}) \quotation{Replete with puns, neologisms,and othr creative play, such work enacts a trading zone in which human-only language and machine-readable code are performed as interpenetrating linguistic realms, thus making visible on the screeni surgace a condition intrinsic to all electronic textuality, namely the intermediating dynamics between human-only languages and machine-readable code. By implication, such works also reference the complex hybridization now underway between human cognition and the very different and yet interlinked cognition of intelligent machines...} (21)

\quotation{...when a work is reconceived to take advantage of the behavioral, visual, and/or sonic capabilities of the Web, the result is not just a Web \quote{version} but an entirely different artistic production that should be evaluated in its own terms with a critial approach fully attentitive to the specificity of the medium. Moreover, in a few cases where the print and digital forms are conceptualized as one work distributed over two instantiations, as is the case with {\it V}, possibilities for emergent meanings multiply exponentially through the differences, overlap, and convergences of the instantiations compared with one another.} (23)

Matthew Kirschenbaum establishes two aspects of digital media's materiality: forensic materiality versus formal materiality (the physical materiality versus the materiality via software (25)

\quotation{Unlike a print book, electronic text literally cannot be accessed without running code. Critics and scholars of digital art and literature should therefore properly consider the source code to be part of the work, a position underscored by authors who embed in the code information or interpretive comments crucial to understanding the work.} (35)

(On Adrian Mackenzie's {\it Cutting Code}) \quotation{Mackenzie's work serves as a salutary reminder that just as one cannot understand the evolution of print literature without taking into account such phenomena as the court decisions establishing legal precedent for copyright and the booksellers and publishers who helped promulgate the ideology of the creative genius authoring the great work of literature (for their own purposes, of course), so electronic literature is evolving within complex social and economic networks that include the development of commercial software, the competing philosophy of open source freeware and shareware, the economics and geopolitical terrain of the internet and World Wide Web, and a host of other factors thaat directly influence how electronic literature is created and stored, sold, or given away, preserved or allowed to decline into obsolescence.} (39)

\quotation{Exploring and understanding the full implications of what the transition from page to screen entails must necessarily be a community effort, a momentous task that calls for enlightened thinking, visionary planning, and deep critical consideration.} (42)


\startlines

\stoptext
