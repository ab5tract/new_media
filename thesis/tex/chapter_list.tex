%\enableregime[utf]  % use UTF-8

\setupcolors[state=start]
\setupinteraction[state=start, color=blue] % needed for hyperlinks

\usemodule[simplefonts]
\setmainfont[Liberation-Serif]
\setmonofont[inconsolata]

\setuppapersize[A4][A4]  % use letter paper
%\setuplayout[width=middle, backspace=1.5in, cutspace=1.5in,
%             height=middle, header=0.75in, footer=0.75in] % page layout
\setuppagenumbering[location={footer,right}]  % number pages
\setupbodyfont[12pt]  % 11pt font
\setupwhitespace[small]  % inter-paragraph spacing

\setupindenting[medium]
\indenting[always]

\setuphead[section][style=\tfc]
\setuphead[subsection][style=\tfb]
\setuphead[subsubsection][style=\bf]

% define descr (for definition lists)
\definedescription[descr][
  headstyle=bold,style=normal,align=left,location=hanging,
  width=broad,margin=1cm]

% prevent orphaned list intros
\setupitemize[autointro]

% define defaults for bulleted lists 
\setupitemize[1][symbol=1][indentnext=no]
\setupitemize[2][symbol=2][indentnext=no]
\setupitemize[3][symbol=3][indentnext=no]
\setupitemize[4][symbol=4][indentnext=no]

\setupthinrules[width=15em]  % width of horizontal rules

% for block quotations
\unprotect

\startvariables all
blockquote: blockquote
\stopvariables

\definedelimitedtext
[\v!blockquote][\v!quotation]

\setupdelimitedtext
[\v!blockquote]
[\c!left=,
\c!right=,
before={\blank[medium]},
after={\blank[medium]},
]

\protect

\starttext

\subject{Thesis Chapter List}

\startitemize[n][stopper=.]
\item
  Introduction
  \startitemize
  \item
    Re-mediation: the drive towards
    \quotation{greater authenticity and immediacy of presentation}
    (Bolter 2001: 70)
  \item
    Historical developments of the various formats (Typewriter
    -\lettermore{} Word, SGML -\lettermore{} HTML, Typesetting math
    -\lettermore{} TeX)
  \item
    Contrast the approaches: WYSIWYG, semantic markup, formal markup
  \item
    the relationship between computers and language (Cramer 2001)
    \startitemize
    \item
      computers run on alphabets;
      \quotation{Literature is therefore a privileged symbolic form in digital information systems.}
      (Cramer 2001: 2)
    \item
      the rule of searching for a specific text phrase is complicated by
      formats: intermediate layers are required for text searching when
      ODT/PDF are involved
    \stopitemize
  \item
    the re-remediation of Project Gutenberg: the availability of ASCII
    (itself a remdiation) allows for translation across interfaces; the
    appearance of the tablets/e-readers as interfaces to the ASCII
    re-mediates them back into a book
  \stopitemize
\item
  A history of cross-media publishing and generative design
  \startitemize
  \item
    Simondon's {\em transduction}:
    \quotation{By transduction we mean an operation---physical, biological, mental, social---by which an activity propagates itself from one element to the next, within a given domain, and founds this propagation on a structuration of the domain that is realized from place to place: each area of the constituted structures serves as the principle and the model of the next area, as a primer for its constitution, to the extent that the modification expands progressibely at the same time as the structuring operation}
    (Simondon 2009: 11).
    \startitemize
    \item
      the \quotation{pre-format} as preindividual (but still individual);
      ontologies on top of ontologies
    \stopitemize
  \item
    History of the term \quote{media,} it's origins in advertising as
    they developed a language to discuss \quote{mediating} messages,
    that is, tailoring the message to suit a particular \quote{medium}
  \item
    Advantages and disadvantages of generative design.
    \startitemize
    \item
      Issues of scale can considerably affect the suitability of a
      generative workflow over a traditional one.
    \stopitemize
  \item
    The Holy Grail:
    \quotation{one system that serves as the universal document source}
    \startitemize
    \item
      Resembles other holy grails, \quotation{artificial intelligence}
      and
      \quotation{real-time collaborative editing of the same document}
    \stopitemize
  \item
    Wrappers: software for remediating formats
    \startitemize
    \item
      the ultimate intersection point
    \stopitemize
  \item
    Showstoppers: errors and/or limitations in functionality that force
    abandoning one approach for another
    \startitemize
    \item
      limitations of bibliographic intersections
    \stopitemize
  \item
    The role of FLOSS in generative design
  \item
    bootstrapping - Engelbart's idea of using computer tools to make
    better computer tools
  \stopitemize
\item
  Approaches to digital typesetting
  \startitemize
  \item
    Print is Static, Code is Process: the physics of text
    \startitemize
    \item
      Hayles requests taking into account physical specificity; the last
      chapter will include memory heap visualizations in order to
      interrogate the utility of such an approach
      \startitemize
      \item
        what about the physicality of the writer in crafting the text? this
        leads directly into questions of workflow
      \stopitemize
    \stopitemize
  \item
    WYSIWYG: the computer as a typewriter
    \startitemize
    \item
      Concrete poetry and \quotation{free form typography} as unique to
      this approach.
    \item
      \quotation{A typewriter (or a computer-drive printer of the same quality) that justifies its lines in imitation of typesetting is a presumptious, uneducated machine, mimicking the outward form instead of the inner truth of typography.}
      (Bringhurst 2008: 28)
    \stopitemize
  \item
    Semantic markup: interpreted plaintext (top-down)
    \startitemize
    \item
      Tuned for remediation, strict separation of display from content.
    \stopitemize
  \item
    Formal markup: the document typesets itself (bottom-up)
    \startitemize
    \item
      The long and varied history of TeX.
    \item
      Why ConTeXt?
    \stopitemize
  \stopitemize
\item
  A Generative Methodology
  \startitemize
  \item
    Is this process useful? Does such a practical approach have
    anything to offer theory? And does theory offer anything useful for
    the practice?
  \item
    Visualizations: memory heap analysis (again, is it useful? does it
    map a \quote{materiality}?)
    \startitemize
    \item
      also, version control visualizations of the git repository that
      hosts the thesis files
    \stopitemize
  \item
    Case studies
    \startitemize
    \item
      HTML
    \item
      OpenOffice.org
    \item
      ConTeXt
    \stopitemize
  \stopitemize
\item
  Conclusion
\stopitemize

\stoptext

