%\enableregime[utf]  % use UTF-8

\setupcolors[state=start]
\setupinteraction[state=start, color=blue] % needed for hyperlinks

\usemodule[simplefonts]
\setmainfont[Liberation-Serif]
\setmonofont[inconsolata]

\setuppapersize[A4][A4]  % use letter paper
%\setuplayout[width=middle, backspace=1.5in, cutspace=1.5in,
%             height=middle, header=0.75in, footer=0.75in] % page layout
\setuppagenumbering[location={footer,right}]  % number pages
\setupbodyfont[12pt]  % 11pt font
\setupwhitespace[small]  % inter-paragraph spacing

\setupindenting[medium]
\indenting[always]

\setuphead[section][style=\tfc]
\setuphead[subsection][style=\tfb]
\setuphead[subsubsection][style=\bf]

% define descr (for definition lists)
\definedescription[descr][
  headstyle=bold,style=normal,align=left,location=hanging,
  width=broad,margin=1cm]

% prevent orphaned list intros
\setupitemize[autointro]

% define defaults for bulleted lists 
\setupitemize[1][symbol=1][indentnext=no]
\setupitemize[2][symbol=2][indentnext=no]
\setupitemize[3][symbol=3][indentnext=no]
\setupitemize[4][symbol=4][indentnext=no]

\setupthinrules[width=15em]  % width of horizontal rules

% for block quotations
\unprotect

\startvariables all
blockquote: blockquote
\stopvariables

\definedelimitedtext
[\v!blockquote][\v!quotation]

\setupdelimitedtext
[\v!blockquote]
[\c!left=,
\c!right=,
before={\blank[medium]},
after={\blank[medium]},
]

\protect

\starttext

\subject{Week 3 - The Happening of A Relative Absence}

\subsubject{Meeting 19 . iv . 010}

\startitemize[n][stopper=.]
\item
  investigating mediums is done by examining qualities/properties
  \startitemize
  \item
    by looking at the formats it \quotation{hosts} (how to pick that
    verb?)
  \stopitemize
\item
  how to study media
  \startitemize
  \item
    definitional step (properties)
  \item
    Manovich: computer complicates medium understanding, mking it no
    longer so easy just to define media strictly by their properties
    \startitemize
    \item
      becomes focused on input and output
    \stopitemize
  \item
    chronological order of theory?
    \startitemize
    \item
      Mcluhand -\lettermore{} Bolter -\lettermore{} Hayles -\lettermore{}
      Manovich
    \stopitemize
  \item
    media through McLuhan: based on sensory captivation;
    \quotation{effects} analysis
  \stopitemize
\item
  how to treat computer typography
  \startitemize
  \item
    what is the relationship between medium analysis and computer
    typography?
  \item
    is it a medium?
  \item
    cross-media challenge
    \startitemize
    \item
      transmedia?
    \item
      transference (a cross-media strategy)
    \item
      there could be other approaches
    \item
      \quotation{one could evaluate various strategies for moving to the screen in terms of media theory.}
      \startitemize
      \item
        is it a remediation problem? cross-media problem? does the problem
        go away if looked at in a different sense, a la transmedia?
      \stopitemize
    \item
      what can theory do?
      \startitemize
      \item
        reformulate the problem in ways that lead to more productive work
        in that area
      \item
        can theory help redefine or reposition the problem with computer
        typography?
      \stopitemize
    \stopitemize
  \stopitemize
\stopitemize

\subsubject{TODO}

\startitemize
\item
  read up \quote{genres}
  \startitemize
  \item
    \quote{genres} vs \quote{formats} ?
  \item
    van der Pol teaches a whole genre theory class
  \stopitemize
\item
  cross-formats? (can formats be found in other formats a la Hayles?)
\item
  cross-media / transmedia (Jenkins)
  \startitemize
  \item
    same story vs story segmented across media
  \stopitemize
\item
  the complicated nature of typography can be seen as through
  remediation?
\item
  Nature of the problem??
  \startitemize
  \item
    classic problems of conversion
    \startitemize
    \item
      Mars rover metric vs english
    \stopitemize
  \item
    the {\em reason} to go to media theory
  \stopitemize
\item
  what are the mmost obvious ways of studying this?
  \startitemize
  \item
    make the point for introucing another point of view
  \item
    how is it researched normally
  \stopitemize
\stopitemize

\stoptext

