%\enableregime[utf]  % use UTF-8

\setupcolors[state=start]
\setupinteraction[state=start, color=blue] % needed for hyperlinks

\usemodule[simplefonts]
\setmainfont[Liberation-Serif]
\setmonofont[inconsolata]

\setuppapersize[A4][A4]  % use letter paper
%\setuplayout[width=middle, backspace=1.5in, cutspace=1.5in,
%             height=middle, header=0.75in, footer=0.75in] % page layout
\setuppagenumbering[location={footer,right}]  % number pages
\setupbodyfont[12pt]  % 11pt font
\setupwhitespace[small]  % inter-paragraph spacing

\setupindenting[medium]
\indenting[always]

\setuphead[section][style=\tfc]
\setuphead[subsection][style=\tfb]
\setuphead[subsubsection][style=\bf]

% define descr (for definition lists)
\definedescription[descr][
  headstyle=bold,style=normal,align=left,location=hanging,
  width=broad,margin=1cm]

% prevent orphaned list intros
\setupitemize[autointro]

% define defaults for bulleted lists 
\setupitemize[1][symbol=1][indentnext=no]
\setupitemize[2][symbol=2][indentnext=no]
\setupitemize[3][symbol=3][indentnext=no]
\setupitemize[4][symbol=4][indentnext=no]

\setupthinrules[width=15em]  % width of horizontal rules

% for block quotations
\unprotect

\startvariables all
blockquote: blockquote
\stopvariables

\definedelimitedtext
[\v!blockquote][\v!quotation]

\setupdelimitedtext
[\v!blockquote]
[\c!left=,
\c!right=,
before={\blank[medium]},
after={\blank[medium]},
]

\protect

\starttext

\subject{Interface}

Galloway offers a tentative definition of interface:
\quotation{the interface is defined as the artificial differentiation of two media}
(Galloway 2010). The interface can be seen in light of Dagognet's
concept of a fertile nexus,

\subsubject{Popularity as the Ultimate Tractor App}

In some ways popularity is the granular mechanism of achieving an
interface that is simply used, not seen. That is to say, the
relative ubiquity of an interface in part defines whether that
interface appears \quote{natural} or not.

The entire levelling of standards within typographic design evolved
in the context of readability. What is attractive to the eye? How
do you \quote{encode} messages through the formatting of text on a
printed page in a way that is striving towards, and at least to
some degree achieving, invisibility?

It is no secret that for a long time text sought invisibility in
its conveyance. That is a romantic ideal which has stuck wth us.
The author is privileged, perhaps in no small part to the
physicality of a text's construction. What can lead to a finished
state of readability? The audience can never know beyond its
aesthetics (including its textual aesthetics, tuat is to say it's
\quotation{wording}.

\subsubject{Concrete Poetry}

Concrete poetry developed through the relations of certain human
processes with the interface of the typewriter.

\stoptext

