%\enableregime[utf]  % use UTF-8

\setupcolors[state=start]
\setupinteraction[state=start, color=blue] % needed for hyperlinks

\usemodule[simplefonts]
\setmainfont[Liberation-Serif]
\setmonofont[inconsolata]

\setuppapersize[A4][A4]  % use letter paper
%\setuplayout[width=middle, backspace=1.5in, cutspace=1.5in,
%             height=middle, header=0.75in, footer=0.75in] % page layout
\setuppagenumbering[location={footer,right}]  % number pages
\setupbodyfont[12pt]  % 11pt font
\setupwhitespace[small]  % inter-paragraph spacing

\setupindenting[medium]
\indenting[always]

\setuphead[section][style=\tfc]
\setuphead[subsection][style=\tfb]
\setuphead[subsubsection][style=\bf]

% define descr (for definition lists)
\definedescription[descr][
  headstyle=bold,style=normal,align=left,location=hanging,
  width=broad,margin=1cm]

% prevent orphaned list intros
\setupitemize[autointro]

% define defaults for bulleted lists 
\setupitemize[1][symbol=1][indentnext=no]
\setupitemize[2][symbol=2][indentnext=no]
\setupitemize[3][symbol=3][indentnext=no]
\setupitemize[4][symbol=4][indentnext=no]

\setupthinrules[width=15em]  % width of horizontal rules

% for block quotations
\unprotect

\startvariables all
blockquote: blockquote
\stopvariables

\definedelimitedtext
[\v!blockquote][\v!quotation]

\setupdelimitedtext
[\v!blockquote]
[\c!left=,
\c!right=,
before={\blank[medium]},
after={\blank[medium]},
]

\protect

\starttext

\subject{Formats}

Formats are function calls. They require standards and
standards-based libraries. There is no point to a format if there
is nothing there to decode it. Nihilism, however, provides no
proper formats---nothing is worth the effort of decoding. So issues
of \quote{nothing} and \quote{something} are complicated by the
mercurial nature of software. There are an infinity of inbetweens
on the way to an invisible interface.

In some ways popularity is the granular mechanism of achieving an
interface that is simply used, not seen. That is to say, the
relative ubiquity of an interface in part defines whether that
interface appears \quote{natural} or not.

The entire levelling of standards within typographic design evolved
in the context of readability. What is attractive to the eye? How
do you \quote{encode} messages through the formatting of text on a
printed page in a way that is striving towards, and at least to
some degree achieving, invisibility?

Formats pre-date digital computing. The realm of academic
publishing is awash in formats: Chicago, APA, MLA, etc. These
formats are achieved via compositional guidelines. These guidelines
are constructed with consideration to issues ranging from ink use
to politics. [need more history of these formats]

\subsubject{Formats as Databases}

[Pending]

\subsubject{Bibliography}

If the format is a database, what of the fact that it can't
accurately describe itself without human intervention? Binary files
are content-opaque. Even text files require some form of semantic
standard if their contents are to be understood without dedicating
time to mining the text for contextual clues. This begs the
question of bibliography.

Bibliography is the most visible difference between the formats of
academia. It is also a point of regression for the tri-mediation
project that brings this text to the screen or page you are reading
it on. This is despite the fact that both OpenOffice.org and
ConTeXt have, through third party modules, facilities to greatly
reduce the tedium and complexity of citation.
\footnote{In the case of OO.o, there is Zotero
(\useURL[1][http://zotero.org][][http://zotero.org]\from[1]) and in
the case of ConTeXt there is the \type{bib} module
(\useURL[2][][][]\from[2]) as well as the possibility of
integrating any other TeX-compatible bibliographic module through
some degree of individual effort.}
The wrapper software \type{pandoc} constrains me to self-managed
citations because it does not integrate bibliography into its own
workflow: it simply passes the TeX command \type{\cite{}} to the
next stage in the toolchain, providing that next stage is intended
for TeX. If any other file format is used, the \type{\cite{}}
command is ignored.

\stoptext

