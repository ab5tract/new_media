%\enableregime[utf]  % use UTF-8

\setupcolors[state=start]
\setupinteraction[state=start, color=blue] % needed for hyperlinks

\usemodule[simplefonts]
\setmainfont[Liberation-Serif]
\setmonofont[inconsolata]

\setuppapersize[A4][A4]  % use letter paper
%\setuplayout[width=middle, backspace=1.5in, cutspace=1.5in,
%             height=middle, header=0.75in, footer=0.75in] % page layout
\setuppagenumbering[location={footer,right}]  % number pages
\setupbodyfont[12pt]  % 11pt font
\setupwhitespace[small]  % inter-paragraph spacing

\setupindenting[medium]
\indenting[always]

\setuphead[section][style=\tfc]
\setuphead[subsection][style=\tfb]
\setuphead[subsubsection][style=\bf]

% define descr (for definition lists)
\definedescription[descr][
  headstyle=bold,style=normal,align=left,location=hanging,
  width=broad,margin=1cm]

% prevent orphaned list intros
\setupitemize[autointro]

% define defaults for bulleted lists 
\setupitemize[1][symbol=1][indentnext=no]
\setupitemize[2][symbol=2][indentnext=no]
\setupitemize[3][symbol=3][indentnext=no]
\setupitemize[4][symbol=4][indentnext=no]

\setupthinrules[width=15em]  % width of horizontal rules

% for block quotations
\unprotect

\startvariables all
blockquote: blockquote
\stopvariables

\definedelimitedtext
[\v!blockquote][\v!quotation]

\setupdelimitedtext
[\v!blockquote]
[\c!left=,
\c!right=,
before={\blank[medium]},
after={\blank[medium]},
]

\protect

\starttext

\subject{Code}

The relationship of code to language is that of a subset
constrained by the specificities of syntax (Cramer 2001). The
digital computer is ruled by syntax, which could be considered the
defining means of mediation between digital computers and human
processes. These processes include both the actions of the users
and the objects created, stored, distributed, and displayed on
digital computers.

Florian Cramer identifies language as a
\quotation{privileged symbolic form} within the context of digital
information processing (2001: 2). This privilege results from the
fact that the digital computer itself operates on an alphabet---the
0s and 1s feeding the processor in intervals measured in
nanoseconds.

Galloway points to the entirely artificial differentiations between
ASCII text on a web page and the ASCII text itself from which the
web page is rendered (Galloway 2010). This is the point where
distinctions between code, formats, and interfaces begin blur. Code
is the mechanism which enables formats. Formats in turn mediate
their contents, rendering (and often rendered by) different
interfaces. The code is an interface to viewing formatted content,
and the format is the interface through which the content is
presented. The intersection of these elements displays the inherent
slipperiness that undergirds our understanding of \quote{media} at
a time when more and more media are being subsumed by and
remediated onto digital devices.

\subsubject{Software Design}

There are definite limits when it comes to constructing software.
Beyond the obvious limitations imposed by computing capacities and
architectures lie sticky issues of development. In his seminal
{\em The Mythical Man-Month}, Frederick Brooks describes boundaries
for software construction that seem to betray logic. Adding money
and manpower, for instance, do not increase productivity. In fact,
there is a real chance that these types of influx will accomplish
the opposite effect (Brooks 1975).

The errors of managing software construction that Brooks describes
appear to be endemic. In his book {\em Dreaming in Code}, Scott
Rosenberg documents the impact of these boundaries of human process
on the development of Chandler, an open-source personal information
manager (PIM) backed by millionare code celebrity Mitch Kapor
(Rosenberg 2008). When announced in the fall of 2002, Kapor stated
that \quotation{optimistically} the project would reach a 1.0
release by the end of 2003, while a pessimistic projection would
place such a release in 2004 (83). At the time of
{\em Dreaming in Code}'s release in 2007, Chandler had yet to
achieve that milestone. Only in 2009 did the software finally
achieve this milestone.

\subsubject{Architectures of FLoSS Typesetting}

\subsubsubject{WYSIWYG}

Professional typesetting today is largely accomplished within the
proprietary program \type{InDesign} from design software superpower
Adobe, Inc. \type{InDesign} provides a WYSIWYG (What You See is
What You Get)[\letterhat{}1] interface that allows direct visual
manipulation of documents.

\subsubsubject{Semantic Systems}

\subsubsubject{Formal Systems}

[\letterhat{}1] : The acronym is generally pronounced
\quotation{wizzy-wig.}

\stoptext

