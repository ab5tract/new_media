%\enableregime[utf]  % use UTF-8

\setupcolors[state=start]
\setupinteraction[state=start, color=blue] % needed for hyperlinks

\usemodule[simplefonts]
\setmainfont[Liberation-Serif]
\setmonofont[inconsolata]

\setuppapersize[A4][A4]  % use letter paper
%\setuplayout[width=middle, backspace=1.5in, cutspace=1.5in,
%             height=middle, header=0.75in, footer=0.75in] % page layout
\setuppagenumbering[location={footer,right}]  % number pages
\setupbodyfont[12pt]  % 11pt font
\setupwhitespace[small]  % inter-paragraph spacing

\setupindenting[medium]
\indenting[always]

\setuphead[section][style=\tfc]
\setuphead[subsection][style=\tfb]
\setuphead[subsubsection][style=\bf]

% define descr (for definition lists)
\definedescription[descr][
  headstyle=bold,style=normal,align=left,location=hanging,
  width=broad,margin=1cm]

% prevent orphaned list intros
\setupitemize[autointro]

% define defaults for bulleted lists 
\setupitemize[1][symbol=1][indentnext=no]
\setupitemize[2][symbol=2][indentnext=no]
\setupitemize[3][symbol=3][indentnext=no]
\setupitemize[4][symbol=4][indentnext=no]

\setupthinrules[width=15em]  % width of horizontal rules

% for block quotations
\unprotect

\startvariables all
blockquote: blockquote
\stopvariables

\definedelimitedtext
[\v!blockquote][\v!quotation]

\setupdelimitedtext
[\v!blockquote]
[\c!left=,
\c!right=,
before={\blank[medium]},
after={\blank[medium]},
]

\protect

\starttext

\subject{Threads (Introduction)}

There are several threads that I perceive as necessary for the
construction of a proper report on the subject. The first is the
theoretical geneology of the medium. What claims have been made
about the {\em composition} of mediums? The seond thread involves
interrogating these claims in light of the nature of generative
processes. What place do discussions of mediums and
medium-specificty have in the realms of morphic generativity? The
third thread revolves around documenting the historical dimensions
of electronic typesetting. What currents flow through the
development of the typesetting tools found today in the world of
FLoSS?

The importance of this question lies in the mercurial context of
open source, and even computing in general, in which time is spent
on coding software rather than writing history. Constructing a
historical record is thus an attempt to contribute this relatively
invisible undercurrent to the scholarly record of new media in hope
of providing a template for further introspection in other fields
of software evolution.

The fourth thread involves direct integration of generative
processes in a typesetting workflow composed of FLoSS softwares.
Through this activity I hope to materially instantiate a zone for
discursive interplay between new media theory and real-world
typographic engineering. Does theory offer methods or tools useful
for interpreting the dynamics of typographic workflows, especially
as they involve generative processes? One obvious toolis a
vocabulary through which theory can attempt to describe and
critique the software's capacities and interactions. This thread of
the project aims to deliver exactly that.

\stoptext

