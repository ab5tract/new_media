%\enableregime[utf]  % use UTF-8

\setupcolors[state=start]
\setupinteraction[state=start, color=blue] % needed for hyperlinks

\usemodule[simplefonts]
\setmainfont[Liberation-Serif]
\setmonofont[inconsolata]

\setuppapersize[A4][A4]  % use letter paper
%\setuplayout[width=middle, backspace=1.5in, cutspace=1.5in,
%             height=middle, header=0.75in, footer=0.75in] % page layout
\setuppagenumbering[location={footer,right}]  % number pages
\setupbodyfont[12pt]  % 11pt font
\setupwhitespace[small]  % inter-paragraph spacing

\setupindenting[medium]
\indenting[always]

\setuphead[section][style=\tfc]
\setuphead[subsection][style=\tfb]
\setuphead[subsubsection][style=\bf]

% define descr (for definition lists)
\definedescription[descr][
  headstyle=bold,style=normal,align=left,location=hanging,
  width=broad,margin=1cm]

% prevent orphaned list intros
\setupitemize[autointro]

% define defaults for bulleted lists 
\setupitemize[1][symbol=1][indentnext=no]
\setupitemize[2][symbol=2][indentnext=no]
\setupitemize[3][symbol=3][indentnext=no]
\setupitemize[4][symbol=4][indentnext=no]

\setupthinrules[width=15em]  % width of horizontal rules

% for block quotations
\unprotect

\startvariables all
blockquote: blockquote
\stopvariables

\definedelimitedtext
[\v!blockquote][\v!quotation]

\setupdelimitedtext
[\v!blockquote]
[\c!left=,
\c!right=,
before={\blank[medium]},
after={\blank[medium]},
]

\protect

\starttext

\subject{Introduction}

Today's new media theory increasingly invokes {\em materiality} as
a significant, perhaps even {\em the} significant, mode of
investigating digital objects and the media through which they are
delivered. This thesis questions such a centrality of materiality
through a practice-based, process-oriented approach. {\em Process}
is proposed as the atomic unit of that which new media theory
investigates. A process-oriented approach will be shown to provide
superior methodologies for engaging with and understanding software
than material analysis alone provides. For instance, certain
problematics within Lev Manovich's concept of
\quote{media hybridity} will be resolved by a re-orientation
towards process (Manovich 2008). Process also allows a fresh
perspective for examining human-digital relations. Human processes
and digital processes are seen as inextricably intertwined, leaving
any discussion of digital process that excludes relevant dimensions
of human process necessarily unfinished.

The method proposed to demonstrate these points is two-fold. The
first is a mhistiographic approach, while the second is a detailed
interrogation of actual practice. The focus of both the
historiography and the practice is the topic of
{\em digital typesetting}. This topic was chosen for several
reasons. The first is a general lack of focus on the processes
behind typesetting among new media theory. The second is the
general lack of media hybridity of typesetting---according to
Manovich's definitions of the terms, typesetting has failed to move
beyond \quote{multimedia} to a state of \quote{media hybridity}
(this is opposed to typography, which undoubtedly has). This allows
us to not only interrogate this distinction but to question what
barriers may cause such a relative rigidity in the first place. The
third aspect is the allowance of a truly reflexive investigation in
which multiple processes of digital typesetting are utilized to
generate the thesis itself. This provides a means to integrate the
process-oriented perspective into a software study of FLoSS
typesetting software.

\subsubject{Screens}

As digital typesetting provides the focus for the application of
the process-oriented perspective, the point of origin is
necessarily that of the screen. Information transmission is
increasingly screen-based, a fact that only intensifies with the
exponentializing ubiquity of mobile devices such as the iPhone. The
long-awaited advent of cheap \quotation{tablet} computers and
e-readers is also now at hand. These devices may all be seen as
mediums for {\em screenic processes} in that their entire
configuration and all of its computation exists to serve as the
basis for screenic interactions with {\em human processes}. These
phrasings introduce the perceptual angle attendent with this
thesis, namely the centering of {\em process} as the atomic unit of
what is discussed in new media theory. The term {\em screenic}
simply means \quote{screen-based,} or perhaps
\quote{screen-native.} It is analagous to \quote{printed.}

One way to define screens is in terms of their interactivity. Some
screens, such as television screens, offer very limited
interactivity: the choice of content. This choice itself can be
constrained by varying degrees, such as the number of available
channels and playback formats (VHS, DVD, Xvid, etc.), even to the
point of disappearing (in the case of many televisions that appear
in public spaces).
\footnote{Mobile devices are beginning to ship IR transeivers with full
hardware access through software. That is, the
{\em entire potential} of the IR spectrum is available to them.}
The medium of the remote control should not be underestimated in
its effects on human processes, to say nothing of the screens at
which they are aimed. Indeed, they drive the interactivity of the
video game consoles, an interactivity that clearly represents the
cultural cutting edge of what a television screen can offer.

The computer screen, on the other hand, is defined by its seemingly
limitless degree of interactivity. Remote controls can be run as
screenic processes and can not only change television
channels---processes on remote systems can be controlled with
similar ease. Indeed, the entire screenic composition of one
computer can be controlled over a network by a second computer
using included, or easy to obtain, applications. Furthermore, the
very interfaces to the screen (keyboards and mice) are examples of
remote controls in cases where the screen has not itself become its
own remote control (touch-screen devices). Typically the only
element of a computer screen that the user does not effectively
control are the structure and visual language of an operating
system's graphical user interface (GUI). Even this, however, is
generally accomplishable by a significantly informed user. In the
case of GNU/Linux the task is not only accomplishable: in the case
of a \quotation{from scratch} installation,
\footnote{Such as is demanded by no-frills distributions such as Gentoo and
ArchLinux, where manual installation and configuration of a GUI is
required for use.}
the user is literally forced to make a choice of GUI structure and
visual styling. Microsoft has generally shipped their operating
systems with multiple choices for widget
\footnote{A widget is the technical term for a GUI element. Scrollbars,
titlebars, menus, and close/minimize/maximize buttons are widgets
attached to most of the \quotation{windows} that appears on any
given GUI-driven computer.}
presentation, including re-mediations of widgets from previous
versions of Windows. Users also developed Apple, however, maintains
strict control of widget presentation, especially on their mobile
devices.

\subsubsubject{Screens as material, screens as process}

Screens offer an ideal point of juxtaposition between the material
and processual frames. From a material view, the very formulation
of \quotation{screens} as {\em the} interface between humans and
computers is problematic: what of the interfaces that have been
developed to work around instances of blindness or other
[disabilities] that prohibit visually screenic interaction?

From a processual orientation, the question becomes: how do
interactions between humans and computers resolve themselves? The
answer returns in the form of the {\em available} remote controls
and the {\em available} response interfaces. The next step might be
to investigate the degree of variance between these availabilities,
and whether they problematize any umbrella-classification. While it
would be insensible to argue that material differences in inputs
and outputs can---or do---not lead to a huge amount of variation
between experiences within humans.

At this point the question becomes, then, whether it is necessary
to instantiate these inherent divergences in every evocation of a
broad level discussion of input and output mechanisms or whether
the inherent, {\em core} similarity between them all remains that
in all instances they serve as {\em the point of contact} between
human and digital processes. Does it make a processual difference
if the output technology is a braille screen or an LCD screen? Only
inasmuch as to what degree the process being examined is unique to,
or highlights differences between, one or the other. From a
discursive level, {\em controls} and {\em screens} can capture the
essence of these dual \quotation{action spaces} that together form
the single point of contact between human and digital process.

[Is it possible to remediate of the term screen into discussions of
previous mediums? For instance could one speak of the
\quotation{screen} of a newspaper or the \quotation{screen} of a
cave wall? What about the \quotation{screen} of a radio? From a
linguistic-conceptual perspective the final example certainly
pushes the limits. From a process perspective, though, the presence
of the radio/what it is playing/what listening choices are
available/how and to what degree does the hardware support
frequency tuning: these questions can all be conceived in terms of
\quote{control} and \quote{screen.} The sounds of a radio do emit,
after all, from the vibrations of a stretched membrane.]

\subsubject{From Screens to Text}

To discuss computer screens one must necessarily engage with the
concept of {\em interface}, a topic that rightfully occupies a
great deal of current new media discourse. Interface, then,
represents one point of departure from our origin. While interfaces
often utilize many visual metaphors (most of them inherited from
the work done developing the first GUI at Xerox's Palo Alto
Advanced Research Lab (PARC) in the 1970s), there are yet few
computer interfaces that do not rely on text as their dominant
mechanism for organizing and presenting a program's internal
capabilities to a user. (Mobile screens, on the other hand,
increasingly display developing trends of icon-only design, though
the web browser remains a popular application). Despite the success
of the GUI over the text-only command-line interface (CLI), text
remains central to contemporary experiences of computer screens.

The centrality of text to the experience of computer screens
represents the main avenue by which we proceed from the origin,
constituting a trunk from which many additional concerns fork away
and then face examination. The arguments of the paper are augmented
by the inclusion of a historiography of digital typesetting.
Engaging critically with the history of {\em software itself} is
proposed as a requisite for responsible software studies: a full
range of influences (economic, cultural, technological) should be
considered in the re-telling of a given processual unfolding.
Inspiring this enagement is the work of Robin Kinross, whose
{\em Modern typography: an essay in critical history} is one of but
a few texts covering a history of typography to adequately engage
with the influence of factors outside of the field on the field
itself (Kinross 2004). By integrating a critical history of digital
typesetting with a process perspective, an equilibrium between
human and digital processes will be illustrated.

Another importannt element of this thesis, one that runs throughout
the entirety of itself, is the underlying presentation of the text.
Through the utilization of FLoSS software, multiple output formats
will be not only be investigated but also materially instantiated.
These output formats represent the top three (open) interfaces
currently used to manage and display texts digitally: HTML, PDF and
OpenOffice.org's ODT (Do we need ODT? I'm thinking not. It can be
excluded as anathemic form of digital process---it's WYSIWYG
interface does not make it an optimal format for an investigation
in generativity).

The process(es) of their generation offers an attempt at mapping
Gilbert Simondon's language of ontogenesis onto file format
translation or, to begin the project immediately,
{\em individuation}. Coupled with Simondon's individuation is the
concept of {\em transduction}, which concerns itself with the
nature of change.

This relates with the increasingly generative nature of
contemporary design. All of which are generated from a plain-text
file whose syntax conforms to a format standard called
\quote{markdown.} The polycephalous nature of {\em the text itself}
thus demands further branching into a discussion of formats. What
are the attributes of the class of process to which formats belong?
Formats are seen as stable, yet they move like glass (or glaciar)
in the nano-magnitudes of the digital. Formats provide another
point of contrast between process and material perceptual
orientations.

The discussion of generativity provides further means to
demonstrate the equilibrium of human and digital processes.
Analyzed materially, these processes are chunks of code
electronically lifted from hard drive platters, loaded into system
memory, and then executed via the assemblage of chips on the
computer's motherboard by way of instructions from the operating
system currently residing as a mass of memory heaps in RAM chips.
Analyzed {\em processually}, however, these digital processes are
properly seen as deriving from interactions with human beings. That
is to say, digital and human processes are intimately intertwined,
from the design of their physical landscape of execution
(microcircuitry) to the instructions derived from the user. From a
process angle the computer becomes something of an external nervous
system, extending and modifying the realm of human potentiality
even as it surpasses the capacity of a single mind to functionally
comprehend the entirety of its workings.
\footnote{The chips produced by Intel, for example, are too complex for any
single person to ever hope to entirely understand (!CITE!).}

The design world is increasingly dominated by generativity.

\subsubject{Remote Controls, or: the Epitome of Cool Media?}

Is it possible to take the remote control and use it to create a
metaphor for all human-computer interaction?

Every digital process has, at its origin, a human. The rate of
computation has increased the impact of human-digital processes in
that the results deliver their results faster. The results will
either match the intentions of the originating human process, or
they will not. In the second case we can find the first evidence of
the effects of digital process on human process:
{\em the code behind the digital process will be re-arranged in an attempt to deliver an output that satisfies the intention of the human processes.}
Whether this modulation of the executed code is through
sliders/input boxes/etc within a GUI interface or through direct
reworking of the source code itself, the effect is the same: the
code executed has been re-configured according to the goal of human
process. The result(s) of the digital process, experienced through
a screen, can match, exceed, or fail this goal. In turn, human
process is effected and the next move is made according to new
goals or revised digital processes.

Video games, for example, can easily be represented by this model.
Human process is obviously shaped by digital at the outset: there
are a finite number of actions that a game offers within its
context. In addition, these actions are often presented as pre-set
mappings of action to controller button.

\subsubject{Recognizing the Ontogenesis in Generativity}

In his text {\em The Position of the Problem of Ontogenesis},
Simondon writes,

By transduction we mean an operation---physical, biological,
mental, social---by which an activity propagates itself from one
element to the next, within a given domain, and founds this
propagation on a structuration of the domain that is realized from
place to place: each area of the constituted structure serves as
the principle and the model for the next area, as a primer for its
constitution, to the extent that the modification expands
progresively at the same time as the structuring operation.
(Simondon 2009: 11)

The idea of transduction is tied to Simondon's solution for the
problematic of {\em individuation}, a long-standing issue in
discussions of ontogenesis.

Note the distinct lack of \quote{computational} in Simondon's list
of operations. Written prior to the advent of Manovich's
formulation of the age of cultural computing, this absence might
simply be read as a matter of temporal context. Nevertheless,
Simondon's solution to the ontogenesis problematic provides a
framework for describing digital processes of a generative nature.

\subsubject{Attributes of Process}

Process is reflective. It's outputs reflect its inputs.
Additionally, process reconfigures the metapotential in any given
system. It's reflectivity, then, has material effect. As it
reflects the inputs into the outputs, the outputs in turn reflect
new (or else simply different) potentials back into the landscape
which receives input.

A new configuration of metapotential in any system results in the
reconfiguration of (all) other systems as well. This fact reflects
the {\em fractal} nature of process---there is a degree of
non-reducibility inherent in any (description of?) process.

\subsubject{Screens as the Site of Digital-Human Process}

\stoptext

