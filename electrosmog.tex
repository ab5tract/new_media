

% startEnvironment

\setuppagenumbering[alternative={doublesided},location={footer,right,margin}]
\setupindenting[small]
\indenting[always]

\usemodule[simplefonts]
\setmainfont[Liberation-Serif]

% stopEnvironment

\starttext

%Title
\title{Building Digital Trust}
{\tfb A Proposed System of Positive Incentives \par} 
%


%todo/itemize
{\it John Haltiwanger} 
{\it Email: john.haltiwanger@gmail.com}
%todo/

% Start Document

\subject{Introduction}

Like drugs, peer to peer file sharing remains popular despite legal risks. If the lessons of the Drug War were learned, it would be obvious that legal/criminal regulation will not succeed in stopping this new counter(cultural|capitalist|protocological|productive) practice. Rather, sociality holds the key to a renewed respect for intellectual assets.\footnote{This is a hopefully less prejorative means of discussing what usually falls under "intellectual property." The shift in language is not just re-framing to make the concept more pallatable. Instead it hopes to liberate both licensing terms and the ideas/art/junk that they are bound to by opening a new language with which to approach the issue. Positioning contributions as assets affirms the relationship of that contribution to an intellect, to a body.} Through a system of positive incentives, trust networks are developed around digital objects. These trust networks can also develop around physical space/objects as well.

This concept applies to ElectroSmog because it has nothing less as its goal than the complete redefinition of intellectual asset distribution. Through positive incentives it is possible to allow assets to be reproduced and sold under license in a "trust-first" fashion that allows new producers to start producing and selling art simply by agreeing to terms laid out by the artist. Instead of shipping CDs and DVDs from central distribution centers all across the globe, for instance, artists can cut out all that overhead by simply allowing anyone who agrees to their terms. 

\subject[specs]{Nitty Gritty Implementation Details}

\startitemize[4,broad]
	\item QR Codes
\stopitiemize

\stoptext
