% thesis_proposal.tex
% Version: 0.0.1
%
% Input: John Haltiwanger
% Compiler: ConTeXt mkIV/LuaTeX
%

\starttext

For a thesis, I would like to propose a critical and operational engagement with the {\TeX} typesetting engine. 

\subject[critical]{Critical Component}

{\TeX} has existed as a typesetting engine for a little over three decades. As an ecosystem, {\TeX} is defined by its self-documentation. The source itself is open, but beyond that the source of individual documents are frequently provided. This allows a culture of technique diffusion and could qualify as a virtuous process as per Benkler. It also means the {\TeX} project could be considered a self-documenting electronic typesetting assemblage, which to my mind provides a unique opportunity for investigating the materiality of electronic type. Whereas an experienced typographer can determine the processes (or at least parameters) used to create output on a page, this capacity requires years of training. On the other hand, after mastering {\TeX} to a certain degree, it is possible to look at source documents and learn the exact parameters utilized to generate various outputs.

{\TeX} documents exist in multiple stages of materiality:

\startitemize[1,packed,broad]
	\item As a source document of marked up text. 
	\item As intermediately processed files that are generated during document compilation.
	\item As a processed output file (DVI, PS, PDF)
	\item As ink on a printed page.
	\item As pixels on a screen.
\stopitemize

(The operational engagement will incorporate an additional materiality: as a pre-{\TeX} source format, most likely reStructuredText. This functionality is provided by a translation layer, most likely pandoc, adding another degree of intrigue to {\TeX}'s materiality. Many other formats besides {\TeX} can be output through pandoc, giving a degree of format parity rarely held by {\TeX} documents.)


Though {\TeX} is powerful, it has also continued to evolve. Through macro packages such as {\LaTeX} and {\ConTeXt}, {\TeX} has become considerably easier to use. Furthermore, developments such as XeTeX and LuaTeX are pushing the envelope in terms of international support (Unicode, non-Western text formatting, etc). As {\ConTeXt} will be utilized in the operational component of the project, combined with the fact that it is under the most heavy and promising development at the moment, I expect that {\ConTeXt} and LuaTeX will be central to the critical engagement (perhaps even to the extent that it becomes a software study of {\ConTeXt} more than {\TeX}. The capacity of {\ConTeXt} to generate electronic documents, for instance, makes for another layer materiality: `hypertext.'

\subsubject[critical_refs]{Preliminary Reference List}

\startitemize
	\head Chun, Wendy Hui Kyong. (2008) {\em Control \& Freedom}. \hfill ()
	  Materialist framework for investigating the `light' of electronic typesetting.
	\head 
\stopitemize

\stoptext
