%\enableregime[utf]  % use UTF-8

\setupcolors[state=start]
\setupinteraction[state=start, color=blue] % needed for hyperlinks

\usemodule[simplefonts]
\setmainfont[Liberation-Serif]
\setmonofont[inconsolata]

\setuppapersize[A4][A4]  % use letter paper
%\setuplayout[width=middle, backspace=1.5in, cutspace=1.5in,
%             height=middle, header=0.75in, footer=0.75in] % page layout
\setuppagenumbering[location={footer,right}]  % number pages
\setupbodyfont[12pt]  % 11pt font
\setupwhitespace[small]  % inter-paragraph spacing

\setupindenting[medium]
\indenting[always]

\setuphead[section][style=\tfc]
\setuphead[subsection][style=\tfb]
\setuphead[subsubsection][style=\bf]

% define descr (for definition lists)
\definedescription[descr][
  headstyle=bold,style=normal,align=left,location=hanging,
  width=broad,margin=1cm]

% prevent orphaned list intros
\setupitemize[autointro]

% define defaults for bulleted lists 
\setupitemize[1][symbol=1][indentnext=no]
\setupitemize[2][symbol=2][indentnext=no]
\setupitemize[3][symbol=3][indentnext=no]
\setupitemize[4][symbol=4][indentnext=no]

\setupthinrules[width=15em]  % width of horizontal rules

% for block quotations
\unprotect

\startvariables all
blockquote: blockquote
\stopvariables

\definedelimitedtext
[\v!blockquote][\v!quotation]

\setupdelimitedtext
[\v!blockquote]
[\c!left=,
\c!right=,
before={\blank[medium]},
after={\blank[medium]},
]

\protect

\starttext

\subject{Introductory speakers}

\startitemize[n][stopper=.]
\item
  What are the possibilities and pitfalls in interactivity, art, and
  public space?
\item
  How to contribute to civility in these precarious times?
\stopitemize

\subject{Affect}

\subsubject{Kluitenberg}

\startitemize
\item
  video of Ligna, a radio-controlled collective street performance;
  flash mob
  \startitemize
  \item
    \quotation{I Am(not)sterdam}
  \stopitemize
\item
  overburdening of the senses
  \startitemize
  \item
    exemplary artist Raphael Barzamo Hammer
    \startitemize
    \item
      \quotation{In relation to the \quote{violence of the visual} that is taking over the public space: we don't want less images, we want {\em more} images}
    \stopitemize
  \stopitemize
\item
  informational environment versus cognition
\item
  The Third Body
  \startitemize
  \item
    emotional attachment to records; first time playing a record, a
    certain experience happens; when the record is put away, the
    experience is not fully present, just a memory; but when the record
    is listened again, the experience returns
  \item
    the experience is not attached to the listeners body, nor the
    record, but this \quotation{third} body
  \item
    the third body is not technologically determined
  \item
    digital technologies are technologies of complete andperfect
    articulation (especially in regards to control sequences, etc)
  \item
    something needs to be left out for digitization; what is missing is
    the precise thing that makes us feel that the digital object is
    anemic;
  \stopitemize
\item
  affect is the thing discarded by digitization
\item
  we need art that points beyond itself in a negative way
  \startitemize
  \item
    the denial of the system's rules, of its inherent methodologies
  \item
    the word processor (elle
  \item
    the bubblespace: intervention and rupture where the intervention
    itself points beyond the technology
  \item
    take scenarios where interactivity is truly and literally forced
  \item
    \quotation{break the frame} as well as \quotation{impose the frame}
  \stopitemize
\stopitemize

\subsubject{Round Table 1}

\startitemize
\item
  interpassive systems such as speed bumps: if the speed bumps are
  there as a result of lobbying from the concrete industry and not as
  a result of democracy, is it still interpassivity? and the tension
  between the cognitive and the body's desire for cigarettes? why
  should we listen to the cognitive over the body?
  \startitemize
  \item
    (Gijs): interactive and interpassive eras are not defined by a
    clean split: \quotation{tectonic plates} shifting together
  \stopitemize
\stopitemize

\subject{Interface}

\subsubject{Lovink}

\startitemize
\item
  Is the interace luring us away from the hidden powers of the new
  media or is it the real battlefield?
\item
  If we want to reach more audiences, one possible solution could be
  that we look more at the interface level. Should we put more money
  into new media arts programs and computer science? Or should we
  raise issues of aesthetics and politics of interfaces to highlight
  the underlying power of the systems?
\stopitemize

-

\subsubject{Galloway}

\startitemize
\item
  Powerful example of the interface in society today: the airport
\item
  \quotation{I wrote them mostly on airplanes,} Paul Baran on where
  he wrote his influential memos (ethernet/distributed networks)
  \startitemize
  \item
    jet-setting panache? foreshadowing of today's distributed office
    spaces?
  \item
    indication of a certain social mobility; the result of speaking
    engagements
  \stopitemize
\item
  the experience of communicated (physically) through the airport
  networks was particularly conducive to writing about communication
  networks themselves
\item
  its not simply the decentralized qualities of Baran's work, it is
  the distributed networks that are more radical
  \startitemize
  \item
    would have been more fittingly written in automobiles
  \stopitemize
\item
  interface: a facade, mostly useless and inconsequential. The desk
  has been outsourced to the passenger. Almost providing tech support
  to assist us in navigating the interface, rather than being
  representative agents of the airline.
\item
  After the check-in interface comes security. Searches, take off
  your hat, jacket, shoes, etc. (old techniques). But now data
  mining, computer vision techniques (faces, gestures monitored).
  Theatrical experience: certain people with certain roles asking
  certain questions, and then we have other actors responding.
\item
  After the security comes the shopping area. All manners of
  meandering pathways through well-lit store rooms. If security is a
  line, the shopping is a curve. A clean room, buffered by the
  security gate. International trade is physically represented.
  \quotation{Made clean for mass consumption.}
\item
  The gate: last interface.
\item
  Airport as a succession of interfaces.
\item
  \quotation{Interfaces are back, and perhaps they never left.}
\item
  Plato's concept of communication as writing words on the soul of
  another person.
\item
  The more these devices erase the evidence of their own functioning,
  the more effective they are. The more invisible they become.
\item
  \quotation{To succeed at an interface is at best self-deception, and at worse self-annihilation.}
  o
\item
  In some ways an interface is only an interface when it disappears
  from view.
\item
  Mediation as the irreducible disintegration of self and others into
  contradiction.
\item
  Interfaces are \quote{intuitive} or \quote{not intuitive}. However,
  looking at interfaces less as surfaces than as a doorway. The
  language of thresholds, of doorways,
\item
  Interfaces become important in the issues of cybernetics in that it
  is the site of discussion of where humans meet machine, or flesh
  meets metal. Or systems theory, where energy moves from one node to
  another in the system.
\item
  Interface and media might be two names for the same thing.
  \startitemize
  \item
    McLuhan: re-mediation, so-called natural laws of media; media are
    merely containers that encapsulate other pieces of media.
    \quotation{Onion} model of media. Media are \quote{scale-based},
    different layers of scale.
  \item
    Media themselves are interfaces: through the containment concept,
    it becomes the means by which the encapsulated media can be
    extracted from the layers
    \startitemize
    \item
      the point of friction, agitation between the different layers
    \stopitemize
  \stopitemize
\item
  An interface as
  \quotation{putting yourself at the beginning and the end} (a la
  ancient Greek poetry)
  \startitemize
  \item
    the outside \quote{possesses} the inside
  \item
    introductory liminal modes of expression
  \item
    \quotation{An area of , a fertile nexus.} The interface is special
    place that has its own autonomy, an area of choice.
  \stopitemize
\item
  the \quote{apostrophe}, addressing an object/character and breaking
  the
\item
  The \quote{text} and the \quote{paratext} (through or beyond text).
  The paratext is the edge, while the text is the center.
\item
  Diagetic versus non-diagetic.
\item
  \quotation{The interface as any artificial differentiation between two media.}
  \startitemize
  \item
    Any examination of the difference between the edge and the center
    leads to understanding that it is difficult to disceren where a
    center begins and an edge ends. Avant garde techniques are very
    interested in this tension.
  \item
    A web page is entirely artificial. Both are ASCII text but we have
    nevertheless created an artificial distrinction between the two.
    The source code of HTML is an interface.
  \stopitemize
\item
  We impose a linguistic construct to artiulate the distinction. An
  interface is not a \quotation{thing,} it is always an effect, a
  process, a mode of translation.
\item
  Norman Rockwell's triple self-portrait. A circulation of coherence
  that gestures toward the outside but ultimately remains afraid of
  the outside and withdraws from it.
  \startitemize
  \item
    The artist is operating in the white kind of nowhere space (visible
    in sci-fi through THX 1138 and The Matrix).
  \item
    On the canvas he appears with the soft wisdom of an elder.
  \item
    The image in the mirror is a \quote{technical} image.
  \item
    Fourth layer of the interface, the image itself, between the viewer
    and the image. Usually this interface is made invisible. Is this
    interface acknowledged or repressed?
    \startitemize
    \item
      The self-consciousness of the painting
    \stopitemize
  \item
    It claims to address concerns of interface within the image itself
    (diagetic space of the image)
  \item
    A diagetic circuit between the artist, the mirror, and the canvas.
    A circulation of intesnity, holding inward the energy. The image is
    a process, not a conglomeration of artistic details.
  \item
    everything is aimed at internalization
  \stopitemize
\item
  Mad magazine's Alfred
  \startitemize
  \item
    has no concern with making himself look better, just in making
    himself look clever
  \item
    the mode of address becomes the core concern
    \startitemize
    \item
      addressing the viewer in an intense way
    \stopitemize
  \item
    circular coherence of Rockwell is broken into three orthogonal
    spikes. Orthogonally outward (mirror), orthognally inward (almost
    at the mirror), and
  \item
    everything is aimed at externalization
  \item
    direct address to the viewer is \quotation{always} treated in a
    special way
    \startitemize
    \item
      narrative forms almost entirely eschews direct address
    \stopitemize
  \item
    \quote{short circuit}
  \stopitemize
\item
  With Rockwell we see an interface that addresses itself to the
  nterface itself. It answers the question of interface by repressing
  it. Alfred solves the question of the interface through a
  schizophrenia. It dwells on the pain of shattered coherence as a
  result of interface.
\item
  All interfaces are looking back at us, even when we become
  engrossed with them ourselves.
\stopitemize

\subsubject{Sommerer}

\startitemize
\item
  \quotation{Interfacing Reality}
  \startitemize
  \item
    from a Sony Research Labs project
  \stopitemize
\item
  Department of interface cultures
\item
  Interfaces in art
\item
  A language of cultural interfaces (from Manovich) ;
\item
  Paul Weibel traces the roots of the word \quotation{interface} back
  to \quotation{surgce science}---etymological and conceptual
  connection between \quotation{surface} and \quotation{interface.}
  \startitemize
  \item
    the representation of an interface becomes the interface (a map of
    the world becomes how we see the world)
  \stopitemize
\item
  social psychology -
\item
  interaction design in the creation of social meaning
\item
  interaction and participation in art.
  \startitemize
  \item
    social forms of participation since Dada-ism
  \stopitemize
\item
  yoko Ono cut piece
  \startitemize
  \item
    "Intelliget dialogue about gender relationship, voyerism and the
    artist-audience relationship
  \stopitemize
\item
  Examples of Cultural Interfaces
  \startitemize
  \item
    Stephen Johnson:
    \quotation{Interfaces will intrude into our lives and it is important for artists and designers to respond to that.}
  \item
    Tangible Interface: InfoTable
  \item
    News Machine
  \item
    Mountain Guitar --- played with body movements
  \item
    Intelligen homes
    \startitemize
    \item
      The living room: listening to conversation and looking up keywords
    \stopitemize
  \stopitemize
\item
  New interfaces are not generally developed for commercial ends.
  Avoid pure art or pure commercial applicatins
\stopitemize

\subsubject{van Thije}

\startitemize
\item
  the notion of interface and how it relates to art, especially in
  the sense of museum.
\item
  \quotation{Systems work because they don't work.}
  \startitemize
  \item
    If a relation remains, it is because the connection has failed.
  \stopitemize
\item
  one can act within the realm of an interface or one could engage
  within the interface
  \startitemize
  \item
    play with the limits vs using already understood rules
  \stopitemize
\item
  an artwork can be an action upon the interface or a moment of
  density within a system
  \startitemize
  \item
    how does a musuem facilitate this?
  \item
    19th century museums did not display art on the wall in an
    organized gallery style but rather in an all-at-once manner
    \startitemize
    \item
      a direct interface to the art \quote{depot}
    \stopitemize
  \item
    the type of knowledge production in a culture is displaed in the
    interface of museums
  \item
    the development in the early twentieth century of museums towards
    an exhibition space (a format in itself)
  \item
    19th century museum was interface pointing to the universal, in the
    20th it was towards an individual or even \quote{dividual}
    \startitemize
    \item
      exhibition interface was designed to disappear the body of the
      audience
    \item
      within the white cube it is just an exercise
    \stopitemize
  \stopitemize
\item
  how can the interface be approached in these two different ways?
\item
  what is currently happening in the museum space?
  \startitemize
  \item
    if the art can no longer function in the Rockwell sense
    (self-contained, internal focus) and instead focuses on the edges,
    how do we display it?
  \item
    less and less artwork just objects to collect but rather
    installations; art can no longer just be collected and put into a
    depot; when the artwork is a complex interface it allows for
    contemplation of how to share and/or display this type of art
  \stopitemize
\stopitemize

\subsubject{Round Table 2}

Interfaces and Affect - \quote{affective computing}: developing
interfaces that engage with feelings

\subsubsubject{Black boxes vs White cubes}

AG: We seem to need both these days?

SvT: The exchange between

: The black box entering the white cube is a great metaphor. Hybrid
artforms are entering the white cube in greater and greater scales.
Everyone can be artistic these days, and so the technology is at
once dissolving the distinction and uniqueness of art in the white
cube.

\subject{Object}

\subsubject{Round Table 3}

\startitemize
\item
  {\em (Summary of Willem van Weelden's absent lecture)}:
  Essentialism in Latour: hidden in ANT is an endless opening of
  black boxes. Willem van Weelden calls this a
  \quotation{road to hell.} There is less and less margin for space
  for contentious objects.
\stopitemize

\subsubsubject{Object Research Lab}

\startitemize
\item
  Integrating different disciplines (material engineering,
  philosophy) in day long workshops. This results in translations
  between the disciplines/discourses.
\item
  What is the Swiss Army going to be lke in the 20th century? None of
  the individual tools on a swiss army knife are optimal.
  \startitemize
  \item
    The morality of the knife: only the officer knives had a corkscrew,
    enlishted knives had bottle cap openers.
  \stopitemize
\item
  \quotation{Relational thing-ness}
  \startitemize
  \item
    A thing is not defined, a thing always exists in relation to other
    things.
  \item
    An object can never be wrong, it can be mistreated.
  \item
    An object can never be right, it can be well treated.
  \stopitemize
\item
  Where do we start talking about the objects? At the molecular
  level? At the tactile level?
\item
  You can't have a good bicycle without a good road.
\item
  Objects that exist specifically to inspire
  \quotation{thoughts about things.}
  \startitemize
  \item
    Designed in a single color to reduce distraction and focus on the
    essence.
  \stopitemize
\stopitemize

\subsubject{}

\startitemize
\item
  Theory is going rampant right now. The future belongs to practice.
\item
  Every object stands as a relation between (at least) two human
  beings.
\stopitemize

\stoptext

