% Links
\setupcolors[state=start]
\setupinteraction[state=start,color=darkblue]

% ..
\setuppagenumbering[location={footer,left,margin}]  

% paper, layout, etc.
\setuppapersize[A4]
%\setuplayout[width=6.5in,height=10.5in,topspace=0.5in,backspace=1in,
%  header=0.5in,footer=0.5in]
% double space
% \setupwhitespace[small1]
% 1/2 inch indents
\setupindenting[medium]
\indenting[always]
% CSL is a new form of handling bibliographic data and citations. It also apparently   http://xbiblio.sourceforge.net/csl/
% jagged-right (aligned to the left)
% \setupalign[right]
% font
%\setupbodyfont[rm,12pt]
\usemodule[simplefonts]
\setmainfont[Liberation-Serif]
\setmonofont[inconsolata, 11pt]

% for long quotes
\definestartstop[longquote][
  before={\indenting[never]
    \setupnarrower[left=0.5in,right=0.5in]
    \startnarrower[2*left,right]},
  after={\stopnarrower
    \indenting[yes]}]

% for heading and header
\def\MLA#1[#2][#3][#4][#5][#6][#7]%
	{\setuppagenumbering[left=#3 ,location={header,right}]
	\indenting[never]
	#2 #3\par#4\par#5\par#6\par\startalignment[middle]#7\stopalignment
	\indenting[yes]}
	
% following hanging indent code (also in workscited) taken from 
%  http://www.ntg.nl/pipermail/ntg-context/2004/005280.html
% [NTG-context] Re: Again: "hanging" for a lot of paragraphs?
%  ~ Patrick Gundlach
\def\hangover{\hangafter=1\hangindent=0.5in}
\definestartstop[workscited][
  before={
    \page[no]
    \indenting[never]
    \startalignment[left]
    \subject{Bibliography}
    \stopalignment
    \setupwhitespace[medium]
    \bgroup\appendtoks\hangover\to\everypar
    },
  after={
    \egroup
    \indenting[yes]}]


\starttext      

\starttabulate[|l|]
\NC {\tfd Flattening Screens}
\NR
\NC {\tfb Investigating the Reduction of Opinion Diversity in the Blogosphere}
\stoptabulate

\starttabulate[|l|l|]
\NC Student:
\NC John Haltiwanger
\NR
\NC Paper:
\NC Politics paper (very) rough draft
\NR
\NC Supervisor:
\NC Thomas Poell
\NR
\NC Date:
\NC 16 March 2010
\NR 
\stoptabulate  


\subject{{\em Preface: Caveats}}

{\em The following is a paper which seeks to investigate the particularities of an event whose central theme is race. It therefore seems imperative to begin by stating that this project approaches from the perspective of new media theory. Furthermore, it is written by a white male. While I consider myself antiracist, these two factors may result in \quote{blind spots} in this analysis. Please consider any absence of relevant theory to be not the result of deliberate snubbing but rather the limitations imposed by my relative ignorance. This last feature of myself is something I'm always seeking to reform, so please alert me if there is something crucial missing at \useURL[1][mailto:john.haltiwanger@gmail.com][][john.haltiwanger@gmail.com]\from[1].}

{\em There is an additional caveat---I would like to weave this as a creative nonfiction piece, rather than as a dry academic liturgy. Be aware that narrative voice can and will switch.}

\subject{Introductory Note on Racism in America}

To say that the shadow of racism looms large over the American
psyche is at once both an over- and an under-statement. It is an
understatement in that for many, many people, race is a defining
element of their personal experience---in this way, the statement
does not convey the real impact of race. It is an overstatement
because a large proportion of people also fail to recognize race as
a significant issue---for them, racism is not a shadow but a lie to
begin with, at least in contemporary terms. This discrepancy is
played out daily on the Internet, television, radio, and in print.
The situation has only become more tense with the advent of our
first non-white president. The group that believes in the advent of
a \quote{post-racial} society uses the election of President Obama
to assert that racism is no longer an issue---society has achieved
equality and thus any calls of \quote{racism} are themselves racist
(this phenomenon has been called \quote{reverse racism}).

This is especially interesting when contrasted with Wendy Hui Kyong
Chun's {\em Control and Freedom}, which documents the manner in
which \quote{post-race} visions of the Internet necessarily shift
the blame of discrimination from the discriminator to the
discriminated. In both cases post-racial rhetoric is used to
discredit accusations of racism and to redirect the blame from the
accused to the accuser. While the Internet (\quotation{cyberspace})
is the realm of a post-racial utopia in the narrative which Chun
critiques, the election of President Obama has brought that same
narrative into discourses about the real world.

The binary logic of reverse racism
\footnote{Expressed in psuedocode:
\overstrikes{\overstrikes{\overstrikes{\overstrikes{\overstrikes{\lettertilde{} if (society.post-racial? \&\& accusation.racism?) then accuser == racist accused == absolved end}}}}}\lettertilde{}}
is predicated on this claim of a \quote{post-racial} society. It
appears almost universally in American discussions of race. The
problem with it is that equality has {\em not} been achieved: the
post-racial society does not exist. At this point it is simply a
clever rhetorical tool by which accusations of racism can be
quickly diffused. The two most effective rhetorical tools to
respond to this from an antiracist perspective are the concepts of
{\em privilege} and {\em structural discrimination}. To demonstrate
the inevitability of reverse racism in online discussions, I've
chosen a well-written comment from a recent boingboing
\footnote{If the selection seems random, remember that I am attempting to
demonstrate ubiquity here. All one needs to do is click on a thread
about race, and reverse racism appears. Then, hopefully, someone
else enters to describe the phenomena of privilege and structural
discrimination. As this one is quite lucid, I've chosen it to
explain the terms. The endless rephrasings of essential concepts
(that somehow no one still seems to know) is both unnecessary and
tiring.}
post promoting a satirical book called
{\em Obamistan! Land Without Racism} in which user CantEvenGo
claims:

\startlongquote

annnnd....here we go. Hello, BoingBoing commenters! I'd like to introduce you to a couple of interlocking concepts: structural discrimination and white privilege.

structural discrimination (often publicly articulated as racism) is a system of practices and beliefs that combine to leverage the resources (material, financial, spiritual, ideological) of one group over another. While racism is the most visible of these practices, there are many other implicit behaviors and tacit beliefs that work to privilege (wait for it...) whiteness in a wide variety of institutional settings. These range from disparate sentencing rates, to beliefs about educational attainment, to redlining. Whether your parents owned slaves or not, if you're white then chances are good that you have benefitted (albeit unknowingly) from longstanding structural discrimination patterns.

Which leads us to Privilege (here we go). I often argue that privilege is the obverse of racism: where racism is the police baton to the back of the head, privilege is the belief that police are always friendly and here to help you. Many whites enjoy privilege that is unearned; that is, there is no other rationale BESIDES whiteness that entitles you to a better mortgage rate than a person of color with the exact same income, credit score, debt-to-income ratio, and assets.

IMHO, understanding structural discrimination and white privilege will enable many of you to relinquish your individualist, color-blind, liberal, and ahistorical (and ultimately, racist) beliefs...and coincidentally keep you from putting your foot in your mouth when talking about stuff you don't understand.

\stoplongquote

The linguistic switch from \quote{racism} to
\quote{structural discrimination} is effective because no one has
ever claimed that American society has become
\quote{post-structural discrimination.} This reduces the impact of
attempts to reverse responsibility for the causes of grievances.
Aiding in this is the presence of \quotation{structure} in the
verbalization of the concept---if discrimination is enabled by a
specific structure, then it becomes harder (though not impossible)
to accuse the discriminated of bringing it on themselves. The
concept of privilege, however, is crucial for understanding not
only structural discrimination, but the very means through which
the arguments of a \quote{post-racial} society is articulated,
promoted, and cited in reverse racism.

\subject{Introductory Note on Race and the Internet}

\letterunderscore{}[Having trouble finding the theorist who
explicitly mentions the importance of \quote{safe spaces} fo]

\subject{Citizen Journalism}

Axel Bruns identifies an cooperative, rather than the oft-touted
antagonistic, interplay between the mainstream media and the realm
of \quote{citizen journalism} constituted in large part by the
blogosphere but encapsulating non-blog projects such as Indymedia.
The former has traditionally held the power of \quote{gatekeepers.}
That is, the editorial decisions of the mainstream media have
historically defined the degree to which a story was given voice.
Certain framings could be ensured, as the opportunity for critique
(on a scale of mass distribution, from non-specialists unlikely to
be interviewed on TV) lay in the reader comments section of a
newspaper. This section is obviously also \quote{gatekept,} leaving
the editors to pick and choose what critique to entertain with
printing space. The limitations presented by deadlines,
profitability, and, above all, space tend to shape the stories
reported on into \quote{single packet} stories that can be
characterized by their reduction of complexities into a short
narrative that carries few, if any, of these complexities across.

The Internet has changed all that. The medium that has enabled
\quote{citizen journalism} through its capacity for many-to-many
communications. Capital investment in launching a new publishing
platform can be as low as zero, allowing anyone to contribute to
the online discourse in any area that may be of interest to them.
This enables a new mode that Bruns labels
\quote{gatewatching}---analyzing the products released by the
gatekeepers and attempting to hold them accountable to their
editorial decisions. This is a whole new form of muckraking, and to
Bruns it represents an integral piece of a healthy news ecosystem.

Organizationally, citizen journalism is heterachical rather than
hierarchical, concerned not with producing discrete, individual
(one could say commodified) versions of events but rather
collaboratively digesting the implications and context of those
events. This engaging with
{\em unfinished artefacts, continuing process} signals citizen
journalism's classification by Bruns as a form \quote{produsage,} a
term he positions in opposition to production as a new form of
productive expression typified by the increasing presence of
\quote{the people formerly known as the audience} in content
creation. The lines have blurred between the old gatekeepers of
industrial news, still a crucial component in citizen journalism
besides the antipathy between the two, and the citizen journalists
in terms of who is trusted to deliver the news. Bruns notes that
the two each have important roles to engage with: the former should
produce factually accurate, neutral reportage while the latter
continues its role as the collective digestive system for social
event processing (92).

Unconstrained by space requirements and the format of the
supposedly definitive \quote{single packet} news story, not to
mention political and commercial elements unique to industrial news
production, citizen journalism allows the flowering of debate
around topics that are unpallateable to the old regime. The
proscribed role of citizen journalism is gatewatching, that is,
observing the misrepresentations and omissions in the industrial
newsfeed, though Bruns notes a tendency towards original reporting
and new \quotation{Pro-Am} hybrid news organizations. This is done
through a process of aggregating personally interesting information
and sharing it (and usually an opinion) with others whose responses
are an integral part of assessing that information. No one is
assumed to have all the knowledge, freeing individuals from a
responsibility to speak from anything besides personal experience.
Single \quotation{articles} within the unfolding of the discussion
are nothing more than the means of furthering discussion by
offering an individual perspective that others may or may not agree
with, may or may not have something imperative to contribute to
that perspective. In this way citizen journalism can be seen as
probabalistic---it strives for quality through the collectivization
of opinion, context, and insight. Not all is worthy, but the worthy
will often arise from the cacaphony. Methods within individual
sites such as karma points are a means for filtering up the worthy,
but even the anarchic relations of the blogosphere reveal the
importance of {\em continued} quality in maintaining the reputation
of a blog.

That citizen journalism represents a new mode of news
making/digestion that gives especially welcome space to voices and
opinions neglected by mainstream media is exemplified in the site
Racialicious
(\useURL[2][http://racialicious.com/][][racialicous.com]\from[2]).
The site hosts a continuous discussion of pop culture from an
anti-racist perspective, gatewatching the media's (mis)portrayal of
race from a strongly moral and critical perspective.

\subject{The Case Study}

In January 2010 the editors of Essence magazine chose a picture of
Reggie Bush, a black football player in a multiracial relationship,
to put on the cover of its
\quotation{Black Men, Love \& Relationships} issue. This set off a
firestorm of discussion that finally \quotation{ended} with a sigh
of relief that the corresponding explosion of debate within the
black blogosphere was not picked up and reported by the mainstream
media (Peterson 2010). Essence is a magazine that markets
specifically to the black female community, sparking a conversation
about the appropriateness of highlighting a multiracial
relationship on the cover of an issue about \quotation{black love.}
Explaining the blog's posting about the issue, Latoya Peterson
states that
\quotation{talking about \quote{the situation within the black female community} isn't really what we do since most of those perceptions are based in stereotypes about black women. However, what is compelling about the whole situation is how conversations about interracial dating play upon stereotypes and deeply held convictions, that tend to drown out any other type of commentary.}

In other words, it is precisely the dynamic, social process itself
that constitutes the news that Racialicious is reporting on, or at
least one instantiation of that process. The other piece of the
news reported is, intriguingly, that the debate around the cover
didn't become news in the industrial media. This is precisely
because the \quote{single-packet} format would inevitable compress,
through institutionalized race and gender stereotypes, into a story
about how \quotation{black women hate interracial dating!}
Precisely because the debate was not subjected to such formative
reduction, it is allowed to become nothing more (or, crucially,
less) than the collective expression of itself. Racialicious
performs an honorable, and perhaps all-too-absent, service by
presenting an overview of the very diverse reactions of commenters
on various pages relating to the issue, thus chronicling the
unfolding of collaborative opinion sharing. No less than twelve
distinct viewpoints receive re-articulation, demonstrating the
destructive capacity of the \quote{single packet} format in the
context of complex social situations.

This project aims to investigate the case in light of Latour's
Actor-Network Theory (ANT). In this space we have actors such as
the editors of Essence, the authors of blogs on the topic, and the
blog software that enables the interconnection of the various
threads expressed in different posts and comments sections.
Intriguingly, the mainstream media itself may appear in this
network as a mediator by the very virtue that their
{\em lack of action} had an important influence on the unfolding of
the discussion.

\stoptext