%\enableregime[utf]  % use UTF-8

\setupcolors[state=start]
\setupinteraction[state=start, color=blue] % needed for hyperlinks

\usemodule[simplefonts]
\setmainfont[Liberation-Serif]
\setmonofont[inconsolata]

\setuppapersize[A4][A4]  % use letter paper
%\setuplayout[width=middle, backspace=1.5in, cutspace=1.5in,
%             height=middle, header=0.75in, footer=0.75in] % page layout
\setuppagenumbering[location={footer,right}]  % number pages
\setupbodyfont[12pt]  % 11pt font
\setupwhitespace[small]  % inter-paragraph spacing

\setupindenting[medium]
\indenting[always]

\setuphead[section][style=\tfc]
\setuphead[subsection][style=\tfb]
\setuphead[subsubsection][style=\bf]

% define descr (for definition lists)
\definedescription[descr][
  headstyle=bold,style=normal,align=left,location=hanging,
  width=broad,margin=1cm]

% prevent orphaned list intros
\setupitemize[autointro]

% define defaults for bulleted lists 
\setupitemize[1][symbol=1][indentnext=no]
\setupitemize[2][symbol=2][indentnext=no]
\setupitemize[3][symbol=3][indentnext=no]
\setupitemize[4][symbol=4][indentnext=no]

\setupthinrules[width=15em]  % width of horizontal rules

% for block quotations
\unprotect

\startvariables all
blockquote: blockquote
\stopvariables

\definedelimitedtext
[\v!blockquote][\v!quotation]

\setupdelimitedtext
[\v!blockquote]
[\c!left=,
\c!right=,
before={\blank[medium]},
after={\blank[medium]},
]

\protect

\starttext

\startitemize
\item
  1992 established a new platform whereby racial injustices and
  inequalities disappeared from the political sales pitch of Bill
  Clinton
  \startitemize
  \item
    \quotation{This universalizing discoursed proved extremely popular, as it allowed avoidance of all discussion of race in favor of concerns that were perceived as more \quote{universalist,} such as funding to support causes dear to suburban dwellers.}
    (2)
  \item
    \quotation{This historical moment intersected with he incetion of the Internet as a mass technology in the United States, It s in this moment that the neoliberal discourse of color blndness would become linked with the Clinton-Gore administratino's identification of the Internet as a privileged aspect of the national political economy.}
    (3)
  \stopitemize
\item
  \quotation{Vijay Prashad identifies this gentler form of racism as the greatest problem of the twenty-first century---the}color-blind"
  replaces the color line as the prevailing practice that permits
  resources to be unevenly allocated based on racial identities." (3)
\item
  \quotation{The language of tolerance, or of disavowing racism by simply oimmitting all language referring to race, runctioned to perpetuate digital inequality by both concrete and symbolic means.}
  (3)
\item
  color blindness stems from the Cold War; Asian Americans held up as
  an example of an \quotation{ethnic} (versus biological) vision of
  race
  \startitemize
  \item
    \quotation{cultural differences replacedracial ones as the salient aspects of identity}
    (3)
  \stopitemize
\item
  \quotation{However, when we look to the post--2000 graphical popular Internet, this utopian story of the Internet's beginnings in popular culture [liberation movements of the 60s] can be told with a different spin, one that instead tracks its continuing discourse of color blindness in terms of access, user experience, and content that is reflected in the scholarship as well as in nineties neoliberalism's emphasis on \quote{moderate redistribution and cultural universalism}.}
  (4--5)
\item
  \quotation{While the policy rhetoric around Internet access may have been inflected strongly with the neoliberal discourse of clor blindness and nondiscrimination---a paradigm in which failure to overtly discriminate on the basis of race, and the freeedom to compete in the \quote{open market} despite an uneven playing field in terms of class, education, and cultural orientation constitutes fairness---the Internet has continued to gain uses and users who unevenly visualize race and gender in online environmnets.}
  (5)
\item
  Gajjala {\em Cyber Selves}; Mitra, \quotation{Virtual Commonality}
\item
  \quotation{Digital racial formation can trace the ways that race is formed online using visual images as part of the currency of communication and dialogue between users. Performing close readings of digital visual images on the Intenet and their relatino to identity, itself now an effect as well as a cause of digitality, produces a kind of critiue that takes account of a visual practice that is quickly displacing television as a media-based activity in the United States.}
  (11)
\item
  Nakamura, {\em Cybertypes}
\item
  \ldots{}\quotation{in this work I locate the Internet as a privileged and extremely rich site for the creation and distribution of hegemonic and counterhegemonic visual images of racialized bodies.}
  (13)
\item
  \quotation{What has yet to be explored are the ways that race and gender permit differential access to digital visual capital, as well as the distinctive means by which people of colow and women create and in some sense redefine it. Women and people of color are both subjects and objects of interactivity; they participate in deigital racial formation via acts of technological appropriation, yet are subjected to it as well.}
  (16)
\item
  rather than focusing on getting women and minorities online,
  asking:
  \quotation{How do they use their digital visual visual capital? In what ways are their gendered and racialized bodies a form of this new type of capital? What sort of laws does this currency operate under? It doesn't change everything, but what does it change? This brings us back to the privileging of interactivity and its traditional linkage with the creation of a newly empowered subject.}
  (16)
\item
  \ldots{}\quotation{we now view the interface as an object that compels particular sorts of identifications, investments, ideological seductions, and conscious as well as unconscious exercises of power.}
  (17)
\item
  \quotation{Instead the interface itself becomes a star, and just like other sorts of stars, it works to compel racialized identifications; interfaces are prime oci for digital racial formation.}
  (17)
\item
  racial formation: different than structral race theory?
  \startitemize
  \item
    Omi and Winant, \quotation{Racial formation}
  \stopitemize
\item
  \quotation{to click on a box or link is to acquire it, to choose it, to replace one set of images with another in a friction-free transaction that seems to cost nothing yet generates capital in the form of digitally racialized images and performances}
  (17)
\item
  \quotation{The difference [in terms of racial formation] between old and new media lies in the new media's interactivity, as mentioned before, but is also related to the blurred line between producers and consumers.}
  (18)
\stopitemize

-

\stoptext

