% Links
\setupcolors[state=start]
\setupinteraction[state=start,color=darkblue]

% ..
\setuppagenumbering[location={footer,left,margin}]  

% paper, layout, etc.
\setuppapersize[A4]
%\setuplayout[width=6.5in,height=10.5in,topspace=0.5in,backspace=1in,
%  header=0.5in,footer=0.5in]
% double space
%\setupinterlinespace[line=5.6ex]
% 1/2 inch indents
\setupindenting[medium]
\indenting[always]
% CSL is a new form of handling bibliographic data and citations. It also apparently   http://xbiblio.sourceforge.net/csl/
% jagged-right (aligned to the left)
\setupalign[right]
% font
%\setupbodyfont[rm,12pt]
\usemodule[simplefonts]
\setmainfont[Liberation-Serif]
\setmonofont[inconsolata]

% for long quotes
\definestartstop[longquote][
  before={\indenting[never]
    \setupnarrower[left=0.5in,right=0.5in]
    \startnarrower[2*left,right]},
  after={\stopnarrower
    \indenting[yes]}]

% for heading and header
\def\MLA#1[#2][#3][#4][#5][#6][#7]%
	{\setuppagenumbering[left=#3 ,location={header,right}]
	\indenting[never]
	#2 #3\par#4\par#5\par#6\par\startalignment[middle]#7\stopalignment
	\indenting[yes]}
	
% following hanging indent code (also in workscited) taken from 
%  http://www.ntg.nl/pipermail/ntg-context/2004/005280.html
% [NTG-context] Re: Again: "hanging" for a lot of paragraphs?
%  ~ Patrick Gundlach
\def\hangover{\hangafter=1\hangindent=0.5in}
\definestartstop[workscited][
  before={
    \page[no]
    \indenting[never]
    \startalignment[left]
    \subject{Bibliography}
    \stopalignment
    \bgroup\appendtoks\hangover\to\everypar
    },
  after={
    \egroup
    \indenting[yes]}]


\starttext      



\subject{Information}




\footnote{The views expressed by Pasquinelli here converge with other recent
readings that have significantly changed the way I would
theoretically treat the example of
\useURL[1][http://github.com][][Github]\from[1], a proprietary Web
2.0 application running on Free Software, than I did in my Masters
of Media post
\useURL[2][http://mastersofmedia.hum.uva.nl/2009/11/01/git-virtue-github-and-commons-based-peer-production/][][Git Virtue?: Github and Commons-based Peer Production]\from[2].
While it may still spread virtue in a Benkler-ian sense, its
parasitical nature needs further articulation. The controversial
aspects of hosting FLOSS projects on a proprietary platform are
much starker in light of Pasquinelli's presentation.}

Pasquinelli scoffs at the paradox behind Lessig's proposal of a new
tax to compensate artists, making me curious about how he feels
about the proposals for a universal livable wage (not tied to
productivity) seen in the work of Michel Bauwens and Christoph
Spehr. The Creative Commons license fails to consider the frictions
that digital copies create in real spaces governed by limited
resources. Not only this, CC fails to follow a key line of
reasoning behind Free Software licenses: rather than granting
rights to users, rights are reserved for copyright holders. Worse
still is the naivete that comes with ignoring the issue of
surplus-value extraction, as evidenced by Free Software's easy
co-option into proprietary business structures and technological
infrastructures alike. The answer may lie in a new form of
licensing, proposed by Dmytri Kleiner, in which class
considerations are taken into account, ie worker co-ops can have
access to licensed works for free but for profit enterprises would
be forced to pay.


\subject{Bibliography}

\startworkscited


Pasquinelli, Matteo.
\quotation{The Ideology of Free Culture and the Grammar of Sabotage.}
{\em Education in the Creative} Economy: Knowledge and Learning in
the Age of Innovation\letterunderscore{}. Eds. D. Araya, and
M.A.~Peters. New York: Peter Lang, 2010.

\stopworkscited

\stoptext

