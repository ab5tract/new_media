% Links
\setupcolors[state=start]
\setupinteraction[state=start,color=darkblue]

% ..
\setuppagenumbering[location={footer,left,margin}]  

% paper, layout, etc.
\setuppapersize[A4]
%\setuplayout[width=6.5in,height=10.5in,topspace=0.5in,backspace=1in,
%  header=0.5in,footer=0.5in]
% double space
%\setupinterlinespace[line=5.6ex]
% 1/2 inch indents
\setupindenting[medium]
\indenting[always]
% CSL is a new form of handling bibliographic data and citations. It also apparently   http://xbiblio.sourceforge.net/csl/
% jagged-right (aligned to the left)
\setupalign[right]
% font
%\setupbodyfont[rm,12pt]
\usemodule[simplefonts]
\setmainfont[Liberation-Serif]
\setmonofont[inconsolata]

% for long quotes
\definestartstop[longquote][
  before={\indenting[never]
    \setupnarrower[left=0.5in,right=0.5in]
    \startnarrower[2*left,right]},
  after={\stopnarrower
    \indenting[yes]}]

% for heading and header
\def\MLA#1[#2][#3][#4][#5][#6][#7]%
	{\setuppagenumbering[left=#3 ,location={header,right}]
	\indenting[never]
	#2 #3\par#4\par#5\par#6\par\startalignment[middle]#7\stopalignment
	\indenting[yes]}
	
% following hanging indent code (also in workscited) taken from 
%  http://www.ntg.nl/pipermail/ntg-context/2004/005280.html
% [NTG-context] Re: Again: "hanging" for a lot of paragraphs?
%  ~ Patrick Gundlach
\def\hangover{\hangafter=1\hangindent=0.5in}
\definestartstop[workscited][
  before={
    \page[no]
    \indenting[never]
    \startalignment[left]
    \subject{Bibliography}
    \stopalignment
    \setupwhitespace[medium]
    \bgroup\appendtoks\hangover\to\everypar
    },
  after={
    \egroup
    \indenting[yes]}]


\starttext      

{\tfd Jokers }

\starttabulate[|l|l|]
\NC Student:
\NC John Haltiwanger
\NR
\NC Paper:
\NC Week 6
\NR
\NC Supervisor:
\NC Thomas Poell
\NR
\NC Date:
\NC 10 March 2010
\NR 
\stoptabulate  

\subject{Gunkel}

David J. Gunkel centers his
\quote{Second thoughts: toward a critique of the digital divide} on
the ambiguous, yet ubiquitous, nature of the term
\quote{digital divide.} In the terms of Latour, he sees the term as
a matter of contention. His reaction is to push for a broadened
understanding of the plural nature of the term. Tracing the origins
of \quote{digital divide} through the claims made by one of its
main promoters (the US Department of Commerce's National
Telecommunications and Information Administration, or NTIA) as to
where they found it, they come across journalism discussing a
division between those who held faith in technology to solve the
world's problems and those that had misgivings about having such
faith. Even at this early stage there is a binary modulation on a
much more complicated scenario---for instance, those that have
faith in technology to have the {\em potential} to solve problems
might not have deep reservations about answering in affirmative if
that potential was phrased in technodetermistic terms in a survey
question. As the term continues its unfolding into its present
state as a household term, it retains these binary and
technodetermistic attributes and gains a plurality of definitions.
From hardware clashes between digital and analog devices coexisting
in information networks to being situated as analagous to a
\quote{racial ravine} where statistics show correlations to race in
terms of access to devices and knowledge, the digital divide is
multifaceted. In Gunkel's view, in order to continue using the term
it should be interrogated for what it has claimed to cover, basing
this on the recognition that the phenomenons it has been used to
describe all fit under the discussions it has been used to
describe. The answer lies in developing a self-reflexivity by
questioning
\quotation{the terms and conditions by which studies of the digital divide define their own mode of questioning}
(Gunkel 2003: 517).

Yet simply recognizing the gradation of potential instances of the
digital divide does not arbitrate its reliance on a binary
structure. Gunkel does not problematize binary thinking, though I
find his justification of its necessity to be problematic because
he magically equates \quotation{Western thought} with
\quotation{meaningful discourse} (508). Yet the point remains:
\quotation{In defining others as deficient, one does not simply provide a neutral expression of inequality}
(508). By separating what is actually a diversity of levels and
modes of connectivity into a dualistic division, the term digital
divide misframes its subject.
\footnote{For Gunkel, the binary mode of thinking is not itself a problem,
but it seems a remaining issue to me.}
Perhaps his division of equating \quote{binary} to \quote{coherent}
and therefore its opposite with \quote{incoherence} is a knowing
example of his suggestion to utilize binary logic to interrogate
the limits of the binary division {\em not} represented but
{\em implanted} by the term digital divide.

The third outstanding issue, then, is the continued reliance of the
digital divide on technodeterminism. In fact, its invocation bears
the weight of accepting by fiat that access to technology
automatically yield social, economic, and, presumably even if never
evoked, spiritual benefits. Gunkel quickly notes that history does
not bear out such an interpretation, nor do statistical surveys of
usage patterns on the internet. The telegraph
\quotation{did not hasten the coming of the kingdom of heaven but supported nationalist aggression and empire-building}
(515). Which is true, though I bristled that Enzensberger was
listed as a member of those who had faith in the capacity of radio
without qualifying his own intense critique of technodeterminism.
The potential of radio was never tested to its natural social
limits, as the internet to some extent has been. This is, after
all, what usage statistics measure. Governments did not trust their
citizens with access to a many-to-many communications mechanism and
the throbbing logic of capitalism provided the perfect excuse to
hand over the reigns to corporations and mandate a mandatory
crippling of all radio devices. Given the progress of the
corporatization of the internet in tandem with a continued push for
copyright control to extend to hardware, Enzensberger's critique
remains as valid as ever. Indeed, a \quote{digital divide} between
access to programmable versus non-programmable devices might one
day emerge. Gunkel's response to the technodeterministic overtones
of the digital divide is similar to his call for self-reflexivity:
he calls for questioning
\quotation{the terms and conditions by which studies of the digital divide define their own mode of questioning.}
(517)

\subject{Chun}

Wendy Hui Kyong Chun's 2006 book
{\em Control and Freedom: Power and paranoia in the age of fiber optics}
critiques many undiscussed and undigested assumptions aboout the
internet. Writing from an antiracist perspective, Chun establishes
a case for examining the ways in which race lies at the core of a
utopian interpretation of cyberspace as a \quote{perfect}
commercial realm. Analyzing rhetoric and visual tropes used by
corporations in their advertising, Chun demonstrates how
\quotation{conceiving race as skin-deep has been crucial to conceiving technology as screen deep}
(Chun 2006: 129). As computers and the internet became increasingly
sold as an \quotation{answer} to dilemmas of race and racism,
racism becomes naturalized---a frame that declares the presence of
intermediary screens as capable of dissolving race has as its
logical conclusion that race, then, is only skin deep. Marketing to
different races at once generates race as a consumer and
pornographic category. The result
\quotation{effectively conceals individual and institutional responsibility for discrimination}
by
\quotation{positing discrimination as a problem that the discriminated must solve}
by gaining access to the internet (132). The internet in this
instance (supposedly) allows those \quote{already marked} to escape
{\em their} differences, making it their own responsibility to do
so.

This side effect is an example of what Chun calls a
\quote{privatization of civil rights}:
\quotation{There is no need any more for battles over discrimination because the Internet can guarantee those rights the state has not been able to provide}
(144). Corporations utilize the (often racialized) issue of the
digital divide as leverage to position themselves as the answer to
the problem. The answer to equality is technology, or specifically,
corporate technology. In marketing towards racial groups, such as
the computer-owning Latino, corporations do not strive to connect
and power those who we think of as marginalized in the terms of a
digital divide. Rather the corporations seek to maximize profit by
targetting those who can {\em already afford} the access. Chun
claims the inevitable side product of race as a consumer class is
race as a pornographic category. This happens through processes of
passing
(\quotation{where one takes on a marked body rather than an unmarked one}
(154)), {\em pornotroping} (where those priveleged enough to have
flesh, and therefore body, engage in psychologically adopting new
flesh), and the closely related \quote{scenes of subjection.} The
last of these is described by Chun:
\quotation{These scenes produce a form of empathy that obliterates difference: self replaces other, the white sel imagines itself the black beaten slave. The other's degradation thus becomes an opportunity for self-reflection, not an event to which one witnesses and testifies}
(136).

Chun's central concern is properly intertwined with issues of race,
but it can also be identified on a more abstract level. For Chun
the internet represents a means of communication that exposes its
users to new vulnerabilities:
\quotation{Invisibly, the Internet turns every spectator into a spectacle}
(130). Information flows two ways, and the corporate processes she
highlights are quite interested in collecting the information you
must necessarily input in order to get a desired output. For
instance, what search term you used or what site you eventually
chose to visit (to name only a few of the banal). Chun discusses
projects by artist-hackers Mongrel that expose, for instance, the
assumed but incorrect \quote{superagency} of a user by producing
\quotation{interfaces and content that are provocative---even offensive---in order to reveal the limits of choice, to reveal the fallacy of the all-powerful, race-free user}
(164). In contrast to many other books on the internet and new
media environments, one of Chun's main goals is to investigate the
\quotation{ramifications and possibilities of vulnerability and connectivity rather than superagency}
(130). From an antiracist perspective this involves {\em not}
avoiding race, taking it from its whitewashed invisibility and
forcing it into the visible. Race must not only be visible, it must
be \quotation{difficult to consume,} arguably a standard attribute
of race to begin with (130).

Chun's answer lies in exposing and discussing the ways in which the
internet does {\em not} enable democracy or provide a race-free
utopia, in discussing the vulnerabilities that come as a result of
our connections. She mentions the possibility that ideas of
\quotation{personal} computers serve to obscure the extent to which
we give away information about ourselves that we might never
consider giving away, if asked for voluntarily rather than
automatically cultivated through the two-way nature of the network.
In her words, there are
\quotation{myriad ways in which cooperation is {\em forced} on us}
(170). I don't read her \quotation{solution} (computers that aren't
perceived as personal) as a request for State mandates and communal
ownership of computer property (she seems much to concerned with
privacy and civil rights to necessarily advocate that). However,
the question is an interesting one. Certainly the ways in which she
problematizes the corporate/governmental tropes of screen-deep
differences is an important contribution to one day living in a
world that is not so dramatically defined by its racial tensions.

{\em [Here goes the unfortunately omnipresent apology. Other matters in my life have unfortunately taken a foreground position in recent weeks. I hate handing in a half-finished paper, but I'd hate it even more if I felt it contained anything half-assed within it. Thus, it is missing some pieces.] }

\startworkscited

Chun, Wendy Hui Kyong.
{\em Control and Freedom: Power and paranoia in the age of fiber optics}.
Cambridge: MIT Press, 2006.

Gunkel, David J.
\quotation{Second thoughts: toward a critique of the digital divide}.
{\em new media \& society}, 5(4): 499--522. 2003.

\stopworkscited

\stoptext
