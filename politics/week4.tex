% Links
\setupcolors[state=start]
\setupinteraction[state=start,color=darkblue]

% ..
\setuppagenumbering[location={footer,left,margin}]  

% paper, layout, etc.
\setuppapersize[A4]
%\setuplayout[width=6.5in,height=10.5in,topspace=0.5in,backspace=1in,
%  header=0.5in,footer=0.5in]
% double space
%\setupinterlinespace[line=5.6ex]
% 1/2 inch indents
\setupindenting[medium]
\indenting[always]

% jagged-right (aligned to the left)
%\setupalign[right]

% font
%\setupbodyfont[rm,12pt]
\usemodule[simplefonts]
\setmainfont[Liberation-Serif]
\setmonofont[inconsolata]

% for long quotes
\definestartstop[longquote][
  before={\indenting[never]
    \setupnarrower[left=0.5in,right=0.5in]
    \startnarrower[2*left,right]},
  after={\stopnarrower
    \indenting[yes]}]

% for heading and header
\def\MLA#1[#2][#3][#4][#5][#6][#7]%
	{\setuppagenumbering[left=#3 ,location={header,right}]
	\indenting[never]
	#2 #3\par#4\par#5\par#6\par\startalignment[middle]#7\stopalignment
	\indenting[yes]}
	
% following hanging indent code (also in workscited) taken from 
%  http://www.ntg.nl/pipermail/ntg-context/2004/005280.html
% [NTG-context] Re: Again: "hanging" for a lot of paragraphs?
%  ~ Patrick Gundlach
\def\hangover{\hangafter=1\hangindent=0.5in}
\definestartstop[workscited][
  before={
    \page[no]
    \indenting[never]
		\setupwhitespace[medium]  
    \startalignment[left]
    \subject{Bibliography}
    \stopalignment
    \bgroup\appendtoks\hangover\to\everypar
    },
  after={
    \egroup
    \indenting[yes]}]


\starttext      

% % % %
% Lazy version of a title.
% % % %
{\tfd What You See With What You Have}
%{\tfb Mapping the Open Source Typesetting Landscape}

% % % %
% Including a list of information for the paper requires 'tabulate'
% % % %
\starttabulate[|l|l|]
	\NC Student:
	\NC John Haltiwanger
\NR
	\NC Paper:
	\NC Week 4
\NR
	\NC Supervisor:
	\NC Thomas Poell
\NR
	\NC Date:
	\NC 24 February 2010
\NR 
\stoptabulate  

\subject{Reaction to Turner}

Fred Turner's text deals primarily with the question of
\quote{digital utopianism.} Where did the sentiment, so common in
the 1990s, that computers represented a new paradigm of
digitally-enabled equality, personal liberation, and utopia
actually come from? Turner's answer lies in the social force that
is one Stewart Brand, founder of the {\em Whole Earth Catalog} and
integral component of a vast number of overlapping social networks.
As a professor of Communications, his focus on Brand's social
networking techniques (extrapolated from the
military-academic-industrial complex's embrace of cybernetic
theory) and the influence of {\em Wired} specifically in shaping
the discourse and understandings of PCs in the 1990s make
sense---Brand, if nothing else, has pushed the potentials of
\quotation{communication} and demonstrated something remarkably
like a successful propaganda campaign. However, as I will
elaborate, there are flaws with this attribution, some of which may
result from Turner's non-technical background.

Inasmuch as Turner's text concerns the question of how computers
came to be seen as tools of personal liberation, his sole-minded
focus on the networks of Stewart Brand overplays their importance.
Steven Levy clearly outlined a moral ethic to which hackers
self-subscribed in his 1984 book {\em Hackers} (Levy 1984,1994).
This morality issues forth declarations of universal access to
computers and information and a judgment system that foreclosed
issues of race, gender, age, and social position in favor of
evaluating fellow hackers based only on code. These positions have
clear relations to (my understanding of) American New Left
ideology, including a predisposition for some degree of socialism.

The origins of the first commercially successful PC (Apple I) in
the activities of Steve Wozniak are intrinsically tied to his
connections to the hackers in the Homebrew Computing Club. Many in
this group of hackers saw communal
\footnote{Yet also very individual at the same time, in the grand tradition
of prestige competition and capitalist resource allocation (some
could afford more components / newer kits than others). An Altair
kit was in the mail to an HBCC member at the time of the first
meeting, for instance.}
development of microcomputers as a moral imperative, a transition
to individual access to computers that likely would never come from
the corporate world. Why would they ever choose to give up their
powerful positions as dominating patriarchs in a world of computer
access defined by its resemblance to a priest class? In fact, when
Hewlett Packard decided whether HP should exercise its rights over
the Apple I (which Wozniak had developed while working a summer job
at the company), it waived its option due to the sheer
incomprehensibility its corporate mind faced at the concept of an
average individual ever wanting their own computer. These hackers
knew the potential of the microcomputer as a tool and endeavored to
ensure it came into existence, yet the relationships of hackers to
machines is complex and probably represent unique
cross-liberations. The hacker hacks often for the sheer thrill---in
other words, the universal liberation known as a high.
\footnote{If there is a moral component, it can be easily exercised by
licensing in a morally accordant format and, likely, through the
nature and utility (scope) of the hack.}
Couldn't the hacking be said to liberate the computer by equipping
it with new modes of potential? Wasn't Wozniak's Apple I a
liberational watershed for all involved? The man gets some money
and a job hacking, the people get PCs, and the microcomputer begins
its reorientation of entire societies. It seems to me that
isolating a liberational view of computers into the result of
Stewart Brand's networking seems exclusionary to the point of
reinforcing Brand's legacy and legitimizing his libertarian
approach. These standpoints are of course valid, so long as they
are not presented in the absence of other clear lineages of
liberational moralities that stand to weaken an argument that
traces the popular liberatory view of computers to Brand and his
networks.
\footnote{Brand once gave away 20,000 dollars to an audience and told them to
decide who should be its recipient. Fred Moore, an early member of
the then-embryonic HBCC, held a moral stance against money and
spent the entire night campaigning against even having a vote about
voting about the money, trying to convince people to organize and
stay in touch as a network, and passing around a petition stating
\quotation{We feel the union of people here tonight is more important than money, a greater resource.}
In the end, the last of the audience gave the money to Moore. I
find this an interesting contrast to Brand's apparrent approach to
social networking (Levy 1994: 156--7).}
In Brand's own words,
\quotation{No other group that I know of has set out to liberate a technology and succeeded}
(Levy 1994: 356).

This is not to undermine the importance of Turner's elucidation of
the effects that Brand's cybernetically inspired networks {\em did}
have. Stewart Brand and associates (the SBAA?) truly did see the
utility potential of the tools at their disposal. Deploying
cybernetic theory to entrepeneurship enabled the social networks
that would one day spread {\em a particular form of} computer
liberation ideology. Turner's frequent invocation of LSD as a
technology for communal organization (in the hands of the New
Communalists) and personal liberation (in general) does directly
mirror the rhetoric, and even experience, of the early Internet
era. There are also important unexplored parallels in the
criminalization of LSD and file sharing. At their most idealistic,
both enable new and unique social potentials and understandings.
Criminalized, these potentials are marginalized and co-opted into a
caricature of its now socially illegitimate (taboo and dangerous)
technology.

By demonstrating that Brand, et al., were instrumental in steering
and shaping the language, and thus the conceptual framings, of the
Internet's Big Bang, Turner exposes them as the agents of the
deregulation that has resulted in oligolopic telecom assemblage
ruling the wires. Forced line-sharing, also known as the principle
of carrier neutrality, is the main driving force behind Japan's
lead in broadband connectivity and is a model that has been
successfully replicated anywhere that is not too corrupt, or just
plain stupid, to try. Yochai Benkler identifies it as the most
effective, yet least politically likely, means to ensure network
neutrality in the United States (Benkler 2005). This clearly
illustrates that when Turner warns that information workers that if
they
\quotation{buy into the notion that computers and the network economy will bring about a peer-to-peer utopia, as many still do, they run the risk of perpetuating the forms of suffering and exclusion that plagued the back-to-the-landers,}
he may in fact be more right than he knows (Turner 2006: 257). It
certainly echoes Enzensberger's assertion that a techno-determinism
that relies on machines and
{\em the simple existence of those machines} to provide paths to
liberation will result in no liberation at all. The path must be blazed to freedom before those in
control can lay concrete slabs in the way. It is hackers, not
Brand's networks, who achieved this trailblazing and thus unlocked
the microcomputer for the masses. Brand may have been instrumental
in publicizing (his own particular vision of) digital utopianism,
but his involvment in constructing the core architecture to support
that vision is minimal to the extreme.

The libertarian, \quotation{market is nature} viewpoint that Brand
embeds in his networking continues to dominate not only the
language of technology and its potential, but of the entire
discourse underpinning America's economy today, with
{\em universally similar results as those Turner warns based on the elitist communard experience}.
Turner's narration of the role in which Brand, et al. plays in
shaping these social conceptions not only provides solid entities
and decisions with which to critique the Californian Ideology, it
also illuminates social networking techniques based on cybernetic
theory that could prove useful in countering that ideology.

\subject{Reactions to Beck / Latour}

Ulrich Beck's {\em World at Risk} attempts to explain a new theory
of sociology, or at least a new sociological understanding, based
on the idea that a new stage of modernity has arrived (Beck 2009).
This new stage of modernity is defined by widespread risks that are
increasingly difficult, if not impossible, to assess in terms of
responsibility. While the first era of modernity is defined by its
ignorance of the risks it was generating, the second era knows of
the risks and therefore seeks to incorporate them (or not) into its
understanding and practice.

A significant countervailing force to this trend of actully
managing risks instead of skirting blame and responsibility lies in
what Beck calls \quote{relations of definition.} Similar to Marx's
\quote{relations of domination,} Beck's explication of these
relations demonstrates that the capacity to define a thing (here,
risk) allows control over society's understanding of that thing. As
the degree and scope of pollution increases, the likelihood for a
polluter to be punished decreases (30). Questions of who determines
what a risk is, and what to do about it, are answered through these
relations of definition (or {\em not} answered, as the case may
increasingly be).

To be perfectly honest, the selections of Beck did not give me a
good sense {\em at all} of what the man is promoting. That the
world is risky? Derrick Jensen's {\em Endgame} provides a much more
robust articulation of the actual problems facing it. That the hope
lies in sub-politics, a la his case study of a 1995 Greenpeace
action? (96) The text bounces around with the action of a medicine
ball, weighed down by its cumbersome associations with the
\quote{social} and \quote{modernity.} As Latour rebuts gracefully,
the former is invalid and the latter impossibly hard to define.
Reading Latour's response brought some of Beck's thinking to light,
but much of that I failed to see in the readings from
{\em World at Risk}.

For Latour, the proof is in the pudding. When asked to look into
Beck's \quotation{reflexive modernity,} which he mercifully calls
re-modernization, Latour approaches with his characteristic
proof-focused analysis (Latour 2003). By comparing the relative
approaches for proving 're-modernization' and ANT, Latour slowly
eviscerates Beck's argument.
\footnote{I'd like to reiterate that I did {\em not} necessarily see this
argument in the text we read.}
For ANT (and, specifically, Latour's own persepctive), the initial
step would be to prove that a perspective shift had occurred in the
understanding and appreciation of
risks---{\em because modernity has never accurately described itself},
the mere act of describing itself differently would prove a) there
never was an acceptable vision of risk in modernity, and b) that a
new master narrative had been introduced. For 're-modernization'
the task is to prove that such a second stage of modernity actually
incorporates \quotation{very different characteristics} (39). In
other words,
\quotation{for Beck and his group the proofs have to be in the {\em substance} of the phenomena they study, for me only in the {\em collective interpretation} given to phenomena which, all along, have never been modern}
(39). The difference in the two approaches is quite large---ANT's
reliance on an {\em actual litmus} for demonstrating the accuracy
of a theory allows it to easily stomp on Beck's social rhetoric. In
fact, Latour uses the litmuses for both approaches to show the
extensive work necessary {\em to even test} Beck's assertions.
Ultimately Latour does not condemn Beck's 're-modernization' but
rather declares it a potential tool in shaping our understanding
and approach {\em going forward}.
\quotation{Re-modernization might not descrie what has already happened, but it can offer a powerful lever to make new things happen}
(46).

{\em Note: I would like to take this moment to say that I am not happy with this paper. While I think my critique of Turner is not only biting, but relevant, I have to say that I a) did not understand or appreciate Beck, and b) mis-managed time and did not end up reading Marres. If Marres' work actually connected with my case study, it would be the first to do so non-tangentially, thus the absence of the case study here as well. Next week I will get this timing right or die trying.}

\startworkscited

Beck, Ulrich. {\em World at Risk}. Cambridge: Polity Press, 2009.
24--46 (chapter 2), 81--108 (chapter 5).

Benkler, Yochai. {\em The Wealth of Networks}. New Haven: Yale
University Press, 2006.

Latour, Bruno.
\quotation{Is {\em Re}-modernization Occurring---And If So, How to Prove It? A Commentary on Ulrich Beck.}
{\em Theory, Culture, \& Society: Explorations in Critical Social Science},
20, no. 2 (2003): 35--48.

Levy, Steven. {\em Hackers: Heros of the computer revolution}. New
York: Dell Publications, 1994. Originally published 1984.

Turner, Fred. {\em From Counterculture to Cyberculture}. Chicago:
University of Chicago Press. 2006. 1--9 (Introduction), 175--262
(chapters: 6--8).

\stopworkscited

\stoptext
