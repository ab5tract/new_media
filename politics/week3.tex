% Links
\setupcolors[state=start]
\setupinteraction[state=start,color=darkblue]

% ..
\setuppagenumbering[location={footer,left,margin}]  

% paper, layout, etc.
\setuppapersize[A4]
%\setuplayout[width=6.5in,height=10.5in,topspace=0.5in,backspace=1in,
%  header=0.5in,footer=0.5in]
% double space
%\setupinterlinespace[line=5.6ex]
% 1/2 inch indents
\setupindenting[medium]
\indenting[always]
% CSL is a new form of handling bibliographic data and citations. It also apparently   http://xbiblio.sourceforge.net/csl/
% jagged-right (aligned to the left)
%\setupalign[right]
% font
%\setupbodyfont[rm,12pt]
\usemodule[simplefonts]
\setmainfont[Liberation-Serif]
\setmonofont[inconsolata]

% for long quotes
\definestartstop[longquote][
  before={\indenting[never]
    \setupnarrower[left=0.5in,right=0.5in]
    \startnarrower[2*left,right]},
  after={\stopnarrower
    \indenting[yes]}]

% for heading and header
\def\MLA#1[#2][#3][#4][#5][#6][#7]%
	{\setuppagenumbering[left=#3 ,location={header,right}]
	\indenting[never]
	#2 #3\par#4\par#5\par#6\par\startalignment[middle]#7\stopalignment
	\indenting[yes]}
	
% following hanging indent code (also in workscited) taken from 
%  http://www.ntg.nl/pipermail/ntg-context/2004/005280.html
% [NTG-context] Re: Again: "hanging" for a lot of paragraphs?
%  ~ Patrick Gundlach
\def\hangover{\hangafter=1\hangindent=0.5in}
\definestartstop[workscited][
  before={
    \page[no]
    \indenting[never]
    \startalignment[left]
    \subject{Bibliography}
    \stopalignment
    \bgroup\appendtoks\hangover\to\everypar
    },
  after={
    \egroup
    \indenting[yes]}]


\starttext    


{\tfd Information}

\starttabulate[|l|l|]
\NC Student:
\NC John Haltiwanger
\NR
\NC Paper:
\NC Week 3: Information
\NR
\NC Supervisor:
\NC Thomas Poell
\NR
\NC Date:
\NC 16 February 2010
\NR 
\stoptabulate  

\subject{Free Culture}

The readings from this week cover the diverse range of issues that
result from a society's approach to \quote{information.} For
Lawrence Lessig, the idea of intellectual property as currently
conceived, in the United States at least, is often the result of
not logical decision making, but a process of \quote{capture} that
leads to governments favoring the interests of those entrenched
over the potential new forms that could affect that entrenchment
(Lessig 2005: 6). This cyclical pattern has led to drastic
extensions of copyright as well as significant expansions of what
falls under intellectual property protections (software patents,
technique patents, database patents in the EU,\ldots{}). The net
result of this transformation of IP is, according to Lessig, a
strangling of culture. Lessig's reaction was to propose the
Creative Commons, a legal hack, in the tradition of the GNU Public
License but different in a key way, that enables copyright holders
to specify the terms under which their property can be used by
others (Lessig 2005: 283). This is particularly necessary because
the expanding rights to propertize the intellect is on-going and
has particular implications for an age defined by new ways to
access information. A quote that Lessig offers to explain the
central thrust of his book explains it
\quotation{that while the Internet has indeed produced something fantastic and new, our government, pushed by big media to respond to this \quote{something new,} is destroying something very old}
(Lessig 2005: 13). Lessig, then, calls for reform of IP laws and
institutions from within legislative processes as well as the
changes to our understanding and approach to information
represented by the Creative Commons license. He does not, however,
object to the general idea of the \quote{intellectual} as
\quote{property.} In fact, his reforms suggestions, backed by his
case studies, are all firmly within the current system---they just
alter the landscape to afford more sensible treatment of
intellectual property and its \quotation{holders.}

\subject{Autonomous Commons}

Finishing such a logical, pragmatic treatment as Lessig delivers in
{\em Free Culture}, it could seem obvious that there is little room
for coherent criticism. This expectation is necessarily shattered
by Matteo Pasquinelli's
\quotation{The Ideology of Free Culture and the Grammar of Sabotage.}
Pasquinelli begins by grounding society as
\quotation{the management of that constantly reincarnates itself in new forms of state and economy,}
a notion he has paraphrased from Bataille (Pasquinelli 2008: 2).
Noting that media all feed on this same \quote{excess of energy,}
he questions whether that energy has ever been effectively
described.
\quotation{The energy of semiotic flows is not the energy of material and economic flows. They interact but not in a symmetrical and specular way}
(2). Even within energetic interactions of the same type (semiotic
or material) there is asymmetry, leading to Pasquinelli's
invocation of the parasite. This concept, developed by Michel
Serres, exists to express that
\quotation{there is never an equal exchange of energy but always a parisite stealing energy and feeding on another organism}
(3). There are {\em immaterial parasites}, Pasquinelli asserts,
which establish themselves through providing infrastructure for
collaboration or communication only to accumulate the energy
produced along thos channels and invest
\quotation{in favour of its physical substratum} (here I am
assuming Pasquinelli means the corporation behind the parasite).
Many examples are given, but the most striking to me personally is
the exploitation of Free Software to sell proprietary hardware.
(The views expressed by Pasquinelli here converge with other recent
readings that have significantly changed the way I would
theoretically treat the example of \useURL[1][http://github.com][][Github]\from[1], a proprietary Web
2.0 application running on Free Software, than I did in my Masters
of Media post \useURL[2][http://mastersofmedia.hum.uva.nl/2009/11/01/git-virtue-github-and-commons-based-peer-production/][][Git Virtue\?\: Github and Commons-based Peer Production]\from[2].
While it may still spread virtue in a Benkler-ian sense, its
parasitical nature needs further articulation. The controversial
aspects of hosting FLOSS projects on a proprietary platform are
much starker in light of Pasquinelli's presentation.)
\quotation{[W]here does profit end up in the so-called Free Society?}
he wonders (5).

Pasquinelli scoffs at the paradox behind Lessig's proposal of a new
tax to compensate artists, making me curious about how he feels
about the proposals for a universal livable wage (not tied to
productivity) seen in the work of Michel Bauwens and Christoph
Spehr. The Creative Commons license fails to consider the frictions
that digital copies create in real spaces governed by limited
resources. Not only this, CC fails to follow a key line of
reasoning behind Free Software licenses: rather than granting
rights to users, rights are reserved for copyright holders. Worse
still is the naivete that comes with ignoring the issue of
surplus-value extraction, as evidenced by Free Software's easy
co-option into proprietary business structures and technological
infrastructures alike. The answer may lie in a new form of
licensing, proposed by Dmytri Kleiner, in which class
considerations are taken into account, ie worker co-ops can have
access to licensed works for free but for profit enterprises would
be forced to pay.
\footnote{It would be extremely interesting to get Richard Stallman's opinion
of this. Indeed, I am trying to conceptualize questions to send him
as an interview that would be compelling enough to invoke a
response.}
If software were to be licensed this way, the issue of immaterial
parasites sucking on the energy of Free Software would be
drastically altered (and one might imagine that some of the
volunteers making software might actually see compensation for it
under such a scheme). This is part of an \quote{autonomous commons}
that would pay considerable attention to issues arising from
passive/personal consumption vs.~productive uses, the role of the
commons in relation to the economic/corporate forces that would
exploit it, the asymmetry of the immaterial versus the material,
and the dynamic hybridity of the commons (including its need for
being built and for being defended) (6).
\quotation{If someone cannot pasture cows and produce milk on it, it's not a real commons}
(7). This reintegration with the ancient form of commons (the
\quotation{something very old} that Lessig mentions) allows
Pasquinelli to begin discussing rent.

It was forced rent-paying for use of common land that fed the
parasitical entities of nobility. The rent aspect of the commons
has not disappeared, however, and therefore it manifests over the
network as well as the land commons (8). To fight back requires
sabotage, Pasquinelli argues, echoing Spehr's idea of the freedom
of refusal and Geert Lovink's \quote{not-working.} Like Galloway
and Thacker's virus, however, Pasquinelli only has one example upon
which the idea can immanentize: \quotation{illegal} file-sharing.
That file-sharing is a form of sabotage can perhaps be corroborated
by the \quotation{anti} tone of defenders of downloading. For a
proposal that he calls
\quotation{the only possible gesture to defend the commons,} I
would appreciated Pasquinelli developing this idea of sabotage into
tangible strategies beyond merely downloading copyrighted material
(12). Does he mean blowing up source code repositories when one
feels betrayed by the project?

\subject{Identification}

Switching gears from modes of handling information as property to
the ways in which information is remolding the concepts of identity
and citizenship is David Lyon's articulation of the effects of the
Information Age on identification (Lyons 2009). Identification is
increasingly defined by its reliance on what Lyon's terms
\quotation{stretched screens.} Screens call up information from a
potentially global network of databases on individuals, and this
information is often taken as a more valid explanation of personal
history than what those individuals can explain for themselves.
This concept of stretched screens allows Lyons to describe four
undercurrent concepts:
{\em liquidity, governing by identification, identification rotocols, interoperability, and ubiquitous computing}.
Liquidity calls attention to the increasing mobility of individuals
across the globe. Filtered through the lens of
\quotation{national security,} this aspect of liquidity is seen as
a necessary liability: the labor must continue to flow, but that
flow must be tightly \quote{triaged} to exclude undesirable
individuals. This filtering process will no longer be precluded to
borders and checkpoints, however. The disconnection of
identification from space (through these globally distributed
databases) means that the
\quotation{border now seems to travel with them [ordinary people] in their portable ID documents; the border really does seem to be everywhere}
(90). The idea of exclusion resurfaces in the discussion of
governing by identification, where difference is converted to
Otherness in a process that exacerbates already existing lines of
tension along race, class, and gender (also called the
\quote{marginal}, a word we speak to delineate an exclusion).
Exclusion is here enabled by identification protocols---the core of
identification in the Information Age by definition. When tied to
the yet-dormant potential of ubiquitous computing, the Marginalized
can literally be transformed into the Excluded with the ease of
protocological control.
\footnote{I find the case of Google's new patent
on techniques of (data) exclusion based on geographical IP data and
national licensing policies to be most telling in this regard (U.S
Patent Office 2010). Your location in physical space can now
determine the privilege of your access to Google.}
Doors can open for the privileged one second can remained sealed to
the unprivilege the next, physically literalizing a metaphor often
invoked in discussions of RCG (race/class/gender) privilege.
\footnote{Please excuse this new acronym. Anything that makes privilege less
awkward to discuss is something to be sought, in my opinion.}
Likewise, interoperability is the extension of historical
tendencies.
\quotation{In the case of international policing, the standardization of identification techniques was also a central goal from the start}
(Lyons 2008: 99). After 9/11 the screens have stretched to check
databases in diverse, distant locations. As a result
\quote{border checking} (a crucial and ancient aspect of governing
by identification) often occur \quote{upstream} of a mobilized
individual as their data is checked against databases in the
destination country while the individual is still in route (98).
Interoperability is closely tied to protocol, and thus involves
many non-governmental entities in technology companies and
standards bodies.

The tendency to desire finer-grained methods of identification is
shared between commercial interests and governments. These already
convergent assemblages could conceivably produce a
\quote{banopticon} through which consumer will presumably no longer
be a blanket term for citizen, instead becoming a signifier for the
privileged as those who are unable to participate in bourgeois
consumerism find excluded from wherever their RCG is deemed
undesirable. This process is called \quote{social sorting} and it
is a key element of government by identity. The lack of standards
and the fact that ubiquitous computing is not yet a reality give
Lyons hope that a significant social response can counterbalance
the convergence of these powerful assemblages.

\subject{Citizen Journalism}

Any such response will require the organization and spread of
knowledge about the issues surrounding identification. To that end,
Bruns' analysis of citizen journalism provides a means of examining
the channels through which such organization and dissemination can
flow (Bruns 2008). Bruns begins the discussion of citizen
journalism by acknowledging not only its similarities to but its
wholesale reliance on open source software as the platforms that
support the phenomenon run primarily on FLOSS (actual statistics
would be quite interesting). Additionally, citizen journalism
resembles FLOSS development practices because of its reliance on
collaborative filtering and the challenges it raises towards the
industrial form of news production. As such, citizen journalism is
heterachical rather than hierarchical, concerned not with producing
discrete, individual (one could say commodified) versions of events
but rather collaboratively digesting the implications and context
of those events. This engaging with
{\em unfinished artefacts, continuing process} signals citizen
journalism's classification by Bruns as a form \quote{produsage,} a
term he positions in opposition to production as a new form of
productive expression typified by the increasing presence of
\quote{the people formerly known as the audience} in content
creation. The lines have blurred between the old gatekeepers of
industrial news, still a crucial component in citizen journalism
besides the antipathy between the two, and the citizen journalists
in terms of who is trusted to deliver the news. Bruns notes that
the two each have important roles to engage with: the former should
produce only factually accurate, flat reportage while the latter
continues its role as the collective digestive system for events
(92).

Unconstrained by space requirements and the format of the
supposedly definitive \quote{single packet} news story, not to
mention political and commercial elements unique to industrial news
production, citizen journalism allows the flowering of debate
around topics that are unpallateable to the old regime. The
proscribed role of citizen journalism is gatewatching, that is,
observing the misrepresentations and omissions in the industrial
newsfeed, though Bruns notes a tendency towards original reporting
and new \quotation{Pro-Am} hybrid news organizations. This is done
through a process of aggregating personally interesting information
and sharing it (and usually an opinion) with others whose responses
are an integral part of assessing that information. No one is
assumed to have all the knowledge, freeing individuals from a
responsibility to speak from anything besides personal experience.
Single \quotation{articles} within the unfolding of the discussion
are nothing more than the means of furthering discussion by
offering an individual perspective that others may or may not agree
with, may or may not have something imperative to contribute to
that perspective. In this way citizen journalism can be seen as
probabalistic---it strives for quality through the collectivization
of opinion, context, and insight. Not all is worthy, but the worthy
will often arise from the cacaphony. Methods within individual
sites such as karma points are a means for filtering up the worthy,
but even the anarchic relations of the blogosphere reveal the
importance of {\em continued} quality in maintaining the reputation
of a blog. Like Pasquinelli, Bruns believes it is important for
individuals be able to profit from contributions that rely on a
collective commons. However, his analysis does not address the
issue of surplus value extraction. The realm of citizen journalism
seems a ripe field for investigating the issue of parasites.

\subject{Case Study}

That citizen journalism represents a new mode of news
making/digestion that gives especially welcome space to voices and
opinions neglected by mainstream media is exemplified, for me, in
the site {\em Racialicious}
(\useURL[4][http://www.racialicious.com][][racialicous.com]\from[4]).
The site hosts a continuous discussion of pop culture from an
anti-racist perspective, gatewatching the media's (mis)portrayal of
race from a strongly moral and critical perspective. An interesting
case arose in January 2010 which began with the editors of
{\em Essence} magazine choosing a picture of Reggie Bush, a black
man in a multiracial relationship, on the cover of its
\quotation{Black Men, Love \& Relationships} issue and
\quotation{ended} with a sigh of relief that the corresponding
explosion of debate within the black blogosphere was not picked up
and reported by the mainstream media (Peterson 2010). {\em Essence}
is a magazine that markets specifically to the black female
community, sparking a conversation about the appropriateness of
highlighting a multiracial relationship on the cover of an issue
about \quotation{black love.} Explaining the blog's posting about
the issue, Latoya Peterson states that
\quotation{talking about \quote{the situation within the black female community} isn't really what we do since most of those perceptions are based in stereotypes about black women. However, what is compelling about the whole situation is how conversations about interracial dating play upon stereotypes and deeply held convictions, that tend to drown out any other type of commentary.}

In other words, it is precisely the dynamic, {\em social} process
itself that constitutes the news that {\em Racialicious} is
reporting on, or at least one instantiation of that process. The
other piece of the news reported is, intriguingly, that the debate
around the cover {\em didn't become news} in the industrial media.
This is precisely because the \quote{single-packet} format would
inevitable compress, through institutionalized race and gender
stereotypes, into a story about how
\quotation{black women hate interracial dating!} Precisely because
the debate was not subjected to such formative reduction, it is
allowed to become nothing more (or, crucially, {\em less}) than the
collective expression of itself. {\em Racialicious} performs an
honorable, and perhaps all-too-absent, service by presenting an
overview of the very diverse reactions of commenters on various
pages relating to the issue, thus chronicling the unfolding of
collaborative opinion sharing. No less than twelve distinct
viewpoints receive re-articulation, demonstrating the destructive
capacity of the \quote{single packet} format in the context of
complex social situations.

\startworkscited

Bruns, Axel.
{\em Blogs, Wikipedia, Second Life, and Beyond: From Production to Produsage}.
New York: Peter Lang, 2008. 69--100, 171--198.

Lessig, Lawrence.
{\em Free Cultue: How Big Media Uses Technology and the Law to Lock Down Culture and Control Creativity}.
New York: Penguin Press, 2004. 1--13 (Introduction), 274--306
(Afterword)

Lyon, David. {\em Identifying Citizens. ID Cards as Surveillance}.
Cambridge: Polity, 2009. 84--109 (Chapter 4), 131--155 (Chapter 6).

Pasquinelli, Matteo.
\quotation{The Ideology of Free Culture and the Grammar of Sabotage.}
{\em Education in the Creative} Economy: Knowledge and Learning in
the Age of Innovation\letterunderscore{}. Eds. D. Araya, and
M.A.~Peters. New York: Peter Lang, 2010.

Peterson, Latoya.
\quotation{{\em Essence} Magazine Accidentally Steps Into an Intra/Interracial Dating Minefield.}
{\em Racialicious}." 22 January 2010.
\letterless{}\useURL[5][http://www.racialicious.com/2010/01/12/essence-magazine-accidentally-steps-into-an-intrainterracial-dating-minefield/][][http://www.racialicious.com/2010/01/12/essence-magazine-accidentally-steps-into-an-intrainterracial-dating-minefield/]\from[5]\lettermore{}

U.S. Patent Office.
\quotation{Variable user interface based on document access privileges.}
USPTO Patent Full-Text and Image Database. 16 February 2010.
\letterless{}\useURL[6][http://patft.uspto.gov/netacgi/nph-Parser?Sect1=PTO1&Sect2=HITOFF&d=PALL&p=1&u=\%2Fnetahtml\%2FPTO\%2Fsrchnum.htm&r=1&f=G&l=50&s1=7,664,751.PN.&OS=PN/7,664,751&RS=PN/7,664,751][][http://patft.uspto.gov/netacgi/nph-Parser?Sect1=PTO1\&Sect2=HITOFF\&d=PALL\&p=1\&u=\%2Fnetahtml\%2FPTO\%2Fsrchnum.htm\&r=1\&f=G\&l=50\&s1=7,664,751.PN.\&OS=PN/7,664,751\&RS=PN/7,664,751]\from[6]\lettermore{}

\stopworkscited

\stoptext
