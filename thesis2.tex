% Thesis proposal 2.0
% 22 . I . 010
% John Haltiwanger

\setuppapersize[A4]
\setuppagenumbering[location={header,right,margin}]
\setupindenting[small]                                 
\indenting[always]

\usemodule[simplefonts]
\setmainfont[Liberation-Serif]

\starttext
\title{Mapping Open Source Typhographical Software}
\starttabulate[|l|]
\NC {\it New Media Masters Thesis Proposal} \NR
\NC {\it John Haltiwanger (john.haltiwanger@gmail.com)} \NR
\NC {\it Supervisor: Richard Rogers} \NR
\NC {\it 22 January 2010} \NR
\stoptabulate

\subject{A brief justification for an interdisciplinary software study}
\quotation{Software studies} has become a buzz word in new media programs.\footnote{It is worthy to note that other buzz words that seem to convey much the same message exist, such as \quote{critical code studies}. A tertiary goal of this thesis will then be to explore the different labels/angles that these similar initiatives are engaged in, hopefully clarifying the landscape of software studies to some extent.} In this approach, a piece of software is studied as a transformative object that encodes cultural practices in it's formulations. Lev Manovich, for instance, produced an essay titled \quotation{Import/Export} for {\it Software Studies: a lexicon} in which he investigates the confluence of visual production software around a new shared vocabulary. Though the fields of cinema, graphic design, and animation all emerged historically with divergent vocabularies based around real variances in production, their vocabularies have been largely rewritten due to the emergence of software that can share seamlessly the data formats that store the objects of the respective fields' outputs. This has enabled a new hybridity that now defines contemporary visual media culture. This observation is hardly rare, however, as that is the common feature of software studies: elucidating hidden cultural modes and ramifications that are obscured (and encoded) by software and software interfaces.

That software shapes our culture and our artistic vocabularies, then, is taken as a given. The concerns to be addressed in this thesis revolve around digital typography, and specifically those modes of typography to which users have access to the program and the source code. Focusing solely on open source software enables not only studies at the source code level (through source code visualization tools, among others), it also allows the state of free typography to be critiqued in relation to itself. Deficiencies in open source and its typographical software can and do exist and it is thought that mapping them without comparison to proprietary tools is the only means to avoiding that all-too-common trap of open source: keeping up, rather than innovating. It is not necessarily true that no proprietary software will be mentioned, but rather that no proprietary software will be an object of this study. Narrowing the scope to only open source software also enables a discussion of peer-to-peer production and governance models articulated by Michel Bauwens.

Furthermore, the fact that software can cause cultural shifts is taken to heart and a second, {\it operational} component of the thesis is proposed. This software aims to implement {\it gradated presence}, a form of \quote{counter-protocological adaptation} \footnote{\quote{Gradated presence} was proposed in my theories paper \quotation{Gradation, Aggregation, and Handshakes: Towards an articulation of counter-protocological adaptations for the organization of networks,} currently undergoing revision.} that relies on exploring new modes of human interaction and network organization through the deliberate obfuscation of individual identification at an interface level. The purpose is to develop a web application that will serve as a platform for a \quote{pure-peer} reviewedjournal. One of its key features will be the utilization of a plain-text markup language, enabling atomic change tracking using the {\tt git} version control system as well as multiple file output options--not least of which will be PDFs characterized by a standard (community established) style that unifies that platform's output with a distinct and identifiable aesthetic.

\subject{Digital typography}

This study originally aimed to focus solely on digital typesetting with the venerable {\TeX} engine, though specifically through its most advanced incarnation: {\ConTeXt}. {\ConTeXt} can generate complicated typesettings that include linking structures and even animation within a PDF file. However, as the parameters of the operational component slightly shifted (the initial version will most likely not support automated generation of PDFs generated with {\ConTeXt}), it suddenly became interesting to compare different typesetting options. Now the discussion can include HTML and CSS, OpenOffice.org Writer, Scribus, and the new Sophie tool from the Institute for the Future of the Book, as well as a disctinction between text editors and typographical software and their relations with various tools (at times it stands in something akin to opposition, as in {\tt vim} and Writer, and in others in a quite complimentary mode, as in {\tt vim} and {\ConTeXt}.

As N. Katherine Hayles has discussed, the materialities of electronic text differ from previous modes, most specifically in the intermediation that occurs as a result of dynamic hierarchies and fluid analogies (Hayles 2008). Typographer Robin Kinross distinguishes a \quote{fundamental character of typography}--\quotation{writing is a single process, while printing is at least two: composition and presswork} (Kinross 2004). What then can be identified as \quote{fundamental character(s)} of documents generated with open source digital typographical software? They seem to differ in important procedural ways---{\TeX}/{\ConTeXt} documents exist as: source code (plain text with mark up), temporary files and memory heaps during compilation, a print-ready document such as PDF or PostScript, and perhaps as a printed physical copy; meanwhile OpenOffice.org documents exist as typography and text mixed within a single file format that can be edited in a WYSIWYG environment. What practices and vocabularies are encoded, obfuscated, or neglected in the difference between these two approaches? What other approaches exist, for that matter? Along these lines exist such research as Jeannette Hofmann's \quotation{Writers, Texts, and Writing Acts: Gendered User Images in Word Processing Software,} which investigates the gendering of word processing at the levels of software and usage patterns. Do to prevailing gender inequality within open source, it behooves an examination of open source typographical software to incorporate identity politics.

The purpose of this theoretical component, then, is to map the qualities and variance amongst open source typographical software in order to a) illuminate the landscape in a critical mode that is currently rare in the development of open source software, and b) to point to new potentialities and even interfaces to address shortcomings or enhance qualities unique to open source software.

\subject{Preliminary bibliography}
\startitemize[5]
	\item Matthew Fuller, et al.'s {\it Software Studies: a lexicon} (2008) will be used to incorporate the current vocabulary of software studies.
	\item N. Katherine Hayle's {\it Electronic Literature} (2008) will allow the importing of her framework of intermediation as well as discussing the qualities of electronic text in a literary sense.
	\item Robin Kinross' {\it Modern typography: an essay in critical history} (2004) provides a critical historical framework for understanding typography.
	\item Wendy Hui Kyong Chun's {\it Control and Freedom: Power and paranoia in the age of fiber optics} (2006) will be utilized in order to incorporate identity politics into the discussion.
	\item Lisa Gitelman's {\it Always Already New: Media, History, and the Data of Culture} (2008) provides a framework for performing media history that will prove useful in placing digital typography within its historical lineage.
	\item Lev Manovich's {\it Software Takes Command} provides additional vocabulary and critical development of software studies.
\stopitemize

\subject{{\bf The operational component}}

The operational component of this thesis emerged as the result of Geert Lovink's proposal for a new \quote{post-peer} review journal. This captured my imagination as I envisioned it tied to version control software, {\ConTeXt}, and the {\it gradated presence} adaptation. While it is now no longer certain that {\ConTeXt} integration is a sane goal for the deadline parameters of the project, the issue of a standardized \quotation{letter-head} that would enable all documents published through the platform to attain a professional aesthetic quality and a recognizable aesthetic form remains central. For a contextual contrast, imagine an educational platform that not only produces a unified look (including a single format for bibliographies!) but also allows for atomic tracking of contributions to a single document: group projects can finally be inspected on the level of actual contribution. This is just one small potential shift that this software could enable. The gradated presence adaptation is specifically oriented towards centering collaboration around its objects, rather than the contributors. Meanwhile version control allows for atomic tracking of contributions. The system could be configured, for instance, to allow for only nameless contribution only to the point of publication, at which point the identites of all contributors could be revealed. Alternatively the contributors could decide to never reveal their identities. It is felt that operating such a network would necessarily catalyze the development of new collaborative vocabularies and processes, one of which is a \quote{pure peer} review process that relies on gradated presence to level traditional hierarchization by centering the information rather than the fact this information \quote{belongs} to this or that person.
                                                                                                                                     
The system could also be used without collaboration. That is, a local instance can be run and a user can take advantage of the version control and typographic systems. In that sense it represents another mode of open source typographical software. The platform will be coded in the Ruby programming language and the Waves web application framework, will leverage the {\tt git} distributed version control system, and will likely utilize Prawn for PDF generation and the Haskell-derived {\tt pandoc} utility for translating the plain text markup to a multitude of formats.

\stoptext
